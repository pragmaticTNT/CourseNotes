%
% This is a borrowed LaTeX template file for lecture notes for CS267,
% Applications of Parallel Computing, UCBerkeley EECS Department.
% Now being used for the Coursera cryptography course, part one.
%

\documentclass[twoside]{article}
\setlength{\oddsidemargin}{0.25 in}
\setlength{\evensidemargin}{-0.25 in}
\setlength{\topmargin}{-0.6 in}
\setlength{\textwidth}{6.5 in}
\setlength{\textheight}{8.5 in}
\setlength{\headsep}{0.75 in}
\setlength{\parindent}{0 in}
\setlength{\parskip}{0.1 in}

%
% ADD PACKAGES here:
%

\usepackage{amsmath,amsfonts,amssymb, graphicx}
\usepackage{color}
\usepackage{comment}
\usepackage{mathtools}

%
% The following commands set up the lecnum (lecture number)
% counter and make various numbering schemes work relative
% to the lecture number.
%
\newcounter{lecnum}
\renewcommand{\thepage}{\thelecnum-\arabic{page}}
\renewcommand{\thesection}{\thelecnum.\arabic{section}}
\renewcommand{\theequation}{\thelecnum.\arabic{equation}}
\renewcommand{\thefigure}{\thelecnum.\arabic{figure}}
\renewcommand{\thetable}{\thelecnum.\arabic{table}}

%
% The following macro is used to generate the header.
%
\newcommand{\lecture}[4]{
   \pagestyle{myheadings}
   \thispagestyle{plain}
   \newpage
   \setcounter{lecnum}{#1}
   \setcounter{page}{1}
   \noindent
   \begin{center}
   \framebox{
      \vbox{\vspace{2mm}
    \hbox to 6.28in { {\bf CSC2420: Algorithm Design, Analysis and Theory
	\hfill Fall 2017} }
       \vspace{4mm}
       \hbox to 6.28in { {\Large \hfill Lecture #1: #2  \hfill} }
       \vspace{2mm}
       \hbox to 6.28in { {\it Lecturer: #3 \hfill Scribe: #4} }
      \vspace{2mm}}
   }
   \end{center}
   \markboth{Lecture #1: #2}{Lecture #1: #2}

   %{\bf Note}: {\it LaTeX template courtesy of UC Berkeley EECS dept.}

   %{\bf Disclaimer}: {\it These notes have not been subjected to the
   %usual scrutiny reserved for formal publications.  They may be distributed
   %outside this class only with the permission of the Instructor.}
   \vspace*{4mm}
}
%
% Convention for citations is authors' initials followed by the year.
% For example, to cite a paper by Leighton and Maggs you would type
% \cite{LM89}, and to cite a paper by Strassen you would type \cite{S69}.
% (To avoid bibliography problems, for now we redefine the \cite command.)
% Also commands that create a suitable format for the reference list.
\renewcommand{\cite}[1]{[#1]}
\def\beginrefs{\begin{list}%
        {[\arabic{equation}]}{\usecounter{equation}
         \setlength{\leftmargin}{2.0truecm}\setlength{\labelsep}{0.4truecm}%
         \setlength{\labelwidth}{1.6truecm}}}
\def\endrefs{\end{list}}
\def\bibentry#1{\item[\hbox{[#1]}]}

%Use this command for a figure; it puts a figure in wherever you want it.
%usage: \fig{NUMBER}{SPACE-IN-INCHES}{CAPTION}
\newcommand{\fig}[3]{
			\vspace{#2}
			\begin{center}
			Figure \thelecnum.#1:~#3
			\end{center}
	}
% Use these for theorems, lemmas, proofs, etc.
\newtheorem{theorem}{Theorem}[lecnum]
\newtheorem{lemma}[theorem]{Lemma}
\newtheorem{proposition}[theorem]{Proposition}
\newtheorem{claim}[theorem]{Claim}
\newtheorem{corollary}[theorem]{Corollary}
\newtheorem{definition}[theorem]{Definition}
\newtheorem{example}[theorem]{Example}
\newenvironment{proof}{{\bf Proof:}}{\hfill\rule{2mm}{2mm}}

% **** IF YOU WANT TO DEFINE ADDITIONAL MACROS FOR YOURSELF, PUT THEM HERE:

\newcommand\E{\mathbb{E}}

\DeclarePairedDelimiter\ceil{\lceil}{\rceil}
\DeclarePairedDelimiter\floor{\lfloor}{\rfloor}
\DeclarePairedDelimiter\anglebrac{\langle}{\rangle}

\begin{document}
%\lecture{**LECTURE-NUMBER**}{**DATE**}{**LECTURER**}{**SCRIBE**}
\lecture{10}{Sublinear Time Algorithms}{Nisarg Shah}{Lily Li}
%\footnotetext{These notes are partially based on those of Nigel Mansell.}

Recall the two axis of consideration: deterministic vs randomized and exact vs inexact. Today we are going to discuss property testing (array is sorted or graph is bipartite), sub-linear space algorithm.

\section{Sub-linear Time Algorithm Continued}
\begin{example}
We have an array $A$ of length $n$. Array access takes $O(1)$. We need to check that the array is sorted. Here we assume that the elements of the array are distinct. 

We define $\epsilon$-far to be arrays in which we need to change at least $\epsilon n$ elements to make $A$ monotonic.

There are many trivial random algorithm that are not good. Consider: (1) randomly pick $t$ elements $a_1, ..., a_t$ of $A$ and check that $a_i$ and its neighbors are sorted. (2) randomly pick $t$ elements $a_1, ..., a_t$ and check that $a_1, ..., a_t$ is a sorted sub-list. Please come up with some counter examples.

Here is an algorithm which works: choose $\frac{2}{\epsilon}$ random indices $i$ in $A$. Look at the element $A[i]$ and perform binary search for $A[i]$. If we cannot find $A[i]$ then output fail.

Consider the analysis: 
\end{example}

\begin{example}
Property testing on graphs. Since there are two graph representations: adjacency matrix (dense) and adjacency list (list) we consider $\epsilon$-far for each. In the first case we need to change at least $\epsilon n^2$ entries to satisfy the property and in the second case we need to change at least $\epsilon nd$ entries where $d$ is some constant upper bound on the degree (if degree is to high, it makes sense to use adjacency matrices). 

Lets consider bipartite testing in the dense (adjacency matrix) model. The algorithm by Goldreich, Goldwasser, Ron: pick a random subset $S$ of the vertices where $\Theta = \left(\frac{\log(1/\epsilon)}{\epsilon^2}\right)$. Check if the induced graph on $S$ is bipartite. Apparently if the graph is $\epsilon$-far then we reject with probability $\frac{2}{3}$. 

Why does this not work as an algorithm for the sparse representation? Look at the definition of $\epsilon$-far for the models. In a way the dense model allows more bad edges. So here is an algorithm for the sparse representation: for $O(1/\epsilon)$ iterations, we will randomly pick a vertex and try to look for odd cycles starting from $v$. If we do we reject. This had a worse case running time and perform  than the algorithm for the dense model, but this is typically the case. 
\end{example}

\section{Sub-linear Space Algorithm}
When we talk about sub-linear space, we are talking about sub-linear for the work space. Some unknown fundamental unknown questions:
\begin{itemize}
\item $NSPACE(S) = DSPACE(S)$ for $S \geq \log n$?
\item $P = L$? (where $L = DSPACE(\log n)$).
\item $USTCON$ vs $STCON$ where $USTCON$ is the problem of finding the connectivity of two vertices for an undirected graph. 
\end{itemize} 

\section{Streaming Algorithms}
We are given the input data as a stream $a_1, ..., a_m$. Each $a_i \in \{1, ..., n\}$. We wish to use $S(m,n)$ memory where $S$ is sub-linear. It is often the case that $m$ is not known before hand. 

\begin{example}
Consider the missing element problem. (And... we did not talk about the most interesting algorithm.)
\end{example}

\subsection{Frequency Moments}
Consider next the problem of computing \emph{frequency moments}. Let $A$ be the data stream from above. 

\subsection{Majority Element}
You know that the stream is of size $m$ and that there exists some element $a_i$ which appears $\frac{m}{2}$ times. Find this element. This is quite an interesting problem. You should take time and think about it on your own. Bad luck... everyone keeps on talking about the solution. Ok, I know what the algorithm should be.  
\end{document}