%
% This is a borrowed LaTeX template file for lecture notes for CS267,
% Applications of Parallel Computing, UCBerkeley EECS Department.
% Now being used for the Coursera cryptography course, part one.
%

\documentclass[twoside]{article}
\setlength{\oddsidemargin}{0.25 in}
\setlength{\evensidemargin}{-0.25 in}
\setlength{\topmargin}{-0.6 in}
\setlength{\textwidth}{6.5 in}
\setlength{\textheight}{8.5 in}
\setlength{\headsep}{0.75 in}
\setlength{\parindent}{0 in}
\setlength{\parskip}{0.1 in}

%
% ADD PACKAGES here:
%

\usepackage{amsmath,amsfonts,amssymb, graphicx}
\usepackage{color}
\usepackage{comment}
\usepackage{mathtools}

%
% The following commands set up the lecnum (lecture number)
% counter and make various numbering schemes work relative
% to the lecture number.
%
\newcounter{lecnum}
\renewcommand{\thepage}{\thelecnum-\arabic{page}}
\renewcommand{\thesection}{\thelecnum.\arabic{section}}
\renewcommand{\theequation}{\thelecnum.\arabic{equation}}
\renewcommand{\thefigure}{\thelecnum.\arabic{figure}}
\renewcommand{\thetable}{\thelecnum.\arabic{table}}

%
% The following macro is used to generate the header.
%
\newcommand{\lecture}[4]{
   \pagestyle{myheadings}
   \thispagestyle{plain}
   \newpage
   \setcounter{lecnum}{#1}
   \setcounter{page}{1}
   \noindent
   \begin{center}
   \framebox{
      \vbox{\vspace{2mm}
    \hbox to 6.28in { {\bf CSC2420: Algorithm Design, Analysis and Theory
	\hfill Fall 2017} }
       \vspace{4mm}
       \hbox to 6.28in { {\Large \hfill Lecture #1: #2  \hfill} }
       \vspace{2mm}
       \hbox to 6.28in { {\it Lecturer: #3 \hfill Scribe: #4} }
      \vspace{2mm}}
   }
   \end{center}
   \markboth{Lecture #1: #2}{Lecture #1: #2}

   %{\bf Note}: {\it LaTeX template courtesy of UC Berkeley EECS dept.}

   %{\bf Disclaimer}: {\it These notes have not been subjected to the
   %usual scrutiny reserved for formal publications.  They may be distributed
   %outside this class only with the permission of the Instructor.}
   \vspace*{4mm}
}
%
% Convention for citations is authors' initials followed by the year.
% For example, to cite a paper by Leighton and Maggs you would type
% \cite{LM89}, and to cite a paper by Strassen you would type \cite{S69}.
% (To avoid bibliography problems, for now we redefine the \cite command.)
% Also commands that create a suitable format for the reference list.
\renewcommand{\cite}[1]{[#1]}
\def\beginrefs{\begin{list}%
        {[\arabic{equation}]}{\usecounter{equation}
         \setlength{\leftmargin}{2.0truecm}\setlength{\labelsep}{0.4truecm}%
         \setlength{\labelwidth}{1.6truecm}}}
\def\endrefs{\end{list}}
\def\bibentry#1{\item[\hbox{[#1]}]}

%Use this command for a figure; it puts a figure in wherever you want it.
%usage: \fig{NUMBER}{SPACE-IN-INCHES}{CAPTION}
\newcommand{\fig}[3]{
			\vspace{#2}
			\begin{center}
			Figure \thelecnum.#1:~#3
			\end{center}
	}
% Use these for theorems, lemmas, proofs, etc.
\newtheorem{theorem}{Theorem}[lecnum]
\newtheorem{lemma}[theorem]{Lemma}
\newtheorem{proposition}[theorem]{Proposition}
\newtheorem{claim}[theorem]{Claim}
\newtheorem{corollary}[theorem]{Corollary}
\newtheorem{definition}[theorem]{Definition}
\newtheorem{example}[theorem]{Example}
\newenvironment{proof}{{\bf Proof:}}{\hfill\rule{2mm}{2mm}}

% **** IF YOU WANT TO DEFINE ADDITIONAL MACROS FOR YOURSELF, PUT THEM HERE:

\newcommand\E{\mathbb{E}}
\newcommand\J{\mathcal{J}}
\newcommand\I{\mathcal{I}}

\DeclarePairedDelimiter\ceil{\lceil}{\rceil}
\DeclarePairedDelimiter\floor{\lfloor}{\rfloor}
\DeclarePairedDelimiter\anglebrac{\langle}{\rangle}

\begin{document}
%\lecture{**LECTURE-NUMBER**}{**DATE**}{**LECTURER**}{**SCRIBE**}
\lecture{2}{Priority Algorithms}{Allan Borodin, Nisarg Shah}{Lily Li}
%\footnotetext{These notes are partially based on those of Nigel Mansell.}

\section{Continuing on from Last Time...}
Also called myopic algorithms. This is a continuation of the discussion from last week. Recall the make-span problem from before. You need to impose a time limit --- say polynomial --- on determining the order of the objects. Otherwise you can just see all the objects. 

\begin{definition}
Consider a deterministic fixed order priority algorithm. We use this framework to argue negative results thus we view the model as a game between the algorithm and an adversary (oblivious not adaptive). Initial there is some (possibly infinite) set $\J$ of potential inputs. The algorithm choses a total ordering $\pi$ on $\J$. The adversary chooses $\I \subset \J$ to be the input of the priority algorithm. The input items are then ordered according to $\pi$. Note: the ordering is local. Finally the algorithm executes on the input seeing the items in order.

In the make-span case you can think of $\pi$ as ordering the jobs in decreasing order. The adaptive case just has $\pi$ determined by the adversary.

Notice here that we do not force each of these steps to have some time complexity. Further we do not consider advice. As it turns out, that really helps even though you might only get one bit of advice. 
\end{definition}

Here is an example of a problem that can be solved by this framework: interval scheduling problem (ISP). It turns out that weighted interval problem cannot be solved with any constant approximation using greedy algorithm, \emph{but} it can be solved optimally by some other method.

\begin{example}
Do you remember what it greedy algorithm for this problem? Answer: greedy by earliest finishing time. Notice here that for this problem tie-breakers do not matter. For proportional values (weight is length) there is a 3-approximation greedy algorithm (there is some intuition for why it should be 3, can you see it?). Strangely if you add a machine so you have to choose the machine then schedule, there is a greedy 2-approximation. 
\end{example}

Multipass Algorithm: um...

\section{Greedy Algorithms for Set Packing}
Here we consider combinatorial auctions. Given $n$ subsets $S_1, ..., S_n$ from a universe $U$ of size $m$. The goal is to find disjoint sub collections of $S$ so as to maximize $\sum_{S_i \in S} w_i$. In the $s$-set packing problem we have $|S_i| \leq s$ for all $i$. 

There are two natural greedy algorithms what are they? Unfortunately both of these greedy algorithms only produce a $s$-approximation.

\section{Greedy Algorithm for Vertex Cover}
Here we want to consider cases where the edges are weighted. If the edges are weighted then there are many 2-approximates. There is a conjecture that there are \emph{no} $2-\epsilon$ approximation. For the weighted case Clarkson's greedy algorithm is the one which gives you the 2-approximation.

\section{Priority Stack Algorithm}
Here the algorithm will maintain a stack of items (possibly with conflicts) that we might want to keep. At the end the algorithm pops the stack and resolves the conflicts. Apparently, with the appropriate rules for keeping stuff on stack, we can get a pretty good approximation.

\end{document}