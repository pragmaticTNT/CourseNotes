%
% This is a borrowed LaTeX template file for lecture notes for CS267,
% Applications of Parallel Computing, UCBerkeley EECS Department.
% Now being used for the Coursera cryptography course, part one.
%

\documentclass[twoside]{article}
\setlength{\oddsidemargin}{0.25 in}
\setlength{\evensidemargin}{-0.25 in}
\setlength{\topmargin}{-0.6 in}
\setlength{\textwidth}{6.5 in}
\setlength{\textheight}{8.5 in}
\setlength{\headsep}{0.75 in}
\setlength{\parindent}{0 in}
\setlength{\parskip}{0.1 in}

%
% ADD PACKAGES here:
%

\usepackage{amsmath,amsfonts,amssymb, graphicx}
\usepackage{color}
\usepackage{comment}
\usepackage{mathtools}

%
% The following commands set up the lecnum (lecture number)
% counter and make various numbering schemes work relative
% to the lecture number.
%
\newcounter{lecnum}
\renewcommand{\thepage}{\thelecnum-\arabic{page}}
\renewcommand{\thesection}{\thelecnum.\arabic{section}}
\renewcommand{\theequation}{\thelecnum.\arabic{equation}}
\renewcommand{\thefigure}{\thelecnum.\arabic{figure}}
\renewcommand{\thetable}{\thelecnum.\arabic{table}}

%
% The following macro is used to generate the header.
%
\newcommand{\lecture}[4]{
   \pagestyle{myheadings}
   \thispagestyle{plain}
   \newpage
   \setcounter{lecnum}{#1}
   \setcounter{page}{1}
   \noindent
   \begin{center}
   \framebox{
      \vbox{\vspace{2mm}
    \hbox to 6.28in { {\bf CSC2420: Algorithm Design, Analysis and Theory
	\hfill Fall 2017} }
       \vspace{4mm}
       \hbox to 6.28in { {\Large \hfill Lecture #1: #2  \hfill} }
       \vspace{2mm}
       \hbox to 6.28in { {\it Lecturer: #3 \hfill Scribe: #4} }
      \vspace{2mm}}
   }
   \end{center}
   \markboth{Lecture #1: #2}{Lecture #1: #2}

   %{\bf Note}: {\it LaTeX template courtesy of UC Berkeley EECS dept.}

   %{\bf Disclaimer}: {\it These notes have not been subjected to the
   %usual scrutiny reserved for formal publications.  They may be distributed
   %outside this class only with the permission of the Instructor.}
   \vspace*{4mm}
}
%
% Convention for citations is authors' initials followed by the year.
% For example, to cite a paper by Leighton and Maggs you would type
% \cite{LM89}, and to cite a paper by Strassen you would type \cite{S69}.
% (To avoid bibliography problems, for now we redefine the \cite command.)
% Also commands that create a suitable format for the reference list.
\renewcommand{\cite}[1]{[#1]}
\def\beginrefs{\begin{list}%
        {[\arabic{equation}]}{\usecounter{equation}
         \setlength{\leftmargin}{2.0truecm}\setlength{\labelsep}{0.4truecm}%
         \setlength{\labelwidth}{1.6truecm}}}
\def\endrefs{\end{list}}
\def\bibentry#1{\item[\hbox{[#1]}]}

%Use this command for a figure; it puts a figure in wherever you want it.
%usage: \fig{NUMBER}{SPACE-IN-INCHES}{CAPTION}
\newcommand{\fig}[3]{
			\vspace{#2}
			\begin{center}
			Figure \thelecnum.#1:~#3
			\end{center}
	}
% Use these for theorems, lemmas, proofs, etc.
\newtheorem{theorem}{Theorem}[lecnum]
\newtheorem{lemma}[theorem]{Lemma}
\newtheorem{proposition}[theorem]{Proposition}
\newtheorem{claim}[theorem]{Claim}
\newtheorem{corollary}[theorem]{Corollary}
\newtheorem{definition}[theorem]{Definition}
\newtheorem{example}[theorem]{Example}
\newenvironment{proof}{{\bf Proof:}}{\hfill\rule{2mm}{2mm}}

% **** IF YOU WANT TO DEFINE ADDITIONAL MACROS FOR YOURSELF, PUT THEM HERE:

\newcommand\E{\mathbb{E}}
\def\P{\mathsf{P}}
\def\NP{\mathsf{NP}}
\def\poly{\mathsf{Poly}}

\DeclarePairedDelimiter\ceil{\lceil}{\rceil}
\DeclarePairedDelimiter\floor{\lfloor}{\rfloor}
\DeclarePairedDelimiter\anglebrac{\langle}{\rangle}

\begin{document}
\lecture{1}{Introductions}{Allan Borodin, Nisarg Shah}{Lily Li}

\section{Introductions}
Allan is doing the first half. He is doing conceptually simple algorithms. The interest is to take a paradigm and put in a theoretical framework (outlining its capabilities and limitations and giving these things \emph{definitions}). Nisarg is doing the second half of the course. 

\section{Graham's Online Algorithm and the Makespan Problem}
These papers are old, we are talking 1966-ish. They precede definition of $\NP$  vs $\P$ but the ideas described are similar. 

The makespan problem is similar to the load assignment/ bin packing problem (the simplified form is partition of course). $n$ jobs $J_1, ..., J_n$ each job has a some execution time. There are $m$ machines assigned to do the jobs, we wish to minimize the finish time.

Graham did some some competitive analysis (before the term was even defined!) for the greedy algorithm for the online version of this problem. Several different ways to do the analysis: worst case (adversarial),  stochastic (assuming, usually, i.i.d), random order model (ROM) --- adversary does some planning to form a distribution but nature gets to make random choices from distribution.   

Possible extensions to the one pass paradigm allows for look-ahead, revoking rights, multiple passes, advice etc. Instead of extension, you could think of different measures of performance. Some of these measures are not at all trivial. 

Back to the makespan problem. First greedy: where you put down the jobs into the currently most free machine. This results in a $2 - 1/m$ approximation (please formalize this). Possible improvement would be to sort in decreasing order. This results in a $4/3 - 1/(3m)$ approximation (please analyze this --- hint: do this by contradiction then take a look at the smallest job, this job has to pretty big... something, something contradiction).

There are possibly better approximation schemes. There is even a PTAS (polynomial time approximation scheme) that is an $1 + \epsilon$ approximation, but of course, more accuracy means more time namely $\poly(e^{1/\epsilon}, n)$ where $n$ is the length of the description. 

When you start sorting, the problem is no longer online. Approximation bound depends on $m$ --- the number of machines. 

Further extensions of the problem include: different machines take different times (speed), or jobs can only be run on some machines (restricted machines), or precedence constraints (Graham actually showed that the ratio is still $2 - 1/m$, but there are some modifications to the problem, add more machines, reducing the time of jobs, removing restricts, that changes the calculations).

\section{Knapsack Problem}

\subsection{Priority Algorithm Model}
This is a way of discussing greedy algorithms. The intuition of these algorithm   considers one input item in each iteration and at each iteration we make a myopic choice (using previously seen info).     


\end{document}