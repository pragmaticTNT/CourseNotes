%
% This is a borrowed LaTeX template file for lecture notes for CS267,
% Applications of Parallel Computing, UCBerkeley EECS Department.
% Now being used for the Coursera cryptography course, part one.
%

\documentclass[twoside]{article}
\setlength{\oddsidemargin}{0.25 in}
\setlength{\evensidemargin}{-0.25 in}
\setlength{\topmargin}{-0.6 in}
\setlength{\textwidth}{6.5 in}
\setlength{\textheight}{8.5 in}
\setlength{\headsep}{0.75 in}
\setlength{\parindent}{0 in}
\setlength{\parskip}{0.1 in}

%
% ADD PACKAGES here:
%

\usepackage{amsmath,amsfonts,amssymb, graphicx}
\usepackage{color}
\usepackage{comment}
\usepackage{mathtools}

%
% The following commands set up the lecnum (lecture number)
% counter and make various numbering schemes work relative
% to the lecture number.
%
\newcounter{lecnum}
\renewcommand{\thepage}{\thelecnum-\arabic{page}}
\renewcommand{\thesection}{\thelecnum.\arabic{section}}
\renewcommand{\theequation}{\thelecnum.\arabic{equation}}
\renewcommand{\thefigure}{\thelecnum.\arabic{figure}}
\renewcommand{\thetable}{\thelecnum.\arabic{table}}

%
% The following macro is used to generate the header.
%
\newcommand{\lecture}[4]{
   \pagestyle{myheadings}
   \thispagestyle{plain}
   \newpage
   \setcounter{lecnum}{#1}
   \setcounter{page}{1}
   \noindent
   \begin{center}
   \framebox{
      \vbox{\vspace{2mm}
    \hbox to 6.28in { {\bf CSC2420: Algorithm Design, Analysis and Theory
	\hfill Fall 2017} }
       \vspace{4mm}
       \hbox to 6.28in { {\Large \hfill Lecture #1: #2  \hfill} }
       \vspace{2mm}
       \hbox to 6.28in { {\it Lecturer: #3 \hfill Scribe: #4} }
      \vspace{2mm}}
   }
   \end{center}
   \markboth{Lecture #1: #2}{Lecture #1: #2}

   %{\bf Note}: {\it LaTeX template courtesy of UC Berkeley EECS dept.}

   %{\bf Disclaimer}: {\it These notes have not been subjected to the
   %usual scrutiny reserved for formal publications.  They may be distributed
   %outside this class only with the permission of the Instructor.}
   \vspace*{4mm}
}
%
% Convention for citations is authors' initials followed by the year.
% For example, to cite a paper by Leighton and Maggs you would type
% \cite{LM89}, and to cite a paper by Strassen you would type \cite{S69}.
% (To avoid bibliography problems, for now we redefine the \cite command.)
% Also commands that create a suitable format for the reference list.
\renewcommand{\cite}[1]{[#1]}
\def\beginrefs{\begin{list}%
        {[\arabic{equation}]}{\usecounter{equation}
         \setlength{\leftmargin}{2.0truecm}\setlength{\labelsep}{0.4truecm}%
         \setlength{\labelwidth}{1.6truecm}}}
\def\endrefs{\end{list}}
\def\bibentry#1{\item[\hbox{[#1]}]}

%Use this command for a figure; it puts a figure in wherever you want it.
%usage: \fig{NUMBER}{SPACE-IN-INCHES}{CAPTION}
\newcommand{\fig}[3]{
			\vspace{#2}
			\begin{center}
			Figure \thelecnum.#1:~#3
			\end{center}
	}
% Use these for theorems, lemmas, proofs, etc.
\newtheorem{theorem}{Theorem}[lecnum]
\newtheorem{lemma}[theorem]{Lemma}
\newtheorem{proposition}[theorem]{Proposition}
\newtheorem{claim}[theorem]{Claim}
\newtheorem{corollary}[theorem]{Corollary}
\newtheorem{definition}[theorem]{Definition}
\newtheorem{example}[theorem]{Example}
\newenvironment{proof}{{\bf Proof:}}{\hfill\rule{2mm}{2mm}}

% **** IF YOU WANT TO DEFINE ADDITIONAL MACROS FOR YOURSELF, PUT THEM HERE:

\newcommand\Z{\mathbb{Z}}

\DeclarePairedDelimiter\ceil{\lceil}{\rceil}
\DeclarePairedDelimiter\floor{\lfloor}{\rfloor}
\DeclarePairedDelimiter\anglebrac{\langle}{\rangle}

\begin{document}
%\lecture{**LECTURE-NUMBER**}{**DATE**}{**LECTURER**}{**SCRIBE**}
\lecture{6}{Randomized Algorithms}{Allan Borodin}{Lily Li}
\section{From Last Time...}
Recall the $\{0,1\}$-metric labeling problem is $\mathsf{NP-Hard}$ for three or more labels.

Next consider an IP/LP relation solution for the makespan problem in the unrestricted machine model. We can use the indicator variables $x_{ij}$ to say if job $b_i$ has been put on machine $h_j$ or not. Further you have $1 \geq x_{ij} \geq 0$ and $\sum x_{ij} \leq T$ for the maximum makespan $T$ on any machine. The exact solution is integral. The relation allows any real value between $0$ and $1$. However you should be able to see that the integrality gap could be as bad as possible.

Concluding remarks for LP rounding. Note that the integrality gap is in relation to the LP formulation so it may help to add more constraints. This shrinks the poly-tope. One useful technique is the LS lift and project method.  

\section{Duality}
(In terms of LPs) The dual of a primal maximization instance is the dual minimization instance and vice versa. Consider the following example:
\begin{example}
Recall the set cover problem in its IP/LP formulation:
\begin{align*}
&\mbox{minmize } \sum_{j} w_jx_j\\
&\mbox{subject to } \sum_{j: e_i \in S_j} x_j \geq 1 \mbox{ for all }i; \quad e_i \in U\\ 
&x_j \in \{0,1\} \quad(x_j \geq 0)
\end{align*}
Now consider its dual:
\begin{align*}
&\mbox{maximize } \sum_{i} y_i\\
&\mbox{subject to } \sum_{i: e_i \in S_j} y_i \leq w_j \mbox{ for all }j\\ 
&y_i \geq 0
\end{align*}

It is a fact that the optimal value for the primal is the same as the dual if both exist and are finite. 
\end{example}

The \textbf{Weak Duality} for a minimization problem is: that if $x$ and $y$ are primal and dual solutions respectively then $\mathcal{D}(y) \leq \mathcal{P}(x)$. 

\section{Randomized Algorithms}
Problems in randomized poly-time not known to be in poly-time:
\begin{enumerate}
\item Symbolic determinant.
\item Given $n$ find prime in $[2^n, 2^{n+1}]$.
\item Estimating volume of a convex body given by a set of linear inequalities. 
\item Solving a quadratic equation in $\Z_p[x]$.  
\item Polynomial Identity Testing (via Schwartz Zipple)
\end{enumerate}

\subsection{Naive Application}
Consider the example of $\mathsf{Max-k-SAT}$. When we are talking about approximations here we are going to use fractional ratios $c < 1$ (what $c \cdot P_{OPT}$ of the clauses can you cover, where $P_{OPT}$ is the maximum number of clauses coverable given any assignment).

Naive randomized approximation algorithm? Randomly assign value the variables! Suppose we are just consider exact $\mathsf{k-SAT}$. Each clause has probability $1 - \frac{1}{2^k}$ of being satisfied.   

\end{document}