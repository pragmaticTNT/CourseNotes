%
% This is a borrowed LaTeX template file for lecture notes for CS267,
% Applications of Parallel Computing, UCBerkeley EECS Department.
% Now being used for the Coursera cryptography course, part one.
%

\documentclass[twoside]{article}
\setlength{\oddsidemargin}{0.25 in}
\setlength{\evensidemargin}{-0.25 in}
\setlength{\topmargin}{-0.6 in}
\setlength{\textwidth}{6.5 in}
\setlength{\textheight}{8.5 in}
\setlength{\headsep}{0.75 in}
\setlength{\parindent}{0 in}
\setlength{\parskip}{0.1 in}

%
% ADD PACKAGES here:
%

\usepackage{amsmath,amsfonts,amssymb, graphicx}
\usepackage{color}
\usepackage{comment}
\usepackage{mathtools}

%
% The following commands set up the lecnum (lecture number)
% counter and make various numbering schemes work relative
% to the lecture number.
%
\newcounter{lecnum}
\renewcommand{\thepage}{\thelecnum-\arabic{page}}
\renewcommand{\thesection}{\thelecnum.\arabic{section}}
\renewcommand{\theequation}{\thelecnum.\arabic{equation}}
\renewcommand{\thefigure}{\thelecnum.\arabic{figure}}
\renewcommand{\thetable}{\thelecnum.\arabic{table}}

%
% The following macro is used to generate the header.
%
\newcommand{\lecture}[4]{
   \pagestyle{myheadings}
   \thispagestyle{plain}
   \newpage
   \setcounter{lecnum}{#1}
   \setcounter{page}{1}
   \noindent
   \begin{center}
   \framebox{
      \vbox{\vspace{2mm}
    \hbox to 6.28in { {\bf CSC2420: Algorithm Design, Analysis and Theory
	\hfill Fall 2017} }
       \vspace{4mm}
       \hbox to 6.28in { {\Large \hfill Lecture #1: #2  \hfill} }
       \vspace{2mm}
       \hbox to 6.28in { {\it Lecturer: #3 \hfill Scribe: #4} }
      \vspace{2mm}}
   }
   \end{center}
   \markboth{Lecture #1: #2}{Lecture #1: #2}

   %{\bf Note}: {\it LaTeX template courtesy of UC Berkeley EECS dept.}

   %{\bf Disclaimer}: {\it These notes have not been subjected to the
   %usual scrutiny reserved for formal publications.  They may be distributed
   %outside this class only with the permission of the Instructor.}
   \vspace*{4mm}
}
%
% Convention for citations is authors' initials followed by the year.
% For example, to cite a paper by Leighton and Maggs you would type
% \cite{LM89}, and to cite a paper by Strassen you would type \cite{S69}.
% (To avoid bibliography problems, for now we redefine the \cite command.)
% Also commands that create a suitable format for the reference list.
\renewcommand{\cite}[1]{[#1]}
\def\beginrefs{\begin{list}%
        {[\arabic{equation}]}{\usecounter{equation}
         \setlength{\leftmargin}{2.0truecm}\setlength{\labelsep}{0.4truecm}%
         \setlength{\labelwidth}{1.6truecm}}}
\def\endrefs{\end{list}}
\def\bibentry#1{\item[\hbox{[#1]}]}

%Use this command for a figure; it puts a figure in wherever you want it.
%usage: \fig{NUMBER}{SPACE-IN-INCHES}{CAPTION}
\newcommand{\fig}[3]{
			\vspace{#2}
			\begin{center}
			Figure \thelecnum.#1:~#3
			\end{center}
	}
% Use these for theorems, lemmas, proofs, etc.
\newtheorem{theorem}{Theorem}[lecnum]
\newtheorem{lemma}[theorem]{Lemma}
\newtheorem{proposition}[theorem]{Proposition}
\newtheorem{claim}[theorem]{Claim}
\newtheorem{corollary}[theorem]{Corollary}
\newtheorem{definition}[theorem]{Definition}
\newtheorem{example}[theorem]{Example}
\newenvironment{proof}{{\bf Proof:}}{\hfill\rule{2mm}{2mm}}

% **** IF YOU WANT TO DEFINE ADDITIONAL MACROS FOR YOURSELF, PUT THEM HERE:

\newcommand\E{\mathbb{E}}

\DeclarePairedDelimiter\ceil{\lceil}{\rceil}
\DeclarePairedDelimiter\floor{\lfloor}{\rfloor}
\DeclarePairedDelimiter\anglebrac{\langle}{\rangle}

\begin{document}
%\lecture{**LECTURE-NUMBER**}{**DATE**}{**LECTURER**}{**SCRIBE**}
\lecture{9}{Randomized Algorithm}{Nisarg Shah}{Lily Li}
%\footnotetext{These notes are partially based on those of Nigel Mansell.}

\section{Sublinear Time Algorithm}
Running time in $o(n)$ where $n$ is the length of the input and assuming constant time access to the $i^{th}$ input. Randomization and approximate solution. \textbf{Property testing} in accuracy (in detail later on): you are familiar with one and two sized error, here if the property is satisfied you must say $YES$, if the property is not satisfied but within some $\epsilon$ of being satisfied then the algorithm can output anything. 

\subsection{Randomized and Exact}
\textbf{Las Vegas} algorithm. 
\begin{example}
\begin{enumerate}
\item[INPUT:] a sorted doubly linked list $L$ with $n$ elements. Note that this is a sorted \emph{doubly linked list}. Access to the $i^{th}$ entry of the underlying array takes $O(n)$, but if you want the sorted order you have to walk as the linked list (entries of the form $x_i, prev_i, nex_i$). A target $x$.
\item[GOAL:] determine if $x_i = x$ for some $x_i \in L$.
\item[ALGORITHM:] we want an algorithm with expected running time $O(\sqrt{n})$.  
\end{enumerate}
\end{example}

\subsection{Yao's Principle}
\begin{claim}
The expected time of a running time of a randomized algorithm $R$ on the worst input $I$ is no better than the expected time taken under the worst probability distribution $D$ over inputs, by the best deterministic algorithm. 
\end{claim}

\subsection{Randomized and Inexact}
\begin{example}
Estimate average degree in a graph. Let a graph $G$ with $|V| = n$. We have a oracle which can tell us the degree in time $O(1)$. 
\end{example}

\section{Property Testing}
Given some input $I$, test if $I$ satisfies some property $P$. If $I$ is satisfying, must output $YES$. If $I$ is $\epsilon$-far from $P$ then must output $NO$. If $I$ is $\epsilon$-close to satisfying the property, then we do not care. 
\end{document}