%
% This is a borrowed LaTeX template file for lecture notes for CS267,
% Applications of Parallel Computing, UCBerkeley EECS Department.
\documentclass[twoside]{article}
\setlength{\oddsidemargin}{0.25 in}
\setlength{\evensidemargin}{-0.25 in}
\setlength{\topmargin}{-0.6 in}
\setlength{\textwidth}{6.5 in}
\setlength{\textheight}{8.5 in}
\setlength{\headsep}{0.75 in}
\setlength{\parindent}{0 in}
\setlength{\parskip}{0.1 in}

%
% ADD PACKAGES here:
%

\usepackage{amsmath,amsfonts,amssymb, graphicx}
\usepackage{algorithm}
\usepackage{algpseudocode}
\usepackage{color}
\usepackage{comment}
\usepackage{mathtools}

%
% The following commands set up the lecnum (lecture number)
% counter and make various numbering schemes work relative
% to the lecture number.
%
\newcounter{lecnum}
\renewcommand{\thepage}{\thelecnum-\arabic{page}}
\renewcommand{\thesection}{\thelecnum.\arabic{section}}
\renewcommand{\theequation}{\thelecnum.\arabic{equation}}
\renewcommand{\thefigure}{\thelecnum.\arabic{figure}}
\renewcommand{\thetable}{\thelecnum.\arabic{table}}

%
% The following macro is used to generate the header.
%
\newcommand{\lecture}[4]{
   \pagestyle{myheadings}
   \thispagestyle{plain}
   \newpage
   \setcounter{lecnum}{#1}
   \setcounter{page}{1}
   \noindent
   \begin{center}
   \framebox{
      \vbox{\vspace{2mm}
    \hbox to 6.28in { {\bf CSC2429: Proof Complexity
	\hfill Winter 2018} }
       \vspace{4mm}
       \hbox to 6.28in { {\Large \hfill Lecture #1: #2  \hfill} }
       \vspace{2mm}
       \hbox to 6.28in { {\it Lecturer: #3 \hfill Scribe: #4} }
      \vspace{2mm}}
   }
   \end{center}
   \markboth{Lecture #1: #2}{Lecture #1: #2}
   \vspace*{4mm}
}
\renewcommand{\cite}[1]{[#1]}
\def\beginrefs{\begin{list}%
        {[\arabic{equation}]}{\usecounter{equation}
         \setlength{\leftmargin}{2.0truecm}\setlength{\labelsep}{0.4truecm}%
         \setlength{\labelwidth}{1.6truecm}}}
\def\endrefs{\end{list}}
\def\bibentry#1{\item[\hbox{[#1]}]}

\newcommand{\fig}[3]{
			\vspace{#2}
			\begin{center}
			Figure \thelecnum.#1:~#3
			\end{center}
	}
% Use these for theorems, lemmas, proofs, etc.
\newtheorem{theorem}{Theorem}[lecnum]
\newtheorem{lemma}[theorem]{Lemma}
\newtheorem{proposition}[theorem]{Proposition}
\newtheorem{claim}[theorem]{Claim}
\newtheorem{corollary}[theorem]{Corollary}
\newtheorem{definition}[theorem]{Definition}
\newtheorem{example}[theorem]{Example}
\newenvironment{proof}{{\bf Proof:}}{\hfill\rule{2mm}{2mm}}

% **** IF YOU WANT TO DEFINE ADDITIONAL MACROS FOR YOURSELF, PUT THEM HERE:

\newcommand\degree{\mbox{deg}}
\newcommand\rd{\mbox{rd}}
\newcommand\opt{\mathsf{opt}}
\newcommand\trace{\mathsf{trace}}
\newcommand\deter{\mathsf{det}}
\newcommand\gaussian{\mathcal{N}}
\newcommand\Pf{\mathcal{P}}
\newcommand\Ecal{\mathcal{E}}
\newcommand\Fcal{\mathcal{F}}
\newcommand\Hcal{\mathcal{H}}
\newcommand\E{\mathbb{E}}
\newcommand\F{\mathbb{F}}
\newcommand\R{\mathbb{R}}
\newcommand\N{\mathbb{N}}

\newcommand\ClassP{\mathsf{P}}
\newcommand\ClassNP{\mathsf{NP}}
\newcommand\ClasscoNP{\mathsf{coNP}}
\newcommand\SAT{\mathsf{SAT}}
\newcommand\MaxSAT{\mathsf{MaxSAT}}
\newcommand\MaxCUT{\mathsf{MaxCUT}}
\newcommand\SDP{\mathsf{SDP}}

\DeclarePairedDelimiter\ceil{\lceil}{\rceil}
\DeclarePairedDelimiter\floor{\lfloor}{\rfloor}
\DeclarePairedDelimiter\anglebrac{\langle}{\rangle}
\DeclarePairedDelimiter\norm{\parallel}{\parallel}

\begin{document}
\lecture{2}{Proof System and Algorithms for LP}{Toni Pitassi}{Lily Li}

\section{Sherali Adams}
\subsection{Review LP}
Consider this as a sound and complete proof system for deriving or refuting a set of linear inequalities over $\R^n$. As always an LP is defined as 
\begin{align*}
\max \mathbf{c}^Tx \qquad \mbox{ over }\R\\
\mbox{Subject to: }A\mathbf{x} \leq \mathbf{b}
\end{align*}
The decision version of this problem asks if the feasible region of the LP is empty.

\begin{lemma}
(Farkas' Lemma.) A set $\{A\mathbf{x} \leq \mathbf{b}\}$ is unsatisfiable over $\R$ if and only if there exists $\mathbf{y} \geq 0$ such that $\mathbf{y}^T A = 0$ and $\mathbf{y}^T\mathbf{b} = -1$. 
\end{lemma}

Farkas' Lemma should be viewed as a soundness and completeness proof of the decision version and the $y^T$ can be seen as a linear combination of the rows of of $A$ which add to $-1$.

\subsection{Duality}
(Considered a form of implicational completeness). We want to show that each feasible solution of the dual produces an upper bound of the primal objective function. Consider any $\mathbf{y}^T \geq 0$ such that $\mathbf{y}^TA \geq \mathbf{c}^T$. Then for any feasible solution $\mathbf{x}$ of the primal $A\mathbf{x} \leq \mathbf{b}$, 
\[\mathbf{c}^T \leq \mathbf{y}^TA\mathbf{x} \leq \mathbf{y}^T\mathbf{b}\]
where $\mathbf{c}^T\mathbf{x}$ is exactly the primal objective function. Thus we say that $\mathbf{y}$ \emph{witnesses} the upper bound $\mathbf{b}^T\mathbf{y}$.

To see how tight this bound is, we consider following primal and dual LP problems:
\begin{align*}
&\mbox{(P) Primal:} &\mbox{(D) Dual:}\\
&\max \mathbf{c}^T\mathbf{x} &\min \mathbf{b}^T\mathbf{y}\\
&\mbox{Such that: } A\mathbf{x} \leq \mathbf{b} &A^T \mathbf{y} \geq \mathbf{c}\\
&\mathbf{x} \geq 0 &\mathbf{y} \geq 0
\end{align*}

\begin{theorem}
(Duality of LP.) \emph{This is implied by Farkas' Lemma.} For an $LP$ with primal solution $P$ and dual solution $D$, exactly of the following is true:
\begin{enumerate}
\item Neither (P) nor (D) has a feasible solution.
\item (P) has arbitrarily large solutions and (D) is unsatisfiable.
\item (P) is unsatisfiable and (D) has an arbitrarily small solution.
\item Both (P) and (D) have optimal solutions $\mathbf{x}^*$ and $\mathbf{y}^*$. Further $\mathbf{c}^T\mathbf{x}^* = \mathbf{b}^T\mathbf{y}^*$
\end{enumerate}
\end{theorem}

In the decision version an LP refutation is just the $\mathbf{y}$ from Farkas' Lemma.

An LP ``derivation" of \[\{A\mathbf{x} \leq \mathbf{b}, \mathbf{x} \geq 0\} \implies \mathbf{c}^T\mathbf{x} \leq \mathbf{c_0}\]
is a $\mathbf{y}^* \geq 0$ such that $(\mathbf{y}^*)^T\mathbf{b} = \mathbf{c_0}$. 

It is known that LP is in $\ClassP$. The satisfiability of an LP is in $\ClassNP$ since we can just guess a $\mathbf{x}$. Further Farkas' lemma shows that the problem is in $\ClasscoNP$ since we can just guess a $\mathbf{y}$ (there are problems which are in $\ClassP \cap \ClassNP$ but not known to be in $\ClassP$, e.g. factoring and graph isomorphism, but these are uncommon). In 1979 Khachiyan use the Ellipsoid Algorithm to show that $LP \in \ClassP$.    

\subsection{SA Introduction}
SA is a sound, lift and project proof system that ``tightens" the LP solution to a good approximation of the IP. SA can be interpretations: as (1) LP tightening, solutions to the extended degree $d$ SA LP pseudo-distribution, and as a proof or refutation system.  

\subsection{SA Degree $d$ Tightening}
When viewed as LP tightening in degree $d$, we do the following:
\begin{itemize}
\item Add new variables to represent all degree $\leq d$ terms.
\item \emph{Lift} the polytope from $n$ dimensions to $n^{o(d)}$ dimensions.
\item Project back to $x_1, ..., x_n$ preserves all $0,1$ solutions (and may remove some fractional solutions).
\end{itemize}

More formally, let the original LP be 
\[A\mathbf{x} \geq \mathbf{b} \mbox{, and let } \mathbf{1} \geq \mathbf{0}, \mathbf{0} \leq \mathbf{x} \leq \mathbf{1}.\] 
We add new variables:
\[y_S: \qquad\forall S \subseteq [n], |S| \leq d.\]
And new constraints (junta) for all subsets $U, V \subset [n]$ with $U \cap \emptyset$ and $|U \cup V| \leq d$:
\[\prod_{i \in U}x_i \cdot \prod_{j \in V}(1 - x_j) \cdot (\mathbf{a}^T\mathbf{x} - \mathbf{b}) \geq 0 \]
for all rows $\mathbf{a}$ of $A$, for all $U, V$.

When we translate the above constrains into linear equalities using the new $y_S$ variables and the multi-linear inequalities $x_i^2 = x_i$ we obtain:
\begin{align*}
y_{\emptyset} &= 1\\
y_{i} &= x_i\\
0 &\leq y_{S} \leq 1\\
\sum_{V' \subseteq V} (-1)^{|V'|} &\cdot \left(\sum_{i = 1}^n a_i y_{U \cup V' \cup \{i\}} - \mathbf{b}y_{U \cup V'}\right) \geq 0
\end{align*}
for all rows $\mathbf{a}$ of $A$.

Further, since we have trivially $\mathbf{1} \geq \mathbf{0}$ as an initial constraint, we can add
\[\prod_{i \in U}x_i \prod_{j \in V}(1-x_j) \geq 0\]
which translates to 
\[\sum_{V' \subseteq V}(-1)^{|V'|}\cdot \left(y_{U \cup V'}\right) \geq 0\]

Now we need to defined a pseudo-distribution for a degree $d$ SA LP (this is a real distribution over sets variables of degree at most $d$ and might not work else-where). Let $\Hcal = \{A\mathbf{x} - \mathbf{b} \geq \mathbf{0}, \mathbf{x} \geq \mathbf{0}, \mathbf{1} \geq \mathbf{0}\}$. $\Ecal_{d}(\Hcal)$ is a set of linear functionals $E: \R[x_1, ..., x_n]_d \rightarrow \R$ such that $\forall E \in \Ecal_{d}(\Hcal)$:
\begin{enumerate}
\item $E(1) = 1$
\item $E(Q) \geq 0$ for all non-negative junta $Q$ with $\degree(Q) \leq d$. 
\item $E(PQ) \geq 0$ for $P \in \Hcal$, and non-negative junta $Q$ with $\degree(PQ) \leq d$ where a non-negative junta is of the form:
\[\alpha \cdot \prod_{i \in U}x_i \prod_{j \in V}(1-x_j)\]
for non-negative coefficient $\alpha$ and $U \cup V = \emptyset$. \emph{This is quite useful to show that what you have is actually a a probability distribution (recall the marginals).}
\end{enumerate}

Essentially you should interpret each $E \in \Ecal_d(\Hcal)$ as a \emph{pseudo-distribution} for the feasible (maybe? my words here) region of $\Hcal$. Further, feasible solutions to the degree-$d$ SA polytope are exactly the degree-$d$ \emph{pseudo-distributions}.

Consider any feasible solution $\alpha$. For each variable $y_S$,$\alpha(y_S)$ is the value in $[0,1]$ assigned to $y_S$ such that all the linear constraints are satisfied.

The corresponding pseudo-distribution $E_{\alpha}$ which assigns values to degree-$d$ polynomials is defined in the obvious way.
\begin{example}
Suppose we have $d = 3$ and the polynomial $f = -x_1x_2x_4 + x_7 - 3x_1x_8$, then
\[E_{\alpha}[f] = -\alpha(y_{1,2,4}) + \alpha(y_{7}) -3\alpha(y_{1,8}).\]
\end{example}
It so happens that the degree-$d$ SA linear constraints exactly enforce the properties of the pseudo-distribution (what $E_{\alpha}$ does is assign a $[0,1]$ value to every set of $\leq d$ of the original variables such that the marginal distributions of each subset $S' \subset S$ is equal to the distribution assigned to each entry of $S'$). 

\subsection{Equivalent view as a Refutation System}
Let $\Fcal$ be a set of polynomial equalities of the form $x_i^2 - x_i = 0$ (this set is actually quite important, why?). Let $\Hcal$ be a set of polynomial inequalities $\{A\mathbf{x} - \mathbf{b} \geq 0$, $1 \geq \mathbf{x} \geq 0$, and $\mathbf{1} \geq \mathbf{0}\}$ our original inequalities. A degree-$d$ SA derivation of $-1$ from $\Fcal$ and $\Hcal$ --- witnessing that no feasible solutions exists --- is a set of polynomials $(g_1, ..., g_m, p_1, ..., p_l)$ such that
 
\begin{enumerate}
\item $\sum_{j=1}^{m} g_jf_j + \sum_{i = 1}^{l}p_ih_i = -1$
\item The $p_i$'s are non-negative linear combinations of juntas $p_i = \alpha_1Q_1 + \cdots + \alpha_mQ_m$ (the $Q$'s are the juntas), $h_i \in \Hcal$, $f_j \in \Fcal$, and $g_j$ are arbitrary polynomials --- since the $f_j$'s evaluate to 0. 
\item $\max\{\degree(g_jf_j), \degree(p_ih_i)\} = d$.
\end{enumerate}

\subsection{SA as a Derivation System}
SA as a derivation system is almost exactly like SA as a refutation system except that the sum of the product of the polynomials $g_jf_j$ and $p_ih_i$ is some other polynomial $f$ which we wish to derive.

\begin{lemma}
Let $\{A\mathbf{x} - b \geq 0, 1 \geq 0, \mathbf{x} \geq 0\} = \Hcal$. Then the degree-$d$ SA LP has no feasible solution if and only if there is a degree-$d$ SA refutation of $\Hcal$.
\end{lemma}
\begin{proof}
Let the SA refutation be a sett of polynomials as above. The SA LP has a feasible solution $\mathbf{x}^*$, then we can plug in $\mathbf{x}^*$ into the set of polynomials and evaluate. Since $\mathbf{x}^*$ is feasible the LHS of each inequality will be positive so it is impossible to obtain a sum equal to $-1$. 

Conversely if the SA LP has no feasible solution we can translate it into a refutation of $\Hcal$ as defined in the ``SA as a Refutation System" section.
\end{proof} 

\begin{example}
We are going to apply SA on the maximum independent set problem with the graph $G = C_7$. If you just formulate the standard LP as:

\begin{align*}
\mbox{Maximize: } &\sum_{i=1}^{7} x_i\\
\mbox{Subject to: } &x_{i} + x_{i+1} \leq 1\\
\mbox{ and } &0 \leq x_i \leq 1  \mbox{ for } 1 \leq i \leq 7
\end{align*}

you would get a fractional optimum value of $3.5$. Lifting to $d = 2$ introduces the following constraints:

\begin{align*}
x_1(x_1 + x_2) \leq x_1 &\implies y_{1,2} \leq 0\\
(1-x_1)(x_2 + x_3) \leq 1 - x_1 &\implies y_{2} + y_{3} - y_{1,2} - y_{1,3} \leq 1 - y_{1}\\
x_1(x_3 + x_4) \leq x_1 &\implies y_{1,3} + y_{1,4} \leq y_{1}\\
(1-x_1)(x_4 + x_5) \leq 1 - x_1 &\implies y_{4} + y_{5} - y_{1,4} - y_{1,5} \leq 1 - y_{1}\\
x_1(x_5 + x_6) \leq x_1 &\implies y_{1,5} + y_{1,6} \leq y_{1}\\
(1-x_1)(x_6 + x_7) \leq 1 - x_1 &\implies y_{6} + y_{7} - y_{1,6} - y_{1,7} \leq 1 - y_{1}\\
x_1(x_1 + x_7) \leq x_1 &\implies y_{1,7} \leq 0\\
\end{align*}

(as well as other constraints, but these are all the ones we need to consider). Running the LP here would produce the answer $3$ (which is optimal).
\end{example}

%\section{Cutting Planes}

\end{document}