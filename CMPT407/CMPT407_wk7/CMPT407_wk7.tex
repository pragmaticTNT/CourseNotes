%
% This is a borrowed LaTeX template file for lecture notes for CS267,
% Applications of Parallel Computing, UCBerkeley EECS Department.
% Now being used for CMU's 10725 Fall 2012 Optimization course
% taught by Geoff Gordon and Ryan Tibshirani.  When preparing 
% LaTeX notes for this class, please use this template.
%
% To familiarize yourself with this template, the body contains
% some examples of its use.  Look them over.  Then you can
% run LaTeX on this file.  After you have LaTeXed this file then
% you can look over the result either by printing it out with
% dvips or using xdvi. "pdflatex template.tex" should also work.
%

\documentclass[twoside]{article}
\setlength{\oddsidemargin}{0.25 in}
\setlength{\evensidemargin}{-0.25 in}
\setlength{\topmargin}{-0.6 in}
\setlength{\textwidth}{6.5 in}
\setlength{\textheight}{8.5 in}
\setlength{\headsep}{0.75 in}
\setlength{\parindent}{0 in}
\setlength{\parskip}{0.1 in}

%
% ADD PACKAGES here:
%

\usepackage{amsmath,amsfonts,amssymb, graphicx}
\usepackage{color}
\usepackage{comment}
\usepackage{mathtools}

%
% The following commands set up the lecnum (lecture number)
% counter and make various numbering schemes work relative
% to the lecture number.
%
\newcounter{lecnum}
\renewcommand{\thepage}{\thelecnum-\arabic{page}}
\renewcommand{\thesection}{\thelecnum.\arabic{section}}
\renewcommand{\theequation}{\thelecnum.\arabic{equation}}
\renewcommand{\thefigure}{\thelecnum.\arabic{figure}}
\renewcommand{\thetable}{\thelecnum.\arabic{table}}

%
% The following macro is used to generate the header.
%
\newcommand{\lecture}[4]{
   \pagestyle{myheadings}
   \thispagestyle{plain}
   \newpage
   \setcounter{lecnum}{#1}
   \setcounter{page}{1}
   \noindent
   \begin{center}
   \framebox{
      \vbox{\vspace{2mm}
    \hbox to 6.28in { {\bf CMPT 407: Computational Complexity
	\hfill Summer 2017} }
       \vspace{4mm}
       \hbox to 6.28in { {\Large \hfill Lecture #1: #2  \hfill} }
       \vspace{2mm}
       \hbox to 6.28in { {\it Lecturer: #3 \hfill Scribe: #4} }
      \vspace{2mm}}
   }
   \end{center}
   \markboth{Lecture #1: #2}{Lecture #1: #2}

   %{\bf Note}: {\it LaTeX template courtesy of UC Berkeley EECS dept.}

   %{\bf Disclaimer}: {\it These notes have not been subjected to the
   %usual scrutiny reserved for formal publications.  They may be distributed
   %outside this class only with the permission of the Instructor.}
   \vspace*{4mm}
}
%
% Convention for citations is authors' initials followed by the year.
% For example, to cite a paper by Leighton and Maggs you would type
% \cite{LM89}, and to cite a paper by Strassen you would type \cite{S69}.
% (To avoid bibliography problems, for now we redefine the \cite command.)
% Also commands that create a suitable format for the reference list.
%\renewcommand{\citep{•}ite}[1]{[#1]}
\def\beginrefs{\begin{list}%
        {[\arabic{equation}]}{\usecounter{equation}
         \setlength{\leftmargin}{2.0truecm}\setlength{\labelsep}{0.4truecm}%
         \setlength{\labelwidth}{1.6truecm}}}
\def\endrefs{\end{list}}
\def\bibentry#1{\item[\hbox{[#1]}]}

%Use this command for a figure; it puts a figure in wherever you want it.
%usage: \fig{NUMBER}{SPACE-IN-INCHES}{CAPTION}
\newcommand{\fig}[3]{
			\vspace{#2}
			\begin{center}
			Figure \thelecnum.#1:~#3
			\end{center}
	}
% Use these for theorems, lemmas, proofs, etc.
\newtheorem{theorem}{Theorem}[lecnum]
\newtheorem{lemma}[theorem]{Lemma}
\newtheorem{proposition}[theorem]{Proposition}
\newtheorem{claim}[theorem]{Claim}
\newtheorem{corollary}[theorem]{Corollary}
\newtheorem{definition}[theorem]{Definition}
\newtheorem{example}[theorem]{Example}
\newenvironment{proof}{{\bf Proof:}}{\hfill\rule{2mm}{2mm}}

% **** IF YOU WANT TO DEFINE ADDITIONAL MACROS FOR YOURSELF, PUT THEM HERE:

\newcommand\E{\mathbb{E}}
\def\N{\mathbb{N}}
\def\R{\mathbb{R}}
\def\Pr{\mathbf{Pr}}
\def\P{\mathsf{P}}
\def\NP{\mathsf{NP}}
\def\coNP{\mathsf{coNP}}
\def\BP{\mathsf{BP}}
\def\BPP{\mathsf{BPP}}
\def\BQP{\mathsf{BQP}}
\def\RP{\mathsf{RP}}
\def\ZPP{\mathsf{ZPP}}
\def\coRP{\mathsf{coRP}}
\def\SAT{\mathsf{SAT} then }
\def\PolySize{\mathsf{PolySize}}
\def\EXP{\mathsf{EXP}}
\def\Maj{\mathsf{Maj}}
\def\NISO{\mathsf{NISO}}
\def\IP{\mathsf{IP}}
\def\PSPACE{\mathsf{PSPACE}}
\def\QBF{\mathsf{QBF}}
\def\PCP{\mathsf{PCP}}

\def\MA{\mathsf{MA}}
\def\AM{\mathsf{AM}}

\DeclarePairedDelimiter\ceil{\lceil}{\rceil}
\DeclarePairedDelimiter\floor{\lfloor}{\rfloor}
\DeclarePairedDelimiter\anglebrac{\langle}{\rangle}

\begin{document}
%\lecture{**LECTURE-NUMBER**}{**DATE**}{**LECTURER**}{**SCRIBE**}
\lecture{7}{Randomized Computation (26 June - End)}{Valentine Kabanets}{Lily Li}
%\footnotetext{These notes are partially based on those of Nigel Mansell.}

\section{Randomness and Interaction}
Today we will show that $\BPP \subseteq \MA \subseteq \AM \subseteq \prod_2^p$. Remark that the first $\subseteq$ is actually quite straight forward. Please consider what $\BPP$ and $\MA$ are and explain why (\emph{hint: remember that $\MA$ is essentially the randomized version of $\NP$}). Remark the verifier Arthur checks $R(x,y,z)$ where $x$ is the input, $y$ is the value the Merlin sends over, and $z$ is the random string.

As before, we can decrease the error probability by choosing $k$ different random strings $z_1, ..., z_k$ and taking the majority of $R(x,y,z_i)$ for all $z_i$. Let us do the analysis (use Chernoff bound):
\begin{enumerate}
\item $x \in L$: then $\exists y, s.t. \forall i \leq i\leq k: \Pr_{z_i}[R(x,y,z_i) = 1] \geq 3/4$. Let the indicator random variable be $X_1, ..., X_k$ where $X_i = R(x,y,z_i)$. Note that $\mu = k\cdot 3/4$. Calculate the probability that fewer than $k/2$ variables accept by letting $X_1 + \cdots + X_n = X$:
\begin{align*}
\Pr[X < \frac{k}{2}] &\leq \Pr[X \leq \frac{2}{3} \cdot \mu] \\
&\leq \Pr[|X-\mu| \geq \frac{\mu}{3}] \\
&\leq 2e^{\frac{\mu}{-27}}
\end{align*}

where the last inequality follows from the Chernoff bound as discussed below. Thus this probability is quite low, $\propto 2^{-k}$.
\item $x \notin L$: then $\forall y \forall 1 \leq i \leq k, \Pr_{z_i}[R(x,y,z_i) = 1] \leq \frac{1}{4}$. By a similar application of Chernoff bounds we can show that $\Pr[X > k/2] \leq \Pr[|X-\mu| > k/4] \leq 2e^{k/48}$.
\end{enumerate}
\emph{Note: } what does the Chernoff bound actually say? It says for independent random variable $X_1, ..., X_n \in \{0,1\}$ s.t. $\Pr[X_i = 1] = p$ and $\Pr[X_i = 0] = 1-p$ for some $0 \leq p \leq 1$. If we let $X = \sum_{i=1}^{n} X_i$, then $\mathsf{Exp}[X] = n\cdot p = \mu$ and $\forall 0 < \epsilon < 1$,
\[\Pr[|X - \mu| > \epsilon \cdot \mu] < 2\cdot e^{-\epsilon^2\cdot \mu /3} \] 
 
In our original protocol, Arthur uses $m$ bits of randomness and gets reasonable error. However, if we use the above modification then we can reduced the error to $\propto 2^{-k}$ by using $k \cdot m$ bits of randomness. 

\begin{theorem}
$\MA \subseteq \AM$.
\end{theorem} 
\begin{proof}
Let us think about this intuitively: in $\AM$, Merlin is allowed more opportunities to cheat since Merlin sees Arthur's random string. Let us think about a language $L \in \MA$ and reduce this to a language in $AM$. What does it mean for $L$ to be in $\MA$? It means when given some $x$ as input, Merlin sends Arthur a certificate $y$. If $x \in L$ then for any random string $z$, $\Pr[R(x,y,z) = 1] \geq 3/4$. And if $x \notin L$ then for any random string $z$, $\Pr[R(x,y,z) = 1] \leq 1/4$. Now we need to turn this into a language in $\AM$, that is Arthur must reveal his random string $z$ to Merlin first before Merlin sends over $y$. 

If you just switched the order of Arthur and Merlin, then with high probability, there exists a certificate $y$ such that given a random string $z$, $R(x,y,z) = 1$. Thus the trick is to reduce the likelyhood that $R(x,y,z) = 1$ in general. Let $|y| = m$. We will reduce the error probability to $\leq 1/2^{2m}$. Observe that if $x \in L$ then Merlin can still give a valid $y$. However suppose $x \notin L$ and Merlin attempted to cheat:
\begin{align*}
\Pr_z[\exists y: R(x,y,z) = 1] &\leq \sum_{y} \Pr_z[R(x,y,z) = 1] \\
&=2^{m} \cdot 2^{-2m} = 2^{-m}  
\end{align*}
\end{proof}

\subsection{Logic of $\MA$ and $\AM$}
Consider the quantifiers
\begin{align*}
\exists 	&\qquad \exists \circ \P = \NP \\
\forall 	&\qquad \forall \circ \P = \coNP \\
\Maj_{2/3}	&\qquad \BP \circ \P = \BPP
\end{align*}

Let us use these as follows (this will be a very informal perusal of this topic). Consider $AM$: here the majority machine goes first then a non-deterministic machine goes. This can be translated into $\BP \circ \exists \circ \P$. Similarly for $MA$, we can write $\exists \circ \BP \circ \P$.

\begin{claim}
$\exists \circ \BP \subseteq \BP \circ \exists$
\end{claim}
\begin{proof}
A moment's though and you will realize that this claim is more or less equivalent to the above theorem that $MA \subseteq AM$. 
\end{proof}

\begin{claim}
$\BP \subseteq \exists \circ \forall \circ \P$ and $\BP \subseteq \forall \circ \exists \circ \P$.
\end{claim}
\begin{proof}
Again this is quite similar to something we have done before, namely $\BPP \subseteq \sum_2^p \cap \prod_2^p$.
\end{proof}

Using the above conversions and the above laws we can make quick work of the following theorems.
\begin{theorem}
$\AM \subseteq \prod_2^p$.
\end{theorem}
\begin{proof}
First we translate this statement using the quantifier analogies to get: 
\[\BP \circ \exists \circ \P \subseteq \forall \circ \exists \circ \P.\]
Then using the above laws:
\begin{align*}
\BP \circ \exists \circ \P &\subseteq \forall \circ \exists \circ \exists \circ \P \\
&\subseteq \forall \circ \exists \circ \P
\end{align*}
\end{proof}

\begin{theorem}
$\MA \subseteq \prod_2^p \cap \sum_2^p$.
\end{theorem}
\begin{proof}
We will only show $\MA \subseteq \prod_2^p$ here, $\MA \subseteq \sum_2^p$ is quite similar.
\[\exists \circ \BP \circ \P \subseteq \BP \circ \exists \circ \P \subseteq \forall \circ \exists \circ \exists \circ \P \subseteq \forall \circ \exists \circ \P \]
\end{proof}

\section{Graph Non-isomorphism $\in \AM$}
Recall the Graph Non-isomorphism ($\NISO$). First it is important to remark that the Graph Non-isomorphism problem is a private key protocol, that is the randomness chosen by Arthur is not reveal to Merlin. However we will show that this can be made into a public key protocol ($\in \AM$).

\emph{Note:} (in terms of graphs) there is a difference between an automorphism and an isomorphism.

\emph{Note:} (regarding hash functions) $H = \{h\}$ where $h: U \rightarrow M$, for universe $U$ and $M \subset U$, is a random function family if 
\begin{enumerate}
\item Uniform: $\forall u \in U, \forall a \in M, \Pr_h[h(u) = a] = 1/|M|$.
\item 2-wise Independent: $\forall u, u' \in U, u \neq u', \Pr_h[h(u) = h(u') = a] = 1/|M|^2$.
\end{enumerate}   

Lets consider an example hash function. 
\begin{example}
Let the universe be $U = \{0,1\}^n$ and the range be $M = \{0,1\}^k$. We create a hash function by picking a random 0-1 matrix $A$ of dimension $k \times n$ and a random 0-1 vector $b$ of dimension $k$. Given an input vector $x$,
\[h_{A, b} (x) = A\cdot x + b (\mod 2)\]

\end{example}

\begin{theorem}
$\NISO \in \AM$.
\end{theorem}
\begin{proof}
We will first solve this problem with certain assumption (not true). Suppose that we have two graphs $G_1$ and $G_2$ which do not have nontrivial automorphisms. Construct the set $W = \{\}$
\end{proof}

\begin{theorem}
$\forall k \geq 2$
\end{theorem}
\begin{proof}

\end{proof}
\emph{Remark: } $k$ needs to be a constant. Every time you move $\BP$ in-front of $\exists$, the predicate increases by a polynomial amount.

It is conjectured that $\BPP = \P$ and $\AM = \MA = \NP$. That is: randomness does not give you any more power than just solving the de-randomized problem. 

What evidence do we have for thinking this way? Consider the following theorems involving circuit complexity. 
\begin{theorem}

\end{theorem}

It is also unknown if $\SAT$ requires $2^{\Omega(n)}$ circuit complexity. If this is shown to be true, then we have $\BPP = \P$. $\SAT$ is an unusually tricky problem. Our best algorithm has exponential running time as is our circuit size. 

\section{Polynomial Identity Test}
(The last remaining problem which is in $\BPP$ and unknown if it is in $\P$ or not). The input are arithmetic formulas $f(x_1, ..., x_n)$ and $g(x_1, ..., x_n)$. We want to determine if $f \cong g$. You should remember this theorem from both the previous complexity and linear algebra courses. Hint: we use Schwartz-Zippel Lemma. 

\begin{claim}
Arithmetic formula of size $\leq A$ computes a polynomial of degree $\leq A$ where the size of an arithmetic formula is the number of leaves in its corresponding tree. 
\end{claim}
\begin{proof}
By induction on $A$ (this is actually quite straight forward). The base case of one variable can be computed by a degree one polynomial. Suppose the claim holds for $n = A > 1$. Show that the claim holds for $n = A+1$. Consider any arithmetic formula with size $n$. This formula is of the form $F_1 \oplus F_2$ where $F_1$ and $F_2$ are child formulas of an arithmetic operator $\oplus \in \{\cdot, + \}$.
\end{proof}

Next we want to consider equality for two formulas given by circuits of size $A$. The standard technique turns out to not work since the degree of the arithmetic formula associated with the circuit could be quite large. The trick is: modular arithmetics.   


\begin{lemma}
3-CNF $\phi(x_1, ..., x_n)$ with $m$ clauses can be turned into a polynomial of degree $\leq 3m$.
\end{lemma}
\begin{proof}
We can introduce a polynomial for each elementary boolean statement. Notably: $\lnot x$ is simply $(1-x)$, $p_1 \lor p_2$ is $ $  
\end{proof}

\begin{theorem}
Multi-round $\# \SAT$. 
\end{theorem}
\begin{proof}
You should remember this theorem from before. Let $f(x)$ be the input 3-CNF. Both the prover and verifier calculate the associated boolean polynomial $\phi_f(x)$. The prover sends the verifier the value $s$ which the prover claims is number of satisfying assignments for $f$. The verifier needs to send a set of challenges to the prover to see if he is lying. Remark that $\phi_f(x) = \sum_{x_1=0}^{1} \cdots \sum_{x_n=1}^{1} f(x_1, ..., x_n)$. The verifier creates the polynomial $f_1(r_1) = \sum_{x_2=0}^{1}\cdots \sum_{x_n=0}^{1} f(r_1, x_2, ..., x_n)$ and choses a random $r_1$ in some large range ${1, 2, ..., 2^n}$.    
\end{proof}

\begin{theorem}
$\IP = \PSPACE$
\end{theorem}
\begin{proof}
In one direction $\IP \subseteq \PSPACE$ is quite simple. 

In the other direction $\PSPACE \subseteq \IP$ we need a $\PSPACE$ complete problem problem. Enter $\QBF: \forall x_1 \exists x_2 \cdots$ 
\end{proof}

Multiple interactive proof. 
property testing???

\section{Probabilistically Checkable Proof}
\begin{definition}
$L \in \PCP[r, q]$ for $r, q$ depending on the such that:
\begin{enumerate}
\item (Perfect) Completeness: $x \in L \implies \exists \pi: \Pr[V^{\pi}(x) = 1 ] = 1$ 
\item Soundness: $x \in L \implies \forall \pi: \Pr[V^{\pi}(x) = 1] \leq 1/2$ 
\end{enumerate}
\end{definition}

\begin{theorem}
(PCP Theorem) Every language $L \in \NP$ has a verifier $V$ such that 
\begin{itemize}
\item $V$ is a randomized polytime algorithm (yep, the randomness the crucial here, prevent cheating),
\item $V$ reads only a constant number of symbols in a give proof such that fer every $x \in \{0,1\}^n$: if $x \in L$, then there is a proof $\phi$ such that $V^{\pi}(x)$ accepts with probability at least $2/3$, and if $x \notin L$, every candidate proof $\pi$ is such that $V^{\pi}$ rejects with probability at least $2/3$.
\end{itemize}
Note there that $V^{\pi}$ means that the verifier $V$ has random access to the proof of $\pi$ and can read constantly many symbol of the proof.

Using the above definition we have $\NP = \PCP [\log n, 1]$. 
\end{theorem}
Adaptive vs. Non-adaptive Verifiers: in the non-adaptive case the verifier $V$ choses a random string then using this randomness chooses to read a constant number of bits. Alternatively, the adaptive verifier also chooses a random string but may change the bits read depending on the bits seen so far. The former case is more common.  
\begin{proof}
The backward direction is actually simpler than the forward direction. Consider some language $L \in \PCP[ \log n, 1]$. 

This definition is almost what we want but, in order to show that $L \in \NP$ we need to make sure of two properties: 1. our proof is short (namely polynomial time) and 2. the verifier runs quickly (again poly-time please). These properties are true because: 	
\end{proof}

An alternative formulation of the $\PCP$ theorem is as follows:
\begin{theorem}
(Reformulation of $\PCP$ Theorem) $\exists 0 < \gamma < 1$ and there exists a polytime reduction $R$ from $\SAT$ to $\SAT$ such that if $\phi \in \SAT$ then $R(\phi) \in \SAT$ and if $\phi \notin \SAT$ then $\leq \gamma$ fraction of the clauses in $R(\phi)$ are simultaniously satisfiable.
\end{theorem}

\begin{corollary}

\end{corollary}
\begin{proof}

\end{proof}

\end{document}