%
% This is a borrowed LaTeX template file for lecture notes for CS267,
% Applications of Parallel Computing, UCBerkeley EECS Department.
% Now being used for CMU's 10725 Fall 2012 Optimization course
% taught by Geoff Gordon and Ryan Tibshirani.  When preparing 
% LaTeX notes for this class, please use this template.
%
% To familiarize yourself with this template, the body contains
% some examples of its use.  Look them over.  Then you can
% run LaTeX on this file.  After you have LaTeXed this file then
% you can look over the result either by printing it out with
% dvips or using xdvi. "pdflatex template.tex" should also work.
%

\documentclass[twoside]{article}
\setlength{\oddsidemargin}{0.25 in}
\setlength{\evensidemargin}{-0.25 in}
\setlength{\topmargin}{-0.6 in}
\setlength{\textwidth}{6.5 in}
\setlength{\textheight}{8.5 in}
\setlength{\headsep}{0.75 in}
\setlength{\parindent}{0 in}
\setlength{\parskip}{0.1 in}

%
% ADD PACKAGES here:
%

\usepackage{amsmath,amsfonts,amssymb, graphicx}
\usepackage{color}
\usepackage{comment}
\usepackage{mathtools}

%
% The following commands set up the lecnum (lecture number)
% counter and make various numbering schemes work relative
% to the lecture number.
%
\newcounter{lecnum}
\renewcommand{\thepage}{\thelecnum-\arabic{page}}
\renewcommand{\thesection}{\thelecnum.\arabic{section}}
\renewcommand{\theequation}{\thelecnum.\arabic{equation}}
\renewcommand{\thefigure}{\thelecnum.\arabic{figure}}
\renewcommand{\thetable}{\thelecnum.\arabic{table}}

%
% The following macro is used to generate the header.
%
\newcommand{\lecture}[4]{
   \pagestyle{myheadings}
   \thispagestyle{plain}
   \newpage
   \setcounter{lecnum}{#1}
   \setcounter{page}{1}
   \noindent
   \begin{center}
   \framebox{
      \vbox{\vspace{2mm}
    \hbox to 6.28in { {\bf CMPT 407: Computational Complexity
	\hfill Summer 2017} }
       \vspace{4mm}
       \hbox to 6.28in { {\Large \hfill Lecture #1: #2  \hfill} }
       \vspace{2mm}
       \hbox to 6.28in { {\it Lecturer: #3 \hfill Scribe: #4} }
      \vspace{2mm}}
   }
   \end{center}
   \markboth{Lecture #1: #2}{Lecture #1: #2}

   %{\bf Note}: {\it LaTeX template courtesy of UC Berkeley EECS dept.}

   %{\bf Disclaimer}: {\it These notes have not been subjected to the
   %usual scrutiny reserved for formal publications.  They may be distributed
   %outside this class only with the permission of the Instructor.}
   \vspace*{4mm}
}
%
% Convention for citations is authors' initials followed by the year.
% For example, to cite a paper by Leighton and Maggs you would type
% \cite{LM89}, and to cite a paper by Strassen you would type \cite{S69}.
% (To avoid bibliography problems, for now we redefine the \cite command.)
% Also commands that create a suitable format for the reference list.
%\renewcommand{\citep{•}ite}[1]{[#1]}
\def\beginrefs{\begin{list}%
        {[\arabic{equation}]}{\usecounter{equation}
         \setlength{\leftmargin}{2.0truecm}\setlength{\labelsep}{0.4truecm}%
         \setlength{\labelwidth}{1.6truecm}}}
\def\endrefs{\end{list}}
\def\bibentry#1{\item[\hbox{[#1]}]}

%Use this command for a figure; it puts a figure in wherever you want it.
%usage: \fig{NUMBER}{SPACE-IN-INCHES}{CAPTION}
\newcommand{\fig}[3]{
			\vspace{#2}
			\begin{center}
			Figure \thelecnum.#1:~#3
			\end{center}
	}
% Use these for theorems, lemmas, proofs, etc.
\newtheorem{theorem}{Theorem}[lecnum]
\newtheorem{lemma}[theorem]{Lemma}
\newtheorem{proposition}[theorem]{Proposition}
\newtheorem{claim}[theorem]{Claim}
\newtheorem{corollary}[theorem]{Corollary}
\newtheorem{definition}[theorem]{Definition}
\newtheorem{example}[theorem]{Example}
\newenvironment{proof}{{\bf Proof:}}{\hfill\rule{2mm}{2mm}}

% **** IF YOU WANT TO DEFINE ADDITIONAL MACROS FOR YOURSELF, PUT THEM HERE:

\newcommand\E{\mathbb{E}}
\def\N{\mathbb{N}}
\def\R{\mathbb{R}}
\def\Pr{\mathbf{Pr}}
\def\P{\mathsf{P}}
\def\NP{\mathsf{NP}}
\def\BPP{\mathsf{BPP}}
\def\BQP{\mathsf{BQP}}
\def\RP{\mathsf{RP}}
\def\ZPP{\mathsf{ZPP}}
\def\coRP{\mathsf{coRP}}
\def\SAT{\mathsf{SAT} then }
\def\PolySize{\mathsf{PolySize}}
\def\EXP{\mathsf{EXP}}

\def\MA{\mathsf{MA}}
\def\AM{\mathsf{AM}}

\DeclarePairedDelimiter\ceil{\lceil}{\rceil}
\DeclarePairedDelimiter\floor{\lfloor}{\rfloor}
\DeclarePairedDelimiter\anglebrac{\langle}{\rangle}

\begin{document}
%\lecture{**LECTURE-NUMBER**}{**DATE**}{**LECTURER**}{**SCRIBE**}
\lecture{7}{Randomized Computation (26 June - 2 July)}{Valentine Kabanets}{Lily Li}
%\footnotetext{These notes are partially based on those of Nigel Mansell.}

\section{Randomness and Interaction}
Today we will show that $\BPP \subseteq \MA \subseteq \AM \subseteq \prod_2^p$. Remark that the first $\subseteq$ is actually quite straight forward. Please consider what $\BPP$ and $\MA$ and explain why. Remark the verifier Arthur check $R(x,y,z)$ where $x$ is the input, $y$ is the value the Merlin sends over, and $z$ is the random string.

As before, we can decrease the error probability by trying $k$ times and taking the majority. Let us do the analysis (use Chernoff bound):
\begin{enumerate}
\item $x \in L$: Let the indicator random variable be $X_1, ..., X_k$ where $X_i = 1 $ if A accepts $R(x,y,z)$ and $X_i = 0$ otherwise. Let us calculate the probability that fewer than $k/n$ variables accept:

Thus this probability is quite low, proportional to exponential decay in $k$.
\item $x \notin L$: $\forall y \forall 1 \leq i \leq k, \Pr_{z_i}[R(x,y,z_i) = 1]$ 
\end{enumerate}
\emph{Note: } what does the Chernoff bound actually say? 
 
In our original protocol, Arthur uses $m$ bits of randomness and gets reasonable error. However, if we use the above then we can reduced the error to $\porp 2^}-k
 $ by using $k\cdot m$ bits of randomness. 

\begin{theorem}
$\MA \subseteq \AM$.
\end{theorem} 
\begin{proof}
Let us think about this intuitively: in $\AM$, Merlin is allowed more opportunities to cheat since Merlin sees Arthur's random string.

So let us think about a language $L \in \MA$ and reduce this to a language in $AM$. So we are given some Arthur and Merlin 
\end{proof}

\subsection{Logic of $\MA$ and $\AM$}
Consider the quantifiers
\begin{align*}
\exists 	&\qquad \exists \circ \P = \NP \\
\forall 	&\qquad \forall \circ \P = \coNP \\
\Maj_{2/3}	&\qquad \BP \circ \P = \BPP
\end{align*}

Let us use these as follows. Consider $AM$: here the majority machine goes first then a non-deterministic machine goes. This can be translated into $\BP \circ \exists \circ \P$. Similarly for $MA$, we can write 


Using the above conversions and the above laws we can make quick work of the following theorems.
\begin{theorem}
$\AM \subseteq \prod_2^p$
\end{theorem}
\begin{proof}

\end{proof}

\begin{theorem}
$MA \subseteq \prod_2^p \cap \sigma_2^p$
\end{theorem}
\begin{proof}

\end{proof}

\subsection{Graph Non-isomorphism $\in \AM$}
Recall the Graph Non-isomorphism. First it is important to remark that the Graph Non-isomorphism problem is a private key protocol, that is the randomness chosen by Arthur is not reveal to Merlin. However we will show that this can be made into a public key protocol ($\in \AM$).

\emph{Note:} (in terms of graphs) remark that there is a difference between an automorphism and an isomorphism.

\emph{Note:} (regarding hash functions) $H = \{h\}$ where $h: U \rightarrow M$, for universe $U$ and $M \subset U$, is a random function family if 
\begin{enumerate}
\item[Uniform:] $\forall u \in U, \forall a \in M, \Pr_h[h(u) = a] = 1/|M|$.
\item[2-wise Independent:] $\forall u, u' \in U, u \neq u', \Pr_h[h(u) = h(u') = a] = 1/|M|^2$.
\end{enumerate}   

Lets consider an example hash function. 
\begin{example}

\end{example}

\begin{theorem}
$\NISO \in \AM$.
\end{theorem}
\begin{proof}
We will first solve this problem with certain assumption (not true). Suppose that we have two graphs $G_1$ and $G_2$ which do not have nontrivial automorphisms. Construct the set $W = \{\}$
\end{proof}
\end{document}