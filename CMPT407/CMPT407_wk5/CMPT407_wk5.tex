%
% This is a borrowed LaTeX template file for lecture notes for CS267,
% Applications of Parallel Computing, UCBerkeley EECS Department.
% Now being used for CMU's 10725 Fall 2012 Optimization course
% taught by Geoff Gordon and Ryan Tibshirani.  When preparing 
% LaTeX notes for this class, please use this template.
%
% To familiarize yourself with this template, the body contains
% some examples of its use.  Look them over.  Then you can
% run LaTeX on this file.  After you have LaTeXed this file then
% you can look over the result either by printing it out with
% dvips or using xdvi. "pdflatex template.tex" should also work.
%

\documentclass[twoside]{article}
\setlength{\oddsidemargin}{0.25 in}
\setlength{\evensidemargin}{-0.25 in}
\setlength{\topmargin}{-0.6 in}
\setlength{\textwidth}{6.5 in}
\setlength{\textheight}{8.5 in}
\setlength{\headsep}{0.75 in}
\setlength{\parindent}{0 in}
\setlength{\parskip}{0.1 in}

%
% ADD PACKAGES here:
%

\usepackage{amsmath,amsfonts,amssymb, graphicx}
\usepackage{color}
\usepackage{comment}
\usepackage{mathtools}

%
% The following commands set up the lecnum (lecture number)
% counter and make various numbering schemes work relative
% to the lecture number.
%
\newcounter{lecnum}
\renewcommand{\thepage}{\thelecnum-\arabic{page}}
\renewcommand{\thesection}{\thelecnum.\arabic{section}}
\renewcommand{\theequation}{\thelecnum.\arabic{equation}}
\renewcommand{\thefigure}{\thelecnum.\arabic{figure}}
\renewcommand{\thetable}{\thelecnum.\arabic{table}}

%
% The following macro is used to generate the header.
%
\newcommand{\lecture}[4]{
   \pagestyle{myheadings}
   \thispagestyle{plain}
   \newpage
   \setcounter{lecnum}{#1}
   \setcounter{page}{1}
   \noindent
   \begin{center}
   \framebox{
      \vbox{\vspace{2mm}
    \hbox to 6.28in { {\bf CMPT 407: Computational Complexity
	\hfill Summer 2017} }
       \vspace{4mm}
       \hbox to 6.28in { {\Large \hfill Lecture #1: #2  \hfill} }
       \vspace{2mm}
       \hbox to 6.28in { {\it Lecturer: #3 \hfill Scribe: #4} }
      \vspace{2mm}}
   }
   \end{center}
   \markboth{Lecture #1: #2}{Lecture #1: #2}

   %{\bf Note}: {\it LaTeX template courtesy of UC Berkeley EECS dept.}

   %{\bf Disclaimer}: {\it These notes have not been subjected to the
   %usual scrutiny reserved for formal publications.  They may be distributed
   %outside this class only with the permission of the Instructor.}
   \vspace*{4mm}
}
%
% Convention for citations is authors' initials followed by the year.
% For example, to cite a paper by Leighton and Maggs you would type
% \cite{LM89}, and to cite a paper by Strassen you would type \cite{S69}.
% (To avoid bibliography problems, for now we redefine the \cite command.)
% Also commands that create a suitable format for the reference list.
%\renewcommand{\citep{•}ite}[1]{[#1]}
\def\beginrefs{\begin{list}%
        {[\arabic{equation}]}{\usecounter{equation}
         \setlength{\leftmargin}{2.0truecm}\setlength{\labelsep}{0.4truecm}%
         \setlength{\labelwidth}{1.6truecm}}}
\def\endrefs{\end{list}}
\def\bibentry#1{\item[\hbox{[#1]}]}

%Use this command for a figure; it puts a figure in wherever you want it.
%usage: \fig{NUMBER}{SPACE-IN-INCHES}{CAPTION}
\newcommand{\fig}[3]{
			\vspace{#2}
			\begin{center}
			Figure \thelecnum.#1:~#3
			\end{center}
	}
% Use these for theorems, lemmas, proofs, etc.
\newtheorem{theorem}{Theorem}[lecnum]
\newtheorem{lemma}[theorem]{Lemma}
\newtheorem{proposition}[theorem]{Proposition}
\newtheorem{claim}[theorem]{Claim}
\newtheorem{corollary}[theorem]{Corollary}
\newtheorem{definition}[theorem]{Definition}
\newtheorem{example}[theorem]{Example}
\newenvironment{proof}{{\bf Proof:}}{\hfill\rule{2mm}{2mm}}

% **** IF YOU WANT TO DEFINE ADDITIONAL MACROS FOR YOURSELF, PUT THEM HERE:

\newcommand\E{\mathbb{E}}
\def\N{\mathbb{N}}
\def\Z{\mathbb{Z}}
\def\Q{\mathbb{Q}}
\def\R{\mathbb{R}}
\def\C{\mathbb{C}}
\def\F{\mathbb{F}}
\def\P{\mathsf{P}}
\def\NP{\mathsf{NP}}
\def\coNP{\mathsf{coNP}}
\def\PH{\mathsf{PH}}
\def\EXP{\mathsf{EXP}}
\def\NEXP{\mathsf{NEXP}}
\def\Time{\mathsf{Time}}
\def\SAT{\mathsf{SAT}}
\def\SIZE{\mathsf{SIZE}}
\def\PSPACE{\mathsf{PSPACE}}
\def\NTime{\mathsf{NTime}}
\def\TiSp{\mathsf{TiSp}}
\def\PolySize{\mathsf{PolySize}}
\def\NC_1{\mathsf{NC}_1}
\def\TC_0{\mathsf{TC}_0}
\def\AC_0{\mathsf{AC}_0}

\DeclarePairedDelimiter\ceil{\lceil}{\rceil}
\DeclarePairedDelimiter\floor{\lfloor}{\rfloor}
\DeclarePairedDelimiter\anglebrac{\langle}{\rangle}

\begin{document}
%\lecture{**LECTURE-NUMBER**}{**DATE**}{**LECTURER**}{**SCRIBE**}
\lecture{4}{Randomized Computation (6 - 9 June)}{Valentine Kabanets}{Lily Li}
%\footnotetext{These notes are partially based on those of Nigel Mansell.}

\section{Review}
\begin{theorem}
$\EXP \subset \PolySize \implies \EXP = \Sigma^p_2$. 
\end{theorem}
\begin{proof}
For all $L$ in $\EXP$ there exists a TM $M$, which runs in time $2^{n^c} = t$ for an input $x$ of size $n$. Imagine a $t x t$ grid which describes the operation of $M$. Where each row is a configuration of $M$. This transcript is valid if and only if all windows (three consecutive cells in row $i$ and the associated cell in row $i+1$) are consistent. Consider the function $T: [t] \times [t] \rightarrow \Sigma^*$ where $T(i, j) = \mbox{ cell } j \mbox{ at time } i$. Since we assumed $\EXP \subset \PolySize$ all we need to do is show that $T \in \EXP$. Well, that's pretty obvious, simply execute the TM $M$. Now show that $T \in \Sigma_2$ as follows: $\exists C \forall i,j:$ window $(i, j)$ (in the tableau) is consistent and the tableau ends in an accepting state. 
\end{proof}

\emph{Note: } (by IKW) it is possible to generalize this implication for $\NEXP$, namely $\NEXP \subset \PolySize \implies \NEXP = \Sigma_2$. Proving this is quite a bit more difficult and requires more tool. 

\section{Circuits} 
Let us consider the set of inclusion of circuit complexity: $\mathsf{AC}_0 \subset \mathsf{TC}_0 \subset \mathsf{NC}_1 \subset \PolySize$

\begin{claim}
$\NC_1 = \PolySize$ formula.
\end{claim}
\begin{proof}
$\NC_1 \subseteq \PolySize$ formula is the easy direction. Now lets attempt to show th other direction, $\PolySize \mbox{ formula } \subseteq \NEXP$. Now a normal expansion of a formula $F$ in $x_1, ..., x_n$ might be of depth $O(n)$. But if you think about it you realize that there are not a lot of "stuff" so a long path can be restructured to be made shorter and wider. In particular cut $F$ into two pieces $F_1$ and $F_2$ each of depth approximately half. Let $f(F)$ be the formula associated the the circuit of $F$. Then  
\end{proof}

\subsection{Valiant's Challenge}
Find an explicit function $f: \{0,1\}^{n} \rightarrow \{0,1\}$ that cannot be computed by a circuit of size $O(n)$ and depth $O(\log n)$.

\begin{definition}
A majority gate is defined as follows: $\mbox{maj}_n: \{\}$
\end{definition}

\begin{example}
Let $a_1, a_2, ..., a_n$ be $n$, $n$ digit numbers. We want to show that this problem in the domain of Valiant's Challenge. So we need to demonstrate a $O(\log n)$ circuit of $O(n)$ size to solve this problem. The algorithm here requires a trick as follows:   
\end{example}

\begin{theorem}
Finding the parity of $n$ numbers is in $\TC_0$. 
\end{theorem}
\begin{proof}
First we need to construct a threshold function. 
\end{proof}

\section{$\AC_0$}
\begin{theorem}
The addition of two $n$ bit numbers is in $\AC_0$.
\end{theorem}
\begin{proof}

\end{proof}
\emph{Note: } as it turns out the class $\AC_0$ is exactly the set of all first order formulas! 
\end{document}