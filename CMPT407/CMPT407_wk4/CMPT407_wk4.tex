%
% This is a borrowed LaTeX template file for lecture notes for CS267,
% Applications of Parallel Computing, UCBerkeley EECS Department.
% Now being used for CMU's 10725 Fall 2012 Optimization course
% taught by Geoff Gordon and Ryan Tibshirani.  When preparing 
% LaTeX notes for this class, please use this template.

\documentclass[twoside]{article}
\setlength{\oddsidemargin}{0.25 in}
\setlength{\evensidemargin}{-0.25 in}
\setlength{\topmargin}{-0.6 in}
\setlength{\textwidth}{6.5 in}
\setlength{\textheight}{8.5 in}
\setlength{\headsep}{0.75 in}
\setlength{\parindent}{0 in}
\setlength{\parskip}{0.1 in}

%
% ADD PACKAGES here:
%

\usepackage{amsmath,amsfonts,amssymb, graphicx}
\usepackage{color}
\usepackage{comment}
\usepackage{mathtools}

%
% The following commands set up the lecnum (lecture number)
% counter and make various numbering schemes work relative
% to the lecture number.
%
\newcounter{lecnum}
\renewcommand{\thepage}{\thelecnum-\arabic{page}}
\renewcommand{\thesection}{\thelecnum.\arabic{section}}
\renewcommand{\theequation}{\thelecnum.\arabic{equation}}
\renewcommand{\thefigure}{\thelecnum.\arabic{figure}}
\renewcommand{\thetable}{\thelecnum.\arabic{table}}

%
% The following macro is used to generate the header.
%
\newcommand{\lecture}[4]{
   \pagestyle{myheadings}
   \thispagestyle{plain}
   \newpage
   \setcounter{lecnum}{#1}
   \setcounter{page}{1}
   \noindent
   \begin{center}
   \framebox{
      \vbox{\vspace{2mm}
    \hbox to 6.28in { {\bf CMPT 407: Computational Complexity
	\hfill Summer 2017} }
       \vspace{4mm}
       \hbox to 6.28in { {\Large \hfill Lecture #1: #2  \hfill} }
       \vspace{2mm}
       \hbox to 6.28in { {\it Lecturer: #3 \hfill Scribe: #4} }
      \vspace{2mm}}
   }
   \end{center}
   \markboth{Lecture #1: #2}{Lecture #1: #2}

   %{\bf Note}: {\it LaTeX template courtesy of UC Berkeley EECS dept.}

   %{\bf Disclaimer}: {\it These notes have not been subjected to the
   %usual scrutiny reserved for formal publications.  They may be distributed
   %outside this class only with the permission of the Instructor.}
   \vspace*{4mm}
}
%
% Convention for citations is authors' initials followed by the year.
% For example, to cite a paper by Leighton and Maggs you would type
% \cite{LM89}, and to cite a paper by Strassen you would type \cite{S69}.
% (To avoid bibliography problems, for now we redefine the \cite command.)
% Also commands that create a suitable format for the reference list.
%\renewcommand{\citep{•}ite}[1]{[#1]}
\def\beginrefs{\begin{list}%
        {[\arabic{equation}]}{\usecounter{equation}
         \setlength{\leftmargin}{2.0truecm}\setlength{\labelsep}{0.4truecm}%
         \setlength{\labelwidth}{1.6truecm}}}
\def\endrefs{\end{list}}
\def\bibentry#1{\item[\hbox{[#1]}]}

%Use this command for a figure; it puts a figure in wherever you want it.
%usage: \fig{NUMBER}{SPACE-IN-INCHES}{CAPTION}
\newcommand{\fig}[3]{
			\vspace{#2}
			\begin{center}
			Figure \thelecnum.#1:~#3
			\end{center}
	}
% Use these for theorems, lemmas, proofs, etc.
\newtheorem{theorem}{Theorem}[lecnum]
\newtheorem{lemma}[theorem]{Lemma}
\newtheorem{proposition}[theorem]{Proposition}
\newtheorem{claim}[theorem]{Claim}
\newtheorem{corollary}[theorem]{Corollary}
\newtheorem{definition}[theorem]{Definition}
\newtheorem{example}[theorem]{Example}
\newenvironment{proof}{{\bf Proof:}}{\hfill\rule{2mm}{2mm}}

% **** IF YOU WANT TO DEFINE ADDITIONAL MACROS FOR YOURSELF, PUT THEM HERE:

\newcommand\E{\mathbb{E}}
\def\N{\mathbb{N}}
\def\Z{\mathbb{Z}}
\def\Q{\mathbb{Q}}
\def\R{\mathbb{R}}
\def\C{\mathbb{C}}
\def\F{\mathbb{F}}
\def\P{\mathsf{P}}
\def\NP{\mathsf{NP}}
\def\coNP{\mathsf{coNP}}
\def\PH{\mathsf{PH}}
\def\EXP{\mathsf{EXP}}
\def\NEXP{\mathsf{NEXP}}
\def\Time{\mathsf{Time}}
\def\SAT{\mathsf{SAT}}
\def\SIZE{\mathsf{SIZE}}
\def\PSPACE{\mathsf{PSPACE}}
\def\NTime{\mathsf{NTime}}
\def\TiSp{\mathsf{TiSp}}
\def\poly{\mathsf{poly}}
\def\PolySize{\mathsf{PolySize}}

\DeclarePairedDelimiter\ceil{\lceil}{\rceil}
\DeclarePairedDelimiter\floor{\lfloor}{\rfloor}
\DeclarePairedDelimiter\anglebrac{\langle}{\rangle}

\begin{document}
%\lecture{**LECTURE-NUMBER**}{**DATE**}{**LECTURER**}{**SCRIBE**}
\lecture{3}{Polynomial Hierarchy (29 May - 2 June)}{Valentine Kabanets}{Lily Li}
%\footnotetext{These notes are partially based on those of Nigel Mansell.}

\section{Polynomial Hierarchy}
\begin{definition}
For $i \geq 1$, a language $L$ is in $\sup^P_2$ if there exists a polynomial-time TM $M$ and a polynomial $q$ such that
\[x \in L \iff \exists u_1\in \{0,1\}^{q(|x|)} \forall u_2 \in 
\{0, 1\}^{q(|x|)} \cdots Q_i u_i \in \{0,1\}^{q(|x|)} M(x, u_1, ..., u_i) = 1\]
where $Q_i$ is a $\forall$ or a $\exists$ depending if $i$ is even or odd. \textbf{Polynomial hierarchy (PH)} is $\PH = \cup_{i} \sum^P_i$.
\end{definition}

\emph{The polynomial hierarchy does not collapse.}

\begin{theorem}
The following hold:
\begin{enumerate}
\item For every $i \geq 1$, if $\sum^P_i = \prod^P_i$ then $\PH = \sum^P_i$ i.e. the hierarchy collapses to the $i$th level.
\item If $\P = \NP$ then $\PH = \P$ i.e. the hierarchy collapses to $\P$.
\item If $\NP = \coNP$, then $\PH = \NP$ e.e. the hierarchy collapses to $\NP$.  
\end{enumerate}
\end{theorem}
\begin{proof}
(We will be doing the third one.) Proof by induction on $i$. Consider $L \in \Sigma_{2}^{p}$. Interpret every element $x \in L$ as $\exists y \forall z: R(x,y,z)$. Think of a special language $L' = \phi(x,y)$ such that $R'(\phi (x,y), z) = R(x,y,z)$. Then we have $\forall z: R'(\phi(x,y), z)$ which is an instance of $\Pi_{1}^{p}$. Since we assumed that $\Pi_{1}^{p} = \Sigma_{1}^{p}$, it follows that $\exists w: R"(\phi(x,y), w) = \forall z: R'(\phi(x,y), z)$. Thus we have collapsed $\Sigma_{2}^{p}$ to $\Sigma_{1}^{p}$. In general, if you want to collapse $\Sigma_{i}^{p}$ to $\Sigma_{1}^{p}$ you can do it by induction. Item two is a corollary of item three (\emph{what we just showed}). The proof of item one is very similar to the proof of item three. In the base case we have that $\sum^P_i = \prod^P_i$. Suppose for $n > i$, $\sum^P_i = \sum^P_n$. Show that the same holds for $n+1$. Consider $L \in \sum^P_{n+1}$. Then there exists a polytime polybalanced relation $R$ such that $x$ is in $L$ if and only if  $\exists y_1\forall y_2\exists y_3 \cdots \exists y_{n+1} R(x, y_1, y_2, ..., y_{n+1}$. Consider the language $L' = \{(x,y): \forall y_2\exists y_3 \cdots \exists y_{n+1} R(x,y,y_2, ..., y_{n+1}\}$. The key is to think of $(x,y)$ as a single variable and observe that $L' \in \Pi^P_{n}$. Since $\Pi^P_{n} = \Sigma^P_{n}$, it must also be the case that $L' \in \Sigma^P_{n} = \Sigma^P_i$. Thus $L'$ consists of all $(x,y)$ such that $\exists y_1\forall y_2\exists y_3 \cdots R'((x,y), y_1, y_2, ..., y_{i})$. Thus $x \in L$ if and only if $\exists (y, y_1)\forall y_2\exists y_3 \cdots R'(x, (y, y_1), y_2, ..., y_{i})$ and $L \in \Sigma^P_{i}$.   
\end{proof}

In addition to number 3, we can show the following:
\begin{theorem}
If $\PH = \Sigma^p_i$ for some $i \geq 1$, then $\Sigma^p_i = \Pi^p_i$. 
\end{theorem}
\begin{proof}
To see this, observe that $\Pi^p_i \subset \Sigma^p_i$. Taking the complement of this inclusion, we get that $\Sigma^p_i \subset \Pi^p_i$.
\end{proof}

Together we have that for $i \geq 1$, $\PH = \Sigma^p_i$ if and only if $\Sigma^p_i = \Pi^p_i$.

\emph{An interesting result:} Graph Isomorphism is \emph{not} $\NP$-complete, \textcolor{red}{unless} $\PH$ collapses $\PH = \Sigma_{2}^{p}$. This is proved by interactive proofs. 

Now we are ready to prove the Time-Space trade-off for $\SAT$. 
\begin{theorem}
(Fortnow) $\SAT \notin \TiSp (n^{1.1}, n^{0.1})$. This means you have $O(n^{1.1})$ time and $O(n^{0.1})$ space, then you definitely cannot solve $\SAT$.
\end{theorem}
\begin{proof}
We will first prove the following lemma. 
\begin{lemma}
$\NTime (n) \not\subseteq \TiSp (n^{1.2}, n^{0.2})$.
\end{lemma}
\begin{proof}
This lemma is quite difficult to prove and it requires a good handle on a large number of moving parts. Pay attention! First we define $\Sigma_2$-computation with running time $t$: these are the languages with formulas
\[\phi(x) = \exists y \in \{0,1\}^{O(t(n))}\forall z \in \{0,1\}^{O(t(n))}\psi(x,y,z)\]
where $\psi$ is a predicate computable in deterministic time. Lets begin. Suppose for a contradiction that $\NTime(n) \subseteq \TiSp(n^{1.2}, n^{0.2})$. By padding the inputs to all languages in $\NTime(n)$ we have that $\NTime(n^{10}) \subseteq \TiSp(n^{12}, n^{2})$. We will now show that $\TiSp(n^{12}, n^{2}) \subset \Sigma_2 \Time(n^8)$. That is, by introducing some alternation we can remove the space bound. 

Choose any language $L \subset \TiSp(n^{12}, n^{2})$ and let TM $M$ decide $L$. Let $x$ be an input to $L$ of length $n$. $x$ can be computed in $O(n^{12})$ and $O(n^2)$ time and space simultaneously. Equivalently, $x \in L$ if and only if $\exists c_1, \exists c_2 \cdots \exists c_{n^{12}} R(x, c_1, ..., c_{n^{12}})$. Were each $c_i$ is a configuration on the work tape so $|c_i| \leq O(n^2)$. Divide the configurations into $n^6$ blocks of $n^6$. We will only consider the first and last configurations as well as configurations between two blocks. These $n+1$ configurations are: $c_0, c_{n^6}, c_{2n^6}, ..., c_{n^{12}}$. We forms a $\Sigma_2 \Time (n^8)$ as follows: 
\[\exists (c_0, c_{n^6}, c_{2n^6}, ..., c_{n^{12}}) \forall i \in {1, ..., n^{6}}: c_{in^6} \mbox{ can be reached from } c_{(i-1)n^6} \mbox{ in } O(n^6) \mbox{ steps.} \]
here $(c_0, c_{n^6}, ..., c_{n^{12}})$ is consider one large input. Checking that $c_{in^6}$ can be reached from $c_{(i-1)n^6}$ can be done in $O(n^{8})$ time since you simply need to keep track of the $n^2$ bit configuration tape over $n^6$ time steps. 

Next we need to see that $\NTime (n) \subseteq \Time (n^{1.2})$ then $\Sigma_2 \Time (n^8) \subset \NTime (n^{9.6})$. What we are not going to do is trade alternation for non-determinism. First note that $\Sigma_2 \Time (n^8)$ is of the form $\exists y \in \{0,1\}^{O(|x|^8)}\forall z \in \{0,1\}^{O(|x|^8)}: \psi (x,y,z)$. Just like in the collapsing Polynomial Hierarchy proof we can rewrite this as: all inputs $(x,y)$, $\forall z, \psi (x,y,z)$. If you squint a little this should look a $\coNP$ instance. If $\NTime (n) \subset \Time (n^{1.2})$ then $\mathsf{coNTime} (n) \subseteq \Time (n^{1.2})$ as well (why?). Look carefully and you will notice that $1.2 \times 8 = 9.6$. This is not a coincidence. With padding, we have that $\mathsf{coNTime} (n^8) \subseteq \Time (n^{9.6})$.  

Since we have $\NTime (n) \subset \TiSp (n^{1.2}, n^{0.2})$ by assumption, $\NTime (n) \subset \Time (n^{1.2})$ (just ignore the space constraint), so indeed we can make the above conversion. Through this chain of inclusions we have reached $\NTime (n^{10}) \subset \NTime (n^{9.6})$, but this contradicts non-deterministic time hierarchy so our assumption $\NTime (n) \subset \TiSp (n^{1.2}, n^{0.2})$ is false.   
\end{proof}

Why is this sufficient? Well, for any language in time $\NTime (t(n))$ can be reduced to a $\SAT$-instance of size $O(t \log t)$. Where the reduction itself takes $\poly (\log n)$ space and time (how?). Thus if $\SAT \in \TiSp(n^{1.1}, n^{0.1})$ then $\NTime (n) \subseteq \TiSp (n^{1.1}\poly (\log n), n^{0.1} \poly (\log n))$.
\end{proof}

Using the above techniques we can improve the above time and space bounds but we cannot get to quadratic space unfortunately. There are also a lot of other weird statements in complexity of the form 
\[\mbox{unlikely statement } \implies \mbox{superunlikely statement}\]
here are a sampling: 

\begin{proposition}
(Karp-Lipton) If $\NP \subseteq \PolySize \implies \PH = \Sigma_2^p$.
\end{proposition}
\begin{proof}
Recall from the above, that showing $\Sigma_2^p = \Pi_2^p$ is enough to prove that $\PH = \Sigma_2^p$. But it is actually enough to show that $\Pi_2^p \subseteq \Sigma_2^p$ to show that $\Sigma_2^p = \Pi_2^p$ (because $\exists y\forall z R(x,y,z)$ logically implies $\forall z\exists y R(x,y,z)$ ). By our assumption that $\NP \subseteq \PolySize$, we can find a polynomial size circuit for $\SAT$. Further we have seen that when we have a reasonably fast decider for $\SAT$ we can also solve $\mathsf{Search-SAT}$ on the same complexity order. Namely, there exists a polynomial family of circuits $C_n$ such that on an propositional formal $\psi$ of size $n$, $C_n(\psi)$ finds a satisfying assignment for $\psi$ or outputs a string of zeros if $\psi$ is not satisfiable. 

Now consider a language $L \in \Pi_2^p$. An input $x \in L$ if and only if $\forall y \exists z: R(x,y,z)$ for some polytime polybalanced relation $R$ (by definition). For the language $L'$ as
\[L' = \{(x,y): \exists z R(x,y,z) \}\]
Recognize that $L' \in \NP$ so by assumption there exists polysize circuit family $C_m(x,y)$ with $m = |x| + |y|$ such that $C_m(x,y)$ outputs a satisfying assignment of $R(x,y,z)$ if one exists. View in a different manner,  $L$ accepts $x$ if and only if $\exists C_m \forall y: R(x,y, C_m(x,y))$. Thus $L \in \Sigma_2^p$ as required.  
\end{proof}
\end{document}