%
% This is a borrowed LaTeX template file for lecture notes for CS267,
% Applications of Parallel Computing, UCBerkeley EECS Department.
\documentclass[twoside]{article}
\setlength{\oddsidemargin}{0.25 in}
\setlength{\evensidemargin}{-0.25 in}
\setlength{\topmargin}{-0.6 in}
\setlength{\textwidth}{6.5 in}
\setlength{\textheight}{8.5 in}
\setlength{\headsep}{0.75 in}
\setlength{\parindent}{0 in}
\setlength{\parskip}{0.1 in}

%
% ADD PACKAGES here:
%

\usepackage{amsmath,amsfonts,amssymb, graphicx}
\usepackage{algorithm}
\usepackage{algpseudocode}
\usepackage{color}
\usepackage{comment}
\usepackage{mathtools}

%
% The following commands set up the lecnum (lecture number)
% counter and make various numbering schemes work relative
% to the lecture number.
%
\newcounter{lecnum}
\renewcommand{\thepage}{\thelecnum-\arabic{page}}
\renewcommand{\thesection}{\thelecnum.\arabic{section}}
\renewcommand{\theequation}{\thelecnum.\arabic{equation}}
\renewcommand{\thefigure}{\thelecnum.\arabic{figure}}
\renewcommand{\thetable}{\thelecnum.\arabic{table}}

%
% The following macro is used to generate the header.
%
\newcommand{\lecture}[4]{
   \pagestyle{myheadings}
   \thispagestyle{plain}
   \newpage
   \setcounter{lecnum}{#1}
   \setcounter{page}{1}
   \noindent
   \begin{center}
   \framebox{
      \vbox{\vspace{2mm}
    \hbox to 6.28in { {\bf CSC2429: Proof Complexity
	\hfill Winter 2018} }
       \vspace{4mm}
       \hbox to 6.28in { {\Large \hfill Lecture #1: #2  \hfill} }
       \vspace{2mm}
       \hbox to 6.28in { {\it Lecturer: #3 \hfill Scribe: #4} }
      \vspace{2mm}}
   }
   \end{center}
   \markboth{Lecture #1: #2}{Lecture #1: #2}
   \vspace*{4mm}
}
\renewcommand{\cite}[1]{[#1]}
\def\beginrefs{\begin{list}%
        {[\arabic{equation}]}{\usecounter{equation}
         \setlength{\leftmargin}{2.0truecm}\setlength{\labelsep}{0.4truecm}%
         \setlength{\labelwidth}{1.6truecm}}}
\def\endrefs{\end{list}}
\def\bibentry#1{\item[\hbox{[#1]}]}

\newcommand{\fig}[3]{
			\vspace{#2}
			\begin{center}
			Figure \thelecnum.#1:~#3
			\end{center}
	}
% Use these for theorems, lemmas, proofs, etc.
\newtheorem{theorem}{Theorem}[lecnum]
\newtheorem{lemma}[theorem]{Lemma}
\newtheorem{proposition}[theorem]{Proposition}
\newtheorem{claim}[theorem]{Claim}
\newtheorem{corollary}[theorem]{Corollary}
\newtheorem{definition}[theorem]{Definition}
\newtheorem{example}[theorem]{Example}
\newenvironment{proof}{{\bf Proof:}}{\hfill\rule{2mm}{2mm}}

% **** IF YOU WANT TO DEFINE ADDITIONAL MACROS FOR YOURSELF, PUT THEM HERE:

\newcommand\degree{\mbox{deg}}
\newcommand\rd{\mbox{rd}}
\newcommand\opt{\mathsf{opt}}
\newcommand\trace{\mathsf{trace}}
\newcommand\deter{\mathsf{det}}
\newcommand\gaussian{\mathcal{N}}
\newcommand\Pf{\mathcal{P}}
\newcommand\Ecal{\mathcal{E}}
\newcommand\Fcal{\mathcal{F}}
\newcommand\Hcal{\mathcal{H}}
\newcommand\E{\mathbb{E}}
\newcommand\F{\mathbb{F}}
\newcommand\R{\mathbb{R}}
\newcommand\N{\mathbb{N}}

\newcommand\ClassP{\mathsf{P}}
\newcommand\ClassNP{\mathsf{NP}}
\newcommand\ClasscoNP{\mathsf{coNP}}
\newcommand\SAT{\mathsf{SAT}}
\newcommand\MaxSAT{\mathsf{MaxSAT}}
\newcommand\MaxCUT{\mathsf{MaxCUT}}
\newcommand\SDP{\mathsf{SDP}}

\DeclarePairedDelimiter\ceil{\lceil}{\rceil}
\DeclarePairedDelimiter\floor{\lfloor}{\rfloor}
\DeclarePairedDelimiter\anglebrac{\langle}{\rangle}
\DeclarePairedDelimiter\norm{\parallel}{\parallel}

\begin{document}
\lecture{1}{Introductions}{Toni Pitassi}{Lily Li}

\section{Administration}
Similar courses include: Boak Barak and Steurer/Kothari. Notes for the the Barak and Steurer lectures that tie into these notes will be adding in \textcolor{blue}{blue}. Notes from the Massico Lauria lectures will be added in \textcolor{red}{red}. Choice between two assignments or an assignment and one presentation.

\section{Introductions Proper}
\begin{definition}
A \textbf{proof system} is a \emph{non-deterministic} procedure for generating a family of algorithms. These can be used to \emph{rule out} approaches for solving problem using certain systems e.g. SA or SOS. 
\end{definition}

What is an \emph{extended formulation}? This is a large class of linear programs. There is a clever reduction that reduces this to SA (thus SA results are lower bounds for the LPs). 

The point of a proof complexity \emph{upper bound} is to produce efficient algorithms.

\subsection{Motivating Examples} 
\begin{example}
$\MaxSAT$: given $f = c_1 \land \cdots \land c_m$ which is a 3CNF formula over the variables $x_1, ..., x_n$. Assign $x_i \in \{0,1\}$ to maximize the number of satisfied clauses.

Consider the integer program formulation: let the variables be $x_1, ..., x_n$ and $c_1, ..., c_m$. We want $\max \sum c_i$ where $c_i$ is set to $1$ if it is satisfied. Each clause gets turned into a formula in the usual way. If $c_1 = x_1 \land \lnot x_2 \land x_3$ then $c_1$ becomes the system 
\[x_1 + (1-x_2) + x_3 \geq c_1\]
where $x_i, c_j \in \{0,1\}$. The standard approach is to relax the LP to an IP and solve the LP (of course you add the constraints $x_i, c_j \in [0,1]$). 
\end{example}

How does the $FRACT OPT$ compare to the integral $OPT$? If $f$ is an exact 3CNF and $f$ is unsatisfiable then $OPT \geq \frac{7}{8}m$, where $m$ is the number of clauses. To see this remark that if every triple of terms appeared in the formula $f$ then one out of every eight clauses must be false; general formulas might have fewer clauses and thus fewer false clauses. Unfortunately, if we run the fraction algorithm, $FRACT OPT = m$ by choosing each $x_i = \frac{1}{2}$ (this allows each $c_i = 1$). Generally 
\[FRACT OPT \geq OPT \geq \rd(FRACT OPT)\]
and we compare the quality of our rounded by
\[\frac{\rd(FRACT OPT)}{OPT} = \frac{\rd(FRACT OPT)}{FRACT OPT} \cdot\frac{FRACT OPT}{OPT} \geq K\]
for some constant $K$ and usually this is done by showing that $\frac{\rd(FRACT OPT)}{FRACT OPT} \geq K$. \textcolor{red}{Given a integer program and its relaxation, the ratio
\[\frac{OPT}{FRAC}\]
is the \textbf{integrality gap} of the relaxation.} 

\begin{example}
$\MaxCUT$: given a weighted graph $G = (V,E)$ with weight function $w$ find a cut $S$ which induces the largest edge set. 

The natural formulation here is a quadratic program: let the variables be $y_i$ for every vertex $v_1, ..., v_n$ then 
\[\max \frac{1}{2}\sum_{i < j}w_{i,j}(1 - y_iy_j)\]
where $y_i \in \{-1,1\}$ for $i \in [n]$. Let the optimum value of this program be $\opt(G)$. We attempt to write down a semidefinite program whose value is an \emph{upper bound} on $\opt(G)$. 

Here comes that crazy semi-definite relaxation: replace each $y_i$ by a vector $\mathbf{u}_i \in S^{n-1} = \{\mathbf{x} \in \R^n: \norm{\mathbf{x}} = 1\}$ --- read: the unit sphere in $n-1$ dimensions. Our semi-definite relaxation becomes:
\begin{align}
\label{form:vectorform}
\mbox{Maximize} &\qquad\frac{1}{2}\sum_{i < j} w_{i,j}(1 - \mathbf{u}_i^T\mathbf{u}_j)\\
\mbox{Subject to} &\qquad\mathbf{u}_i \in S^{n-1}, \quad i \in [n]
\end{align}
Note to convert $y_i$ into a vector $\mathbf{u}_i$ it suffices to write $\mathbf{u}_i = (0, ..., 0, y_i)$. Further observe that every solution to the original optimization problem is a solution to our relaxation (the relaxation could contain other solutions as well). We need one more substitution to get to our actual semi-definite program, namely $x_{i,j} = \mathbf{u}_i^T \mathbf{u}_j$. The system now becomes
\begin{align}
\label{form:semidefform}
\mbox{Maximize} &\qquad\frac{1}{2}\sum_{i < j} w_{i,j}(1 - x_{i,j})\\
\mbox{Subject to} &\qquad x_{i,i} = 1 \quad i \in [n] \mbox{ and }  X \succeq 0.
\end{align}
This works because $X$ can be written as $X = U^TU$ where $\mathbf{u}_i$ is the $i^{\text{th}}$ column of $U$. Thus every solution to \ref{form:vectorform} is a solution of \ref{form:semidefform} and vice versa (there is some nuance here regarding the constraints). Then the optimum values of the semidefinite program $\SDP(G) \geq \opt(G)$ and we can find the optimum matrix $X^* \succeq 0$ with $x_{i,i}^* = 1$ for all $i \in [n]$ satisfying 
\[\sum_{i < j} w_{i,j}(1 - x_{i,j}^*) \geq \SDP(G) - \epsilon\]
for every $\epsilon > 0$ in polynomial time (don't ask how... I don't know yet). We find the \emph{Cholesky factorization} of $X^*$ into $(U^*)^TU^*$ and take the columns of $U^*$ to be an almost-optimal solution to the vector problem.

Finally we need to round the vector solutions to get a feasible integer valued solution. Do this by exploiting the geometry of the $\mathbf{u}_i$'s. Since $\mathbf{u}_i \in S^{n-1}$ we will \emph{randomly} pick a vector $\mathbf{p} \in S^{n-1}$ and use the normal to $\mathbf{p}$ to divide the plane into two halves. If $\mathbf{u}_i$ is in one half then round $\mathbf{u}_i$ to $1$ otherwise round $\mathbf{u}_i$ to $-1$. 

The reason why this works relies on a somewhat cryptic probability argument. After taking an appropriate derivative you find that the expected number of cut edges is at least $0.878\opt(G)$. 

\textcolor{blue}{An alternative formulation of the max-cut problem for an $n$-vertex graph $G$ is as the following degree two polynomial $f_G(x)$ where $\mathbf{x}$ is a cut:
\[f_G(x) = \sum_{e_{i,j} \in E} (x_i - x_j)^2.\]
By choosing $x_i \in \{0,1\}$ --- thus choosing $\mathbf{x} \in \{0,1\}^n$ --- we effectively get to choose the partition. Deciding if $c - f_G(\mathbf{x}) \geq 0$ for all $\mathbf{x}$ is the same as deciding if the maximum cut of $G$ is greater than $c$.}  
\end{example}

\textcolor{blue}{The above is a special cases of the problem we want to consider. These \emph{non-negativity over the hyper-cube} problems ask: Given a low-degree polynomial $f:\{0,1\}^n \rightarrow \R$, decided if $f \geq 0$ over the hyper-cube or if there exists a point $\mathbf{x} \in \{0,1\}^n$ such that $f(x) < 0$. 
}

\textcolor{blue}{
\begin{definition}
The \textbf{sum-of-squares algorithm}, when restricted to the above special cases, takes as input the polynomial $f$ and outputs either (1) a proof that $f(\mathbf{x}) \geq 0$ for all $\mathbf{x}$ or (2) a collection of objects which represent points $\mathbf{x}$ such that $f(\mathbf{x}) < 0$. 
\end{definition}
An appropriate certificate is the following:
\begin{definition}
A degree-$d$ \textbf{sum-of-squares certificate} (of non-negativity) for $f: \{0,1\}^n \rightarrow \R$ is a set of polynomials $g_1, ..., g_r: \{0,1\}^n \rightarrow \R$ of degree $\leq \frac{d}{2}$ for some $r \in \N$ such that 
\[f(\mathbf{x}) = g_1^2(\mathbf{x}) + \cdots + g_r^2(\mathbf{x})\]
for every $\mathbf{x} \in \{0,1\}^n$. This set of polynomials is a \emph{degree-$d$ sos proof} of the inequality $f \geq 0$.   
\end{definition}
}

\subsection{Resolution}
Think of this as a system which \emph{tightens} linear programs. But first: proof system basics!

\begin{definition}
A \textbf{proof system} takes as input a set of constraints (CSP) over $x_1, ..., x_n$ usually $x_i \in \{0,1\}$ or $\{-1, 1\}$ (sometimes $x_i \in \F, \R_{\geq 0}$) and then 
\[\Pf(y) \rightarrow F\]
that is the proof system takes as a proof (string) $y$ and outputs what $y$ is a proof of, namely $F$ (generally we think of these are unsatisfiable CSPs). We want our algorithm to run in polynomial time and the proof to be complete and sound. 
\end{definition}

\begin{definition}
A proof system $\Pf$ is \textbf{polynomially-automatizable} if there for all $n$ sufficiently large there exists an algorithm $A$ such that $A(F)$ --- $F$ is $KCNF$ in $n$ variables --- outputs a valid $\Pf$-proof (that is: $\Pf(A(F)) = F$) and such that the running time of $A$ is polynomial in the size of the shortest $\Pf$-proof of $F$.
\end{definition}

\subsubsection{Basics}
Resolution is a refutation based proof system which operates on CNF (with variables which take values $\{0,1\}$). You take clauses and resolve new clauses. If you ever get the empty clause then the initial CNF formula is unsatisfiable. You could have a decision tree or DAG versions, but these are actually truth tables which quit early.  

The way you want to think about this is as follows: replace each $\{0,1\}$ valued $x_i$ with fractional $0 \leq x_i \leq 1$. The constraints are extended when we derive new clauses, and this shrinks the size of the poly-tope containing all the feasible solutions. 

In-terms of automizability, a tree-resolution proofs can be found in time $n^{\log(\mbox{size of the shortest proof})}$ where size is the size of the resolution tree. There is a BenSasson-Wigderson theorem that helps you show the above. 

\begin{example}
Consider a 3CNF formula over $x_1, ..., x_n$ and you want to decide if $F$ is satisfiable or not. Surprisingly the above resolution technique bets the brute-force approach.
\end{example} 
\end{document}