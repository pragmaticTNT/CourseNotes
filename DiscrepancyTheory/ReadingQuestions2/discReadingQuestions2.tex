\documentclass[11pt]{article}

% ===> PACKAGES
\usepackage{amsmath,amsthm,amssymb}
\usepackage{color}
\usepackage{comment}
\usepackage{fancyhdr}
\usepackage{mathtools}
\usepackage[margin=1in]{geometry}
\usepackage{thmtools}

% ===> PARAMETERS
\pagestyle{fancy}

% ===> MACROS
\setlength{\parindent}{0em}
\setlength{\parskip}{0.5em}

%	\def\macrosName{Fill in the content of the macros and use \textit{\\macrosName} whenever necessary}
\def\N{\mathbb{N}}
\def\Z{\mathbb{Z}}
\def\Q{\mathbb{Q}}
\def\R{\mathbb{R}}
\def\C{\mathbb{C}}
\newcommand\A{\mathcal{A}}
\newcommand\disc{\mathsf{disc}}
\newcommand\K{\mathcal{K}}
\newcommand\SSet{\mathcal{S}}
\newcommand\T{\mathcal{T}}
\newcommand\lindisc{\mathsf{lindisc}}
\newcommand\herdisc{\mathsf{herdisc}}

% Use these for theorems, lemmas, proofs, etc.
\newtheorem{theorem}{Theorem}
\newtheorem{lemma}[theorem]{Lemma}
\newtheorem{proposition}[theorem]{Proposition}
\newtheorem{claim}[theorem]{Claim}
\newtheorem{corollary}[theorem]{Corollary}
\newtheorem{definition}[theorem]{Definition}
\newtheorem{problem}{Problem}
\newtheorem{solution}[theorem]{Solution}

\DeclarePairedDelimiter\ceil{\lceil}{\rceil}
\DeclarePairedDelimiter\floor{\lfloor}{\rfloor}
\DeclarePairedDelimiter\anglebrac{\langle}{\rangle}
\DeclarePairedDelimiter\norm{\parallel}{\parallel}

\begin{document}

\lhead{Discrepancy Reading}
\chead{Lily Li}
\rhead{\today}

\section*{Question Set}
% ===> START ASSIGNMENT

\begin{comment}
\begin{problem}
1.3.1 Prove: that if $D(n, \A) = o(n)$ and $D(n, \A) \geq f(n)$ for all $n$, with $\A$ a class of Lebesgue-measure sets in $\R^d$ containing a set $A_0$ with $[0,1]^d \subseteq A_0$ and $f$ satisfying $f(2n) \leq (2-\delta)f(n)$ for some fix $\delta > 0$, then $\disc(n, \A) \geq cf(n)$ holds for infinitely many $n$ with a suitable constant $c = c(\delta) > 0$.
\end{problem}
\begin{proof}

\end{proof}
\end{comment}


\begin{problem}
1.3.2 Let $\K_2$ denote the collection of all closed convex sets in the plane. Show that $D(n, \K_2) = o(n)$ and $\disc(n, \K_2) \geq \frac{n}{2}$.
\end{problem}
\emph{Solution.}
We choose a triangular lattice of points. This obtains a Lebesgue-measure discrepancy of about $\sqrt{n} \in o(n)$ according to the book (page 3). To show that $\disc(n, \K_2) \geq \frac{n}{2}$ consider the following set of points $P$ in the plane where $|P| = n$. Let $P$ be $n$ points evenly spaced about the circle of radius $\frac{1}{2}$ centered at $\left( \frac{1}{2}, \frac{1}{2} \right)$ within the unit cube. Denote this circle by $C$. By the pigeon hole principle every coloring $\chi$ of $C$ will color one at least $\frac{n}{2}$ of the points of $P$ the same color. Let the closed convex polygon formed by these points be denoted by $G$. Then $|\chi(P \cap G)| \geq \frac{n}{2}$.

\begin{comment}
\begin{problem}
\textcolor{blue}{
1.3.3 Find a class $\A$ of measurable sets in the plane such that $D(n, \A) = \Omega(n)$.}
\end{problem}
\emph{Solution.}
\end{comment}


\begin{problem}
4.2.4 Let $A = \frac{1}{2}(H + J)$ be the incidence matrix of set system $\SSet$. Show that the eigenvalue bound is quite weak for $A$, namely that the smallest eigenvalue of $A^TA$ is $O(1)$
\end{problem}
\begin{proof}
Similar to the proof of Proposition 4.4 (Hadamard set system) in the book we first calculate $A^TA$:
\begin{align*}
A^TA &= \frac{1}{4} \left(H^T + J^T\right)\left(H + J\right)\\
&= \frac{1}{4} \left(H^TH + J^TJ + H^TJ + J^TH\right)\\
&= \frac{n}{4} \left(I + J + R + R^T\right)
\end{align*}
where $R$ is the $n\times n$ matrix whose first row is all ones and all the remaining entries are zeros. The eigenvalue bound says that $\disc(A) \geq \disc_2(A) \geq \sqrt{\lambda_n}$ where $\lambda_n$ is the smallest eigenvalue. Since $\lambda_n = \min_{\norm{x} = 1}x^T A^TAx$ we will find an $x$ such that $\norm{x} = 1$ and $x^TA^TAx \in O(1)$. Consider 
\[x = \left\langle -\sqrt{\frac{n-1}{n+3}}, \frac{2}{\sqrt{(n-1)(n+3)}}, \frac{2}{\sqrt{(n-1)(n+3)}}, ..., \frac{2}{\sqrt{(n-1)(n+3)}}\right\rangle.\]
Observe that:
\begin{align*}
x^TA^TAx &= \frac{n}{4}x^T\left( I + J + R + R^T \right)x\\
&= \frac{n}{4}\left( \sum_{i=1}^{n}(x_i)^2 + \left(\sum_{i=1}^{n}x_i\right)^2 + 2x_1\left(\sum_{i=1}^{n}x_i\right) + \right)\\
&= \frac{n}{4}\left( 1 + \frac{n-1}{n+3} - \frac{2(n-1)}{n+3}\right) \\
&= \frac{n}{4}\left( 1 - \frac{(n-1)}{n+3}\right) \\
&= \frac{n}{n+3} \leq 1
\end{align*}
Thus the eigenvalue bound is not tight.
\end{proof}

\pagebreak
\begin{problem}
4.3.2 Find a set system $(X, \SSet)$ and a set $A \subset X$ such that $\disc(\SSet) = 0$ but $\disc(\SSet \cup \{A\})$ is arbitrarily large.
\end{problem}
\emph{Solution.}
Let $X = Y \cup Z$ where $Y$ and $Z$ are two disjoint sets of size $n$. Let $\SSet$ be the set of all subsets of $X$ of the form $Y' \cup Z'$ where $Y' \subset Y$, $Z' \subset Z$ and $|Y'| = |Z'|$. As we have discussed, $\disc(\SSet) = 0$ by coloring elements of $Y$ color $1$ and the elements of $Z$ color $-1$. Next let $A = Y$. Consider $\disc(\SSet \cup \{A\})$. Consider any coloring $\chi$ of $X$. Suppose without loss of generality that $Y' \subseteq Y$ such that for all $y \in Y'$, $\chi(y) = 1$ and $|Y'| \geq \frac{n}{2}$. Either $|Y'| - (|Y| - |Y'|) \geq \frac{n}{4}$ or $|Y'| - (|Y| - |Y'|) < \frac{n}{4}$. In the former case $\disc(\SSet \cup \{A\}) \geq \frac{n}{4} \in O(n)$. In the latter case $|Y'|$ and $|Y| - |Y'|$ (i.e. the number of elements colored $1$ and $-1$ respectively by $\chi$) differ by at most $\frac{n}{4}$. Consider a subset $Z' \subset Z$ such that $|Z'| = |Y'|$. And let $Z' = Z_{1} \cup Z_{-1}$ such that for all $z \in Z_{1}$, $\chi(z) = 1$ and for all $z \in Z_{-1}$, $\chi(z) = -1$ Either $|Z_1| \geq \frac{n}{4}$ or $|Z_{-1}| \geq \frac{n}{4}$. In the former case $\chi(Y' \cup Z') \geq \frac{n}{4}$. In the latter case $\chi((Y - Y') \cup Z'') \geq \frac{n}{4}$ where $Z'' \subset Z'$, $Z_{-1} \subset Z''$ and $|Z''| = |Y - Y'|$. Since $Y' \cup Z'$ and $(Y-Y') \cup Z'' \in \SSet$, in both case we have $\disc(\SSet \cup \{A\}) \geq \frac{n}{4} \in O(n)$.


\begin{problem}
4.3.5 Let $A$ be an $m \times n$ real matrix and set
\[\Delta = \max_{w \in \{-1,0,1\}^n}\min_{x \in \{-1, 1\}^n} \norm{A(x - w)}_{\infty}\]
(\emph{linear discrepancy with weights $-1$, $0$, $1$}). Prove that $\lindisc(A) \leq 2\Delta$.
\end{problem}
\begin{proof}
\begin{comment}
We will show that $\Delta \geq \herdisc(A)$. Then by Theorem 4.6 in the book ($\lindisc(A) \leq 2\cdot \herdisc(A)$), we have that $\lindisc(A) \leq 2\Delta$. First observe that $\Delta \geq \disc(A)$ since $\disc(A) = \min_{x \in \{-1, 1\}^n} \norm{A(x - \mathbf{0})}_{\infty}$ where $\mathbf{0}$ is the zero vector. Next consider $A'$ which consists of a subset of the columns of $A$; this is equivalent to restricting the set system to a subset of the universe $X$. Observe that the colorings $\chi$ forms a bijection with the $n$-dimensional vectors $x$. Let $x = \anglebrac{x_1, x_2, ..., x_n}$. Suppose we wish to restrict to the columns $c_1, c_2,..., c_k$. If we take $w = \anglebrac{w_1, w_2, ..., w_n}$ where $w_{c_i} = 0$ and $w_j = -x_j$, then we have such a restriction. Thus $\Delta \geq \disc(A')$. Since $\Delta$ is greater than or equal to all restrictions of the universe $X$, $\Delta \geq \herdisc(A)$ as required. 
\end{comment}
\textcolor{blue}{From our discussion two weeks ago, we observe that the proof of $\lindisc \leq 2\cdot \herdisc$ is essentially the same as that for $\lindisc(A) \leq 2\cdot \Delta$. We will reproduce the proof here in our own words to make sure it sticks. $\Delta$ can be interpreted as the moment when $B = \cup_{a \in \{-1,1\}^n} U+a$ covers the points $w \in \{-1,0,1\}^n$. We claim that at the same moment $C = \cup_{a \in \{-1,1\}^n} 2U+a$ covers all points $x \in [-1,1]^n$. Since $C$ is closed (all boundary points of $C$ are in $C$), we can restrict our attention to the dense set of dyadic rationals of the form $v = \frac{z}{2^k} \in [-1,1]^n$ where $z \in \Z^n$. The base case holds since $B$ covers the points $w$ described above. Consider any point $v = \frac{z}{2^k} \in [-1,1]^n$. Then $2v = \frac{z}{2^{k-1}} \in [-2,2]^n$. There exists $b \in \{-1,1\}^n$ such that $2v-b \in [-1,1]^n$. Since $2v - b = \frac{z + 2^{k-1}b}{2^{k-1}}$, by the induction hypothesis there exists some $a \in \{-1,1\}^n$ such that $2v - b \in U + a$ and $v \in U + \frac{a+b}{2}$ (1). $\frac{a+b}{2} \in \{-1,0,1\}^n$. Since $B$ covers all such points, there exists some $c \in \{-1,1\}^n$ such that $\frac{a+b}{2} \in U + c$. By substituting this into (1) we have
\[v \in U + (U + c) = 2U + c.\]
It is important to note that $U$ is convex here. Otherwise it might not be the case that elements of $U$ are closed under addition.}
\end{proof}

\textcolor{blue}{
\begin{problem}
4.3.4 Show that the $1 \times n$ matrix $[1,2,2^2,..., 2^{n-1}]$ has hereditary discrepancy $2^{n-1}$ and linear discrepancy at most $2$. 
\end{problem}
\begin{proof}
(I know you talked about solution to this problem last time, but I just want to write it down to make sure.) Consider the subset of the universe consisting of the element $2^{n-1}$. Coloring this element $\pm 1$ will yield hereditary discrepancy of $2^{n-1}$. Consider any $w \in [0,1]^n$ and let $[1,2,2^2, ..., 2^{n-1}]\cdot w = k \in [0, 2^{n}-1]$. Observe that each of the $2^{n}$ elements of $x \in {-1,1}^n$ yields a distinct integer in the range $\{0,1,..., 2^{n}-1\}$. Thus $[1,2,2^2, ..., 2^{n-1}] \cdot x$ is within distance $1$ to any value $k$. Now if we consider the definition of linear discrepancy where $w \in [-1,1]^n$ and $x \in \{-1,1\}^n$, we see that the linear discrepancy is at most $2$.     
\end{proof}
}
% ===> END ASSIGNMENT
\end{document}