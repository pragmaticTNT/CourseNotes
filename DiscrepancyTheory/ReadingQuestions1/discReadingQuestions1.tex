\documentclass[11pt]{article}

% ===> PACKAGES
\usepackage{amsmath,amsthm,amssymb}
\usepackage{color}
\usepackage{comment}
\usepackage{fancyhdr}
\usepackage{mathtools}
\usepackage[margin=1in]{geometry}
\usepackage{thmtools}

% ===> PARAMETERS
\pagestyle{fancy}

% ===> MACROS
\setlength{\parindent}{0em}
\setlength{\parskip}{1em}

%	\def\macrosName{Fill in the content of the macros and use \textit{\\macrosName} whenever necessary}
\def\N{\mathbb{N}}
\def\Z{\mathbb{Z}}
\def\Q{\mathbb{Q}}
\def\R{\mathbb{R}}
\def\C{\mathbb{C}}
\newcommand\A{\mathcal{A}}
\newcommand\disc{\mathsf{disc}}
\newcommand\K{\mathcal{K}}
\newcommand\SSet{\mathcal{S}}
\newcommand\T{\mathcal{T}}

% Use these for theorems, lemmas, proofs, etc.
\newtheorem{theorem}{Theorem}
\newtheorem{lemma}[theorem]{Lemma}
\newtheorem{proposition}[theorem]{Proposition}
\newtheorem{claim}[theorem]{Claim}
\newtheorem{corollary}[theorem]{Corollary}
\newtheorem{definition}[theorem]{Definition}
\newtheorem{problem}{Problem}
\newtheorem{solution}[theorem]{Solution}

\DeclarePairedDelimiter\ceil{\lceil}{\rceil}
\DeclarePairedDelimiter\floor{\lfloor}{\rfloor}
\DeclarePairedDelimiter\anglebrac{\langle}{\rangle}

\begin{document}

\lhead{Discrepancy Reading}
\chead{Lily Li}
\rhead{\today}

\section*{Question Set}
% ===> START ASSIGNMENT

\begin{comment}
\begin{problem}
1.3.1 Prove: that if $D(n, \A) = o(n)$ and $D(n, \A) \geq f(n)$ for all $n$, with $\A$ a class of Lebesgue-measure sets in $\R^d$ containing a set $A_0$ with $[0,1]^d \subseteq A_0$ and $f$ satisfying $f(2n) \leq (2-\delta)f(n)$ for some fix $\delta > 0$, then $\disc(n, \A) \geq cf(n)$ holds for infinitely many $n$ with a suitable constant $c = c(\delta) > 0$.
\end{problem}
\begin{proof}

\end{proof}

\begin{problem}
1.3.2 Let $\K_2$ denote the collection of all closed covex sets in the plane. Show that $D(n, \K_2) = o(n)$ and $\disc(n, \K_2) \geq \frac{n}{2}$.
\end{problem}
\emph{Solution.}

\begin{problem}
Find a class $\A$ of measurable sets in the place such that $D(n, \A) = \Omega(n)$.
\end{problem}
\emph{Solution.}
\end{comment}

\begin{problem}
4.1.2 (Discrepancy of the product of set systems) Let $\SSet$ and $\T$ be set systems on finite sets. Let $\SSet \times \T = \{S \times T: S \in \SSet, T \in T\}$.
\end{problem}
\begin{enumerate}
\item Show that $\disc(\SSet \times \T) \leq \disc(\SSet)\disc(\T)$.
\begin{proof}
Let $\disc(\SSet) = d_{\SSet}$ and $\disc(\T) = d_{\T}$. Then there exists colorings $\chi_{\SSet}$ of $X_{\SSet}$ and $\chi_{\T}$ of $X_{\T}$ such that $\disc(\chi_{\SSet}, \SSet) = d_{\SSet}$ and $\disc(\chi_{\T}, \T) = d_{\T}$. We will construct a coloring $\chi: X_{\SSet} \times X_{\T} \rightarrow \{-1, 1\}$ such that for all $S \times T \subseteq \SSet \times \T$, $|\chi(S \times T)| \leq d_{\SSet} \cdot d_{\T}$. For $s \in X_{\SSet}$ and $t \in X_{\T}$ define: $\chi(s, t) = \chi_{\SSet}(s) \cdot \chi_{\T}(t)$. Observe that
\begin{align*}
|\chi(S \times T)| &=  \left| \sum_{(s,t) \in \SSet \times \T} \chi(s, t) \right| = \left| \sum_{(s,t) \in \SSet \times \T}\chi_{\SSet}(s) \cdot \chi_{\T}(t)\right| \\
&\leq \left| \sum_{s \in \SSet} \chi_{\SSet}(s)\right| \cdot \left| \sum_{t\in\T} \chi_{\T}(t) \right| \\
&= |\chi_{\SSet}(S)|\cdot|\chi_{\T}(T)| = d_{\SSet}\cdot d_{\T}
\end{align*}
where the inequality on the second line follows from Cauchy-Schwarz. Thus $\disc(\SSet \times \T) \leq \disc(\SSet)\disc(\T)$ as required.
\end{proof}
\item Find an example with $\disc(\SSet) > 0$ and $\disc(\SSet \times \SSet) = 0$. 

\emph{Solution.} Well, its not fair now since I looked at the answers but something like 
\[\{\{0,1\},\{0,2\},\{0,3\},\{0,4\}, \{1,2,3,4,5,6\}\}\] works since you have all the properties that you need.
\end{enumerate}


\begin{comment}
\begin{problem}
4.2.4 Let $A = \frac{1}{2}(H + J)$ be the incidence matrix of set system $\SSet$. Show that the eigenvalue bound is quite weak for $A$, namely that the smallest eigenvalue of $A^TA$ is $O(1)$
\end{problem}
\begin{proof}

\end{proof}


\begin{problem}
4.3.2 Find a set system $(X, \SSet)$ and a set $A \subset X$ such that $\disc(\SSet) = 0$ but $\disc(S \cup \{A\})$ is arbitrarily large.
\end{problem}
\emph{Solution.}
\end{comment}


\begin{problem}
4.3.3 Show that the set system $\SSet = \{\{1\}, \{2\}, ..., \{n\}, \{1, 2, ..., n\}\}$ has hereditary discrepancy $1$ and linear discrepancy at least $2 - \frac{2}{n+1}$.
\end{problem}
\emph{Solution.} To show that the hereditary discrepancy is $1$ it suffices to show that for set system $\SSet_1 = \{\{1\}\}$ and set system $\SSet_2 = \{\{1, ..., n\}\}$ it is the case that $\disc(\SSet_1) = 1$ and that $\disc(\SSet_2) = 1$ (all other subsets of $\SSet$ can be reduced to these two). $\disc(\SSet_1) = 1$ is trivial. In the case that $\disc(\SSet_2)$ simply color half of the $-1$ and the other half $1$. With this coloring $|\chi(\{1, ..., n\})| \leq 1$ with equality when $n$ is odd. 

To show that the linear discrepancy is at least $2-\frac{2}{n+1}$ we will consider the adjacency matrix of $\SSet$. This is just the $n \times n$ identity matrix with an extra row of all ones. Consider the weight vector $\mathbf{w} = \anglebrac{1 - \frac{2}{n+1}, ..., 1 - \frac{2}{n+1}}$. Observe that:
\[
\begin{bmatrix}
1 & 0 & 0 & \cdots & 0\\
0 & 1 & 0 & \cdots & 0\\
0 & 0 & 1 & \cdots & 0\\
\vdots & \vdots & \vdots & \ddots & \vdots\\
0 & 0 & 0 & \cdots & 1\\
1 & 1 & 1 & \cdots & 1\\
\end{bmatrix}
\cdot \left(\mathbf{x} - \mathbf{w}\right)
=
\begin{bmatrix}
x_1 - 1 + \frac{2}{n+1}\\
x_2 - 1 + \frac{2}{n+1}\\
x_3 - 1 + \frac{2}{n+1}\\
\vdots \\
x_n - 1 + \frac{2}{n+1}\\
\sum_{i=1}^{n} \left(x_i - 1 + \frac{2}{n+1}\right)
\end{bmatrix}
\]
If $x_i = -1$ then the $i^{th}$ element would be $\geq 2 - \frac{2}{n+1}$. However, ever if $x_i = 1$ for every $i$ then the last entry would be $\geq 2 - \frac{2}{n+1}$. Thus the linear discrepancy is $\geq 2 - \frac{2}{n+1}$.
% ===> END ASSIGNMENT
\end{document}