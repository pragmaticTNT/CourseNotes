%
% This is a borrowed LaTeX template file for lecture notes for CS267,
% Applications of Parallel Computing, UCBerkeley EECS Department.
% Now being used for my course notes.
%

\documentclass[twoside]{article}
\setlength{\oddsidemargin}{0.25 in}
\setlength{\evensidemargin}{-0.25 in}
\setlength{\topmargin}{-0.6 in}
\setlength{\textwidth}{6.5 in}
\setlength{\textheight}{8.5 in}
\setlength{\headsep}{0.75 in}
\setlength{\parindent}{0 in}
\setlength{\parskip}{0.1 in}

%
% ADD PACKAGES here:
%

\usepackage{amsmath,amsfonts,amssymb, graphicx}
\usepackage{color}
\usepackage{comment}
\usepackage{mathtools}

%
% The following commands set up the lecnum (lecture number)
% counter and make various numbering schemes work relative
% to the lecture number.
%
\newcounter{lecnum}
\renewcommand{\thepage}{\thelecnum-\arabic{page}}
\renewcommand{\thesection}{\thelecnum.\arabic{section}}
\renewcommand{\theequation}{\thelecnum.\arabic{equation}}
\renewcommand{\thefigure}{\thelecnum.\arabic{figure}}
\renewcommand{\thetable}{\thelecnum.\arabic{table}}

%
% The following macro is used to generate the header.
%
\newcommand{\lecture}[4]{
   \pagestyle{myheadings}
   \thispagestyle{plain}
   \newpage
   \setcounter{lecnum}{#1}
   \setcounter{page}{1}
   \noindent
   \begin{center}
   \framebox{
      \vbox{\vspace{2mm}
    \hbox to 6.28in { {\bf Introduction to Machine Learning
	\hfill Fall 2017} }
       \vspace{4mm}
       \hbox to 6.28in { {\Large \hfill Lecture #1: #2  \hfill} }
       \vspace{2mm}
       \hbox to 6.28in { {\it Lecturer: #3 \hfill Scribe: #4} }
      \vspace{2mm}}
   }
   \end{center}
   \markboth{Lecture #1: #2}{Lecture #1: #2}
   \vspace*{4mm}
}

\renewcommand{\cite}[1]{[#1]}
\def\beginrefs{\begin{list}%
        {[\arabic{equation}]}{\usecounter{equation}
         \setlength{\leftmargin}{2.0truecm}\setlength{\labelsep}{0.4truecm}%
         \setlength{\labelwidth}{1.6truecm}}}
\def\endrefs{\end{list}}
\def\bibentry#1{\item[\hbox{[#1]}]}

%Use this command for a figure; it puts a figure in wherever you want it.
%usage: \fig{NUMBER}{SPACE-IN-INCHES}{CAPTION}
\newcommand{\fig}[3]{
			\vspace{#2}
			\begin{center}
			Figure \thelecnum.#1:~#3
			\end{center}
	}
% Use these for theorems, lemmas, proofs, etc.
\newtheorem{theorem}{Theorem}[lecnum]
\newtheorem{lemma}[theorem]{Lemma}
\newtheorem{proposition}[theorem]{Proposition}
\newtheorem{claim}[theorem]{Claim}
\newtheorem{corollary}[theorem]{Corollary}
\newtheorem{definition}[theorem]{Definition}
\newtheorem{example}[theorem]{Example}
\newenvironment{proof}{{\bf Proof:}}{\hfill\rule{2mm}{2mm}}

% **** IF YOU WANT TO DEFINE ADDITIONAL MACROS FOR YOURSELF, PUT THEM HERE:

\newcommand\X{\mathbf{X}}
\newcommand\XT{\mathbf{X}^T}
\newcommand\x{\mathbf{x}}
\newcommand\y{\mathbf{y}}
\newcommand\wstar{\mathbf{w}^*}
\newcommand\w{\mathbf{w}}
\newcommand\R{\mathbb{R}}
\newcommand\I{\mathbf{I}}
\newcommand\Var{\mathsf{Var}}
\newcommand\Cov{\mathsf{Cov}}
\newcommand\z{\mathbf{z}}

\DeclarePairedDelimiter\ceil{\lceil}{\rceil}
\DeclarePairedDelimiter\floor{\lfloor}{\rfloor}
\DeclarePairedDelimiter\anglebrac{\langle}{\rangle}
\DeclarePairedDelimiter\norm{\parallel}{\parallel}

\begin{document}
\lecture{1}{Introductions}{Ethan Fetaya}{Lily Li}

\section{Administration}
Our Instructor is Ethan Fetaya. A recommended textbook is \emph{Understanding Machine Learning: from Theory to Algorithm} by Shal Shalev-Shwartz, Shal Ben-David.

For graduate students there is a project instead of a final (it is still worth $30\%$). Details are forth-coming.

Assignments are due by 22:00 on the day (observe that it is \emph{NOT} due at 23:59).

\section{Introduction to Machine Learning}
Learning is: goal orientated skill acquisition. As computer scientists we use code to solve our problems, however, there are some complications: the solution maybe difficult to formalize, the task maybe mailable.

\subsection{ML Categories}
\begin{enumerate}
\item Supervised learning: correct outputs known. Given input $X$ and output $Y$. We assume there is an distribution $D$ on $X \times Y$. There is a also a lose function: $l: Y \times Y \rightarrow R$. We are given a set of $m$ independent and identically distributed input-output pairs. What we want is a \textbf{hypothesis} function $f_\textbf{w}(\textbf{x}) = w_0 + w_1x_1 + \cdots + w_dx_d$ for $\mathbf{w} \in \R^{d+1}$ which minimizes the loss $l$. 
\item Unsupervised Learning: find structure in data.
\item Online Learning: data keeps coming in there are no separate learning and validation data.
\item Reinforcement Learning: maximize future reward.
\end{enumerate}

\subsection{ML Viewpoints}
\begin{enumerate}
\item Agnostic approach: minimize loss on unseen data.
\item Discriminative approach: fit with some parametric model.
\item Generative approach: fit $P(x, y: \theta)$ by parametric model then use the model to improve $P(x,y: \theta)$.
\end{enumerate}

\section{Linear Regression}
Let the inputs be: $\mathbf{x} \in \R^d$. The outputs are $\mathbf{y}$ where $y \in \R$. And the input-output pair $(x^{(1)}, y^{(1)}), ..., (x^{(k)}, y^{(k)})$.

Any (fixed) transformation $\phi(x) \in \R$, run linear regressions with features $\phi(x)$. Observe that a polynomial $w_0 + w_1x + \cdots + w_d x^d$ are actually linear in the parameters $w_i$! (Just consider the $x^i$ as the features.)

The common loss is often set to $L_2 = (y - \hat{y})^2$. Unfortunately this loss model punishes infrequent large mistakes. The best prediction under this model is the mean. Another loss model is $L_1 = |y - \hat{y}|$. The best prediction here is the median. Another loss function is the Huber loss (which stitches together $L_1$ and $L_2$ losses --- $L_2$ near the zero point) in a smooth way.

Note: we often include the bias in $\mathbf{x}$ as follows: $x^{(i)} = [1, x^{(i)}_1, ..., x^{(i)}_d]$ where our prediction is $\mathbf{x}^T\mathbf{w}$. 

The target vector is $\mathbf{y} = [y^{(1)}, ..., y^{(N)}]^T$. The feature vectors are: $\mathbf{f}^{(j)} = [\mathbf{x}^{(1)}_j, ..., \mathbf{x}^{(N)}_j]^T$ and the design matrix $\mathbf{X}$ has the property that $\mathbf{X}_{i,j} = \mathbf{x}^(i)_j$. It is easiest think about $f^{(j)}$ as the $j^{th}$ column and $x^{(i)}$ as the $i^{th}$ row. 

\begin{theorem}
The optimal $\mathbf{w}$, with respect to $L_2$, is 
\[\mathbf{w}^* = \arg\min \sum_{i=1}^{N} \left( y^{(i)} - \mathbf{w}^Tx^{(i)} \right)^2 = (\mathbf{X}^T\mathbf{X})^{-1} \mathbf{X}^T \mathbf{y}\]
\end{theorem}
\begin{proof}
Remark that the prediction vector is $\mathbf{\hat{y}} = \mathbf{Xw}$. Consider the $L_2$ loss of $\mathbf{w}$. All we really need to do here is simplify this equation then take the derivative with respect to $\mathbf{w}$. Lets do just that 
\begin{align*}
L_2(\mathbf{w}) &= \parallel \mathbf{y} - \mathbf{\hat{y}}\parallel^2 \\
&= \parallel \mathbf{y} - \mathbf{Xw} \parallel^2 \\
&= (\mathbf{y} - \mathbf{Xw})^T \cdot (\mathbf{y} - \mathbf{Xw}) \\
&= \mathbf{y}^T\mathbf{y} + \mathbf{w}^T\mathbf{X}^T\mathbf{Xw} - 2\mathbf{w}^T\mathbf{X}^T\mathbf{y}
\end{align*}
Now if we take the partial derivative with respect to $\mathbf{w}$ we get:
\[\nabla L(\mathbf{w}^*) = 2\mathbf{X}^T\mathbf{X}\mathbf{w}^* - 2\mathbf{X}^T\mathbf{y} = 0.\]
By rearranging the above equation we obtain $\mathbf{w}^* = \left( \mathbf{X}^T\mathbf{X} \right)^{-1}\mathbf{X}^T\mathbf{y}$. 
\end{proof}

Since our prediction is $\mathbf{\hat{y}} = \mathbf{X}^T\mathbf{y}$ and we calculated that the optimal weights is $\mathbf{X}^T\mathbf{X}\mathbf{w}^* = \mathbf{X}^T\mathbf{y}$, this gives us some information about the residual $r = \mathbf{y} - \mathbf{\hat{y}}$. In particular if we substitute  for the value of $\mathbf{\hat{y}}$ and multiply by $\mathbf{X}^T$, then we see that $\mathbf{X}r = 0$. This means that the residual (think of this as the remainder) is orthogonal to each $\mathbf{x}^{(i)}$.

\textcolor{red}{There is something else here about covariance that I didn't quite catch.}

\subsection{Regularization}
Typically, if you over fit data, the model tends to have terms with large norm. Thus it is a good idea to introduce a regularizer term $R(\mathbf{w})$. The modified optimal model $w^* = \arg\min_{\mathbf{w}} L_S(\mathbf{w}) + R(\mathbf{w})$.

Commonly used regularizers include:
\begin{enumerate}
\item $L_2$ regularization: 
\[R(\mathbf{w}) = \frac{\lambda}{2} \mathbf{w}^T \mathbf{w}.\]
Note that the analytic solution is $\mathbf{w}^* = (\XT\X + \lambda \I)^{-1}\X\y$ (why). Normally we do not regularize the bias $w_0$ and we use validation/ cross-validation to find a good $\lambda$.
\item $L_1$ regularization: 
\[ R(\w) = \lambda \norm{\w}_1 = \lambda \sum |w_i|.\]
This regularization is convex (Simple Gaussian Distribution) but has no analytic solution. It also tends to induce \emph{sparse} solutions. 
\end{enumerate} 

\section{Tutorial: Probability}
Sample space $\Omega$: set of all possible outcomes of the experiment. Observation $\omega \in \Omega$ are sample outcomes, realizations, or elements. $E \subset \Omega$ are subsets of the sample space.

\begin{definition}
\textbf{Joint Probability} of $A$ and $B$ is $P(A, B)$. Note that the joint probability is simply $P(A, B) = P(A|B)P(B) = P(B|A)P(A)$.

Events $A$ and $B$ are \textbf{conditionally independent} given $C$ if $P(A, B|C) = P(B|A, C)P(A|C) = P(B|C)P(A|C)$.
\end{definition}

\begin{definition}
\textbf{Marginalization (Sum Rule)} $P(X) = \sum_{Y} P(X, Y)$. 

\textbf{Law of Total Probability} $P(X) = \sum_{Y} P(X, Y)$.
\end{definition}

Bayes' Rules can be reworded to be more useful for Machine Learning as follows:
\begin{align*}
P(\theta | x) &= \frac{P(x|\theta)P(\theta)}{P(x)} \\
\mbox{Posterior} &= \frac{\mbox{Likelihood}*\mbox{Prior}}{\mbox{Evidence}} \\
\mbox{Posterior} &\propto \mbox{Likelihood} \times \mbox{Prior}
\end{align*}

where $P(x|\theta)$ is the likelihood, $P(\theta)$ is the prior, and $x$ is the evidence.

The difference between discrete and continuous random variables is the difference between summation and integration when marginalizing. Further:
\begin{enumerate}
\item[Discrete:] distribution defined by probability mass function (PMF). Marginalization: $p(x) = \sum_{y} p(x,y)$.
\item[Continuous:] distribution defined by probability density function (PDF). Marginalization: $p(x) = \int_{y}p(x,y) dy$.
\end{enumerate}

The mean $\mu$ is the \textbf{First Moment} and the variance $\sigma^2$ is the \textbf{Second Moment}. Variance is defined as $\int_{-\infty}^{\infty} (x - \mu)^2 p(x) dx$. With a little bit of algebra, you can work out that $\Var[x] = E[x^2] - E[x]^2$.

\subsection{Covariance Matrix}
Let $\x$ be a $D$-dimensional vector and $\mu$ be a $D$-dimensional mean vector. Then $\Sigma$ is a $D\times D$ covariance matrix with determinant $|\Sigma |$. Note that the $(i,j)$ entry of $\Sigma$ is the covariance of $x_i, x_j$: 
\[\Cov(x_i, x_j) = E[(x_i - \mu_i)(x_j - \mu_j)] = E[(x_ix_j)] - \mu_i\mu_j.\]
Thus the diagonal entries are the variance of each element. $\Sigma$ has the property that it is positive and semi-definite. 

\textbf{Whitening} is a linear transformation that takes a vector of random variables with a known covariance changes these into a set of new random variables with the identity matrix as its covariance. This means that the random variables are all independent of one another. Formally, the random $d$-dimensional vector is $\x  = (x_1, ..., x_d)^T$, the mean $\mu = E[\x] = (\mu_1, ..., \mu_d)^T$ and postive definite $d \times d$ covariance matrix $\Cov(\x) = \sigma$ into. $\x$ is changed into $\z = (z_1, ..., z_d)^T = W\x$ with \emph{white} covariance matrix, $\Cov(\z) = \I$. 
\end{document}