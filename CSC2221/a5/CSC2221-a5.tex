\documentclass[11pt]{article}

% ===> PACKAGES
\usepackage{amsmath,amsthm,amssymb}
\usepackage{algorithm}
\usepackage{algpseudocode}
\usepackage{color}
\usepackage{comment}
\usepackage{fancyhdr}
\usepackage{mathtools}
\usepackage[margin=1in]{geometry} 
\usepackage{thmtools}

% ===> PARAMETERS
\pagestyle{fancy}

% ===> MACROS
\setlength{\parindent}{0em}
\setlength{\parskip}{0.5em}

%	\def\macrosName{Fill in the content of the macros and use \textit{\\macrosName} whenever necessary}
\newcommand\lbufread{\textsc{L-buffer-read}}
\newcommand\lbufwrite{\textsc{L-buffer-write}}

% Use these for theorems, lemmas, proofs, etc.
\newtheorem{theorem}{Theorem}
\newtheorem{lemma}[theorem]{Lemma}
\newtheorem{proposition}[theorem]{Proposition}
\newtheorem{claim}[theorem]{Claim}
\newtheorem{corollary}[theorem]{Corollary}
\newtheorem{definition}[theorem]{Definition}
\newtheorem{problem}{Problem}

\DeclarePairedDelimiter\ceil{\lceil}{\rceil}
\DeclarePairedDelimiter\floor{\lfloor}{\rfloor}
\DeclarePairedDelimiter\anglebrac{\langle}{\rangle}

\begin{document}

\lhead{CSC 2221}
\chead{Xinyuan Li: 1000858244}
\rhead{\today}

\section*{Assignment 5}
% ===> START ASSIGNMENT
An $L$-buffer object supports two operations, $\lbufread$, which takes no input, and $\lbufwrite$, which takes a bit as input. An $\lbufread$ operation returns the last $L$ inputs to $\lbufwrite$ operations that were previously performed on the object, in order from least recent to most recent. If fewer than $L$ have been performed on the object, $\lbufread$ returns the sequence of all inputs to $\lbufwrite$ operations that have been performed, in order from least recent to most recent.

Determine the consensus number of the $L$-buffer object, for all $L \geq 1$.

\begin{claim}
$L$-buffer objects have consensus number $L$.
\end{claim}
\begin{proof}
$L$-buffer objects can solve $L$ processor wait-free consensus by Lemma \ref{lem:lprocwaitfree}. There is no wait-free consensus algorithm for $L+1$ processors using only $L$-buffer objects and read/write registers by Lemma \ref{lem:nol+1procwaitfree}. Thus $L$-buffer objects have consensus number $L$ as claimed. 
\end{proof}

\begin{lemma}
\label{lem:lprocwaitfree}
$L$-buffer objects can solve $L$-processor wait-free consensus.
\end{lemma}
\begin{proof}
Lemma \ref{lem:lprocwaitfreebin}, shows that it is possible to solve $L$-processor wait-free \emph{binary} consensus using only $L$-buffer objects and read/write registers. Using a technique discussed in-class, we combine several binary consensus gadgets to solve $L$-processor wait-free consensus for arbitrary inputs.

Let the $L$ processors be $p_1$, $p_2$, ..., $p_{L}$ and let processor $p_i$ get input $u_i$. Further, let $m$ be the maximum length of any $u_i$ in binary representation. 

The processors want to reach consensus for an $m$-bit binary string. First, each processor $p_i$ writes down $u_i$ in a set of $m$ binary registers $r^{(i)}_1, ..., r^{(i)}_m$ (padding with zero as necessary). Once $p_i$ is finished writing, it sets a check register $c_i$ to be $1$. Then $p_i$ tries to reach consensus for each bit of the output by interacting with $m$ $L$-buffer objects $l_1, ..., l_m$ in sequential order. It does so using the $\textsc{L-binary-consensus}(l_j, r^{(i)}_j)$ operation which runs the wait-free binary consensus algorithm on $L$-buffer $l_j$ with input $r^{(i)}_j$. If $p_i$ ever ``losses" some $l_j$ i.e. the consensus for the $j^{th}$ $L$-buffer object is different from the $j^{th}$ bit of $u_i$, then $p_i$ looks at every $u_k$ for $k \neq i$ and adopts some $u_k$ such that the first $j$ bits of $u_k$ match the consensus for $l_1, ..., l_j$ and whose check bit $c_k$ is set to $1$ (such a $u_k$ must exist since processors must first write down their input before they interact with the $L$-buffer objects). See Algorithm \ref{pseudocode:lproccons} for the pseudo-code.

\begin{algorithm}
	\caption{$L$-processor consensus using only $L$-buffer objects and read/write registers: code for processor $p_i$.}
    \label{pseudocode:lproccons}
    \begin{algorithmic}[1]
	\State $r^{(i)}_1, ..., r^{(i)}_m \leftarrow u_i$
	\State $c_i \leftarrow 1$
	\For{$j$ from $1$ to $m$}
		\State $v \leftarrow \textsc{L-binary-consensus}(l_j, r^{(i)}_j)$
		\If{$v \neq r^{(i)}_j$}
			\For{$k$ from $1$ to $L$ with $k \neq i$}
				\If{$r^{(k)}_1 = l_1$, ..., $r^{(k)}_j = l_j$ and $c_k = 1$}
					\State $r^{(i)}_j \leftarrow r^{(k)}_j$, ..., $r^{(i)}_m \leftarrow r^{(k)}_m$
				\EndIf
			\EndFor
		\EndIf
	\EndFor 
	\State $u \leftarrow r^{(i)}_1, ..., r^{(i)}_m$
	\State return $u$
    \end{algorithmic}
\end{algorithm}

Observe that the termination condition holds trivially. Since the partial sequence $r^{(i)}_1, ..., r^{(i)}_j$ is the prefix of some input for every $1 \leq j \leq m$, the validity condition holds. Since the wait-free binary consensus algorithm works when upto $L-1$-processors fail, all processors must agree on all $m$ bits so the agreement condition holds as well. 

\end{proof}

\begin{lemma}
\label{lem:lprocwaitfreebin}
$L$-buffer objects can solve $L$-processor wait-free binary consensus.
\end{lemma}
\begin{proof}
We present an algorithm which solves wait-free \emph{binary} consensus for $L$ processors using only $L$-buffer objects and read/write registers. Let the set of processors be $p_1, ..., p_{L}$ and let $u_i$ be the input to $p_i$. Further let $r_i$ be a single-writer register associated with processor $p_i$ and $B$ be a global $L$-buffer object.

The high-level description of the algorithm is as follows. Each processor $p_i$ will write $u_i \in \{0,1\}$ to $B$ then invoke $\lbufread$ to get a vector $V_i$ (if $V_i$ is not empty then $V_i[0]$ is the least recent bit in $B$). If $V_i$ is empty then $p_i$ outputs $u_i$. Otherwise $p_i$ outputs $V_i[0]$. See Algorithm \ref{pseudocode:lprocbincons} for the associated pseudo-code.

\begin{algorithm}
	\caption{$L$-processor binary consensus using only $L$-buffer objects and read/write registers: code for processor $p_i$.}
    \label{pseudocode:lprocbincons}
    \begin{algorithmic}[1]
	\State $\lbufwrite(B, u_i)$
	\State $V_i \leftarrow \lbufread(B)$
	\If{$V_i = \emptyset$}
		\State return $u_i$
	\Else
		\State return $V_i[0]$
	\EndIf 
    \end{algorithmic}
\end{algorithm}

The termination condition holds trivially. The validity condition holds since every bit written to $B$ is an input of some $p_i$. To see that the agreement condition holds, suppose that $p_i$ was the first to write $u_i$ to $B$. Then $p_i$ outputs $u_i$. For all $p_j$ where $i \neq j$, $V_j[0] = u_i$ since there can be at most $L$ bits in $B$; every non-faulty processor writes once so there can be at most $L$ write operations.   
\end{proof}

\begin{lemma}
\label{lem:nol+1procwaitfree}
There is no $L+1$ wait-free consensus algorithm for $L+1$ processors using only $L$-buffer objects and read/write registers.
\end{lemma}
\begin{proof}
Suppose for a contradiction there exists wait-free consensus algorithm for $L+1$ processors using only $L$-buffer objects and read/write registers (to simplify the argument we can restrict our consideration to binary-consensus). We proceed by the valency argument. As discussed in-class, there exists an initial bivalent configuration. We will show that there does not exist a critical configuration $C$ during the execution of the algorithm. Since the algorithm cannot end in a bivalent configuration, the algorithm cannot end, contradicting the termination condition.

Let $p_1$, ..., $p_{L+1}$ be the $L+1$ processors. Suppose $C$ is a critical configuration. Let $\alpha_i$ be a step by processor $p_i$ from $C$. Since $C$ is a critical configuration, we can partition the processors into two disjoint, non-empty sets $U$ and $V$ such that for $p_u \in U$, configuration $C_u = C\alpha_u$ is $0$-valent and for $p_v \in V$, configuration $C_v = C\alpha_v$ is $1$-valent. Let $p_i \in U$ and $p_j \in V$. If $\alpha_i$ and $\alpha_j$ are $\lbufread$ or $\lbufwrite$ operations to different $L$-buffer objects then $\alpha_i$ and $\alpha_j$ are commutative. From the perspective of $p_i$ or $p_j$ configurations $C_i \alpha_j$ and $C_j\alpha_i$ are indistinguishable. This is a contradiction since $C_i$ and $C_j$ have different valency. The same is true if $\alpha_i$ and $\alpha_j$ are $\lbufread$ operations to the same $L$-buffer object. Next suppose without loss of generality that $\alpha_i$ is an $\lbufread$ operation and $\alpha_j$ is an $\lbufwrite$ operation both to $L$-buffer $B$. Then $C_j \stackrel{p_j}{\sim} C_i\alpha_j$. $p_j$ should output the same value starting from $C_j$ and starting form $C_i \alpha_j$, but this is again a contradiction. Thus for every processor $p_k$, $\alpha_k$ must be an $\lbufwrite$ operation to the same $L$-buffer object $B$. 

Observe that $p_i \in U$ and $p_j \in V$ must write different values to $B$. Otherwise the operations are again commutative. Thus, WLOG, assume operation $\alpha_i$ writes $0$ and operation $\alpha_j$ writes $1$ to $B$. Let $k_1, ..., k_{L-1}$ be some permutation of the indices $\{1, ..., L+1\} - \{i, j\}$. Consider the configurations $C' = C_i\alpha_{k_1}\cdots\alpha_{k_{L-1}}$ and $C'' = C_j\alpha_i\alpha_{k_1}\cdots\alpha_{k_{L-1}}$. There are $L$ objects in $B$ in configuration $C'$ and $L+1$ objects in $B$ in configuration $C''$. Any $\lbufread$ operation executed by $p_i$ will return the most recent $L$ bits in $B$ which are the same in both cases so $C' \stackrel{p_i}{\sim} C''$. 

Since in all cases a pair of configurations originating from $C$ is indistinguishable to some processor even though the configurations have different valency, $C$ cannot be a critical configuration. Thus there does not exists a wait-free consensus algorithm for $L+1$ processors using $L$-buffer objects and read/write registers.  
\end{proof}

% ===> END ASSIGNMENT
\end{document}