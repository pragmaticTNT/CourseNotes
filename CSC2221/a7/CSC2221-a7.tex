\documentclass[11pt]{article}

% ===> PACKAGES
\usepackage{amsmath,amsthm,amssymb}
\usepackage{algorithm}
\usepackage{algpseudocode}
\usepackage{color}
\usepackage{comment}

\usepackage{graphicx}
\usepackage{epstopdf}
	\epstopdfsetup{update}

\usepackage{fancyhdr}
\usepackage{mathtools}
\usepackage[margin=1in]{geometry} 
\usepackage{thmtools}

% ===> PARAMETERS
\pagestyle{fancy}

% ===> MACROS
\setlength{\parindent}{0em}
\setlength{\parskip}{0.5em}
\algdef{SE}[DOWHILE]{Do}{doWhile}{\algorithmicdo}[1]{\algorithmicwhile\ #1}%

%	\def\macrosName{Fill in the content of the macros and use \textit{\\macrosName} whenever necessary}
\newcommand\writeF{\textsc{write}}
\newcommand\updateF{\textsc{update}}
\newcommand\scanF{\textsc{scan}}
\newcommand\COLLECT{\textsc{COLLECT}}
\newcommand\READ{\textsc{READ}}
\newcommand\N{\mathbb{N}}
\newcommand\R{\mathbb{R}}

% Use these for theorems, lemmas, proofs, etc.
\newtheorem{theorem}{Theorem}
\newtheorem{lemma}[theorem]{Lemma}
\newtheorem{proposition}[theorem]{Proposition}
\newtheorem{claim}[theorem]{Claim}
\newtheorem{corollary}[theorem]{Corollary}
\newtheorem{definition}[theorem]{Definition}
\newtheorem{problem}{Problem}

\DeclarePairedDelimiter\ceil{\lceil}{\rceil}
\DeclarePairedDelimiter\floor{\lfloor}{\rfloor}
\DeclarePairedDelimiter\anglebrac{\langle}{\rangle}

\begin{document}

\lhead{CSC 2221}
\chead{Xinyuan Li: 1000858244}
\rhead{\today}

\section*{Assignment 7}
% ===> START ASSIGNMENT
\paragraph{Problem Statement}
Give an algorithm for solving $\epsilon$-approximate agreement with any $0 < \epsilon < 1$ for inputs in $[0,1]$ that has $O\left(\log_2\left(\frac{1}{\epsilon}\right)\right)$ step complexity. Prove your algorithm is correct.

\paragraph{Algorithm.}
\emph{(This algorithm is inspired by the one presented in-class as well as the one shown in the Attiya textbook, page 352.)} Let the processes be denoted $p_0, ..., p_{n-1}$ where each process $p_i$ receives input $x_i \in [0,1]$. The processes share $r = \ceil{\log_2(1/\epsilon)} + 1$ atomic snapshot objects denoted $S_0, ..., S_{r-1}$. These objects support: (1) $\updateF(S, i, v)$ where $S$ refers to the object, $i$ is the index to update, and $v$ is updated value and (2) $\scanF(S)$. Initially all entries of $S_i$ are set to $\bot$.

The high-level description of the algorithm is as follows: each process begins with their input $x_i$ as their desired output. For iteration $k$ where $k$ ranges from $0$ to $r-1$, each process write their desired output into the atomic snapshot object $S_k$ and scans $S_k$. They change their desired output to the mid-point between the largest and smallest values seen in $S_k$. After $r$ iterations, each process outputs their desired value. The associated pseudo-code is shown in Algorithm \ref{pseudocode:approxagv1}. 

Each process takes a constant number of steps for each iteration of the for-loop. Since the for-loop runs for $\ceil{\log_2(1/\epsilon)} + 1 \in O(\log_2(1/\epsilon))$ iterations, the algorithm has the required step complexity.

\begin{algorithm}
	\caption{Approximate agreement with $O\left(\log_2\left(\frac{1}{\epsilon}\right)\right)$ step complexity: code for process $p_i$.}
    \label{pseudocode:approxagv1}
    \begin{algorithmic}[1]
    \State $r \leftarrow \ceil{\log_2(1/\epsilon)}+1$
	\State $v_{i,0} \leftarrow x_i$
	\For{$k$ from $0$ to $r-1$}
		\State $\updateF(S_k, i, v_{i,k})$
		\State $V_{i,k} \leftarrow \scanF(S_k)$		
		\State $m_{i,k} \leftarrow \min\{x \in V_{i,k}: x \neq \bot\}$
		\State $M_{i,k} \leftarrow \max\{x \in V_{i,k}: x \neq \bot\}$
		\State $v_{i,k+1} \leftarrow \frac{m_{i,k} + M_{i,k}}{2}$
	\EndFor
	\State output $v_{i,r}$
	\end{algorithmic}
\end{algorithm}

\paragraph{Proof of Correctness.} The termination condition is easy to see, every process executes a for-loop for a finite number of iterations and every iteration is wait-free. Validity holds by Lemma \ref{lem:validityv1}. $\epsilon$-agreement holds by Lemma \ref{lem:agreementv1}. 

\begin{lemma}
\label{lem:validityv1}
Algorithm \ref{pseudocode:approxagv1} satisfies the validity condition. 
\end{lemma}
\begin{proof}
We will show that the value $v_{i,k}$ of each process $p_i$ satisfies the validity condition at iteration $k$ of the for-loop by induction on the number of iterations. Initially $v_{i,0} = x_i$ for every $p_i$ so the base case holds. Suppose the validity condition holds for iterations $0, 1, ..., k$ on all processes. We show that the condition must hold for iteration $k+1$. During iteration $k+1$, the process updates $S_{k+1}$ with $v_{i,k+1}$, scans $S_{k+1}$, and finds the minimum and maximum elements ($m_{i,k+1}$ and $M_{i,k+1}$ respectively) in $V_{i,k+1}$. By the induction hypothesis all entries of $V_{i,k+1}$ satisfies the validity condition. Thus there exists inputs $x_1, x_2, x_3, x_4$ such that $x_1 \leq m_{i,k+1} \leq x_2$ and $x_3 \leq M_{i,k+1} \leq x_4$. Then the desired value for iteration $r+1$ satisfies:
\[\min\{x_1, x_3\} \leq \frac{m_{i,k+1} + M_{i,k+1}}{2} \leq \max\{x_2, x_4\}.\]
Since $\min\{x_1, x_3\}$ and $\max\{x_2, x_4\}$ are inputs of some process, $v_{i,k+2}$ at iteration $k+1$ satisfies the validity condition for every process $p_i$.
\end{proof}

\begin{lemma}
\label{lem:agreementv1}
Algorithm \ref{pseudocode:approxagv1} satisfies the $\epsilon$-agreement condition. 
\end{lemma}
\begin{proof}
Let $v_{i,k}$ and $v_{j,k}$ be the desired values of process $p_i$ and $p_j$ in iteration $k$. We show that $|v_j - v_i| \leq \frac{1}{2^{k}}$ by induction on the number of iterations. The base case is true since the inputs are taken from the interval $[0,1]$. Suppose the claim is true for iterations $0, 1, ..., k$. Show that the claim holds on iteration $k+1$. Consider process $p_i$ in iteration $k+1$ of the for-loop. $p_i$ writes the current desired value $v_{i,k+1}$ into $S_{k+1}$ and scans $S_{k+1}$ into $V_{i,k+1}$. Let any $v_{j,k+1}\in V_{i,k+1}$ such that $j \neq i$ and $v_{j,k+1} \neq \bot$. Consider the views $V_{i,k}$ and $V_{j,k}$ from iteration $k$. Note that $v_{i,k+1} = \frac{m_{i,k} + M_{i,k}}{2}$ and $v_{j,k+1} = \frac{m_{j,k} + M_{j,k+1}}{2}$ where $m_{i,k}$, $M_{i,k}$, $m_{j,k}$, $M_{j,k}$ are the largest and smallest elements in $V_{i,k}$ and $V_{j,k}$ respectively. Since atomic snapshot objects are linearizable, either $p_j$ or $p_i$ performed the $\scanF$ operation first. Assume the former. By the induction hypothesis, $|m_{i,k} - M_{i,k}| \leq \frac{1}{2^{k}}$, $|m_{j,k} - M_{j,k}| \leq \frac{1}{2^{k}}$, $|m_{i,k} - M_{j,k}| \leq \frac{1}{2^{k}}$, and $|m_{j,k} - M_{i,k}| \leq \frac{1}{2^{k}}$. Since $p_i$ performed $\scanF$ after $p_j$, $V_{i,k}$ contains values $m_{j,k}$, $M_{j,k}$, and $v_{j,k}$. Either $\frac{1}{2^{k+1}} \geq |v_{i,k+1} - m_{i,k}| \geq |v_{i,k+1} - m_{j,k}| \geq |v_{i,k+1} - v_{j,k+1}|$ or $\frac{1}{2^{k+1}} \geq |v_{i,k+1} - M_{i,k}| \geq |v_{i,k+1} - M_{j,k}| \geq |v_{i,k+1} - v_{j,k+1}|$. See Figure \ref{fig:approxagv1epag}.

\begin{figure}[ht]
\centering
	\includegraphics[scale=1]{approxagv1epag}
\caption{Situation where $\frac{1}{2^{k+1}} \geq |v_{i,k+1} - M_{i,k}| \geq |v_{i,k+1} - M_{j,k}| \geq |v_{i,k+1} - v_{j,k+1}|$.}
\label{fig:approxagv1epag}
\end{figure}

The same is true if $p_j$ performed $\scanF$ after $p_i$ as $p_j$ will set $v_{j,k+1}$ such that $|v_{i,k+1}-v_{j,k+1}| \leq \frac{1}{2^{k+1}}$. Thus $|v_{j,k} - v_{i,k}| \leq \frac{1}{2^k}$ at iteration $k$ for all $k \in \N$. Since $\frac{1}{2^r} \leq \epsilon$ where $r$ is the number of iterations, $\epsilon$-agreement condition is satisfied after $r$ iterations of the for-loop. 
\end{proof}

\emph{Extra credit: extend the algorithm to run in 
\[O\left(\log_2\left(\frac{\max\{|x_0|, ..., |x_{n-1}|\}}{\epsilon}\right)\right).\] 
steps when the inputs $x_0, ..., x_{n-1}$ are arbitrary real numbers.}

\paragraph{Algorithm.}
\emph{(Referenced the Attiya textbook page 355.)} As before we have $n$ processes $p_0, ..., p_{n-1}$ where the input to $p_i$ is $x_i$. We use one atomic snapshot objects $S$ where each entry $S[i]$ is $\anglebrac{x_i,c_i,v_i}$ which are the input, current round counter, and value of process $p_i$ respectively. The operations on $S$ are the same as before. Initially all values in $S$ are set to $\bot$. Further define the modified logarithm function $f: \R_{\geq 0} \rightarrow \R_{\geq 0}$ as 
\[f(x) = \begin{cases}
0 &\mbox{if } x = 0 \\
\log\left(\frac{x}{\epsilon}\right) &\mbox{otherwise}
\end{cases}\]

The general idea for this algorithm is the similar to the one for the previous problem. At each round a process $p_i$ reads the values in the atomic snapshot object $S$ and updates its local value to be the midpoint of the range of the observed values. However, since processes are initially unaware of the range of the inputs, $p_i$ must keep a round counter $c_i$ in $S[i]$ to let all the other processes know which step it is on. At every iteration, $p_i$ will look at a process $p_j$ with the largest counter $c_j$ and jump forward to round $c_j$ by taking the mid-point of values from all processes at round $c_j$. Further, $p_i$ will keep an estimate of the maximum number of rounds, $r_{\max}$, using all the values of $x_j$ it has seen and exit the loop when $c_i$ reaches $r_{\max}$. See Algorithm \ref{pseudocode:approxagv2} for the associated pseudo-code.

Every iteration of the while-loop takes a constant number of steps. Since the while-loop terminates after $\ceil{\log\left(2\cdot\frac{\max\{|x_0|, ..., |x_{n-1}|\}}{\epsilon}\right)}+1$ iterations by Lemma \ref{lem:terminationv2}, the step complexity is as required. 

\begin{algorithm}
	\caption{Approximate agreement with $O\left(\log_2\left(\frac{\max\{|x_0|, ..., |x_{n-1}|\}}{\epsilon}\right)\right)$ step complexity: code for process $p_i$.}
    \label{pseudocode:approxagv2}
    \begin{algorithmic}[1]
	\State $v_i \leftarrow x_i$
	\State $c_i \leftarrow 1$
	\State $r_{\max} \leftarrow 2$
	\While{$r_{\max} > c_i$}
		\State $\updateF(S, i, \anglebrac{x_i, c_i, v_i})$
		\State $V \leftarrow \scanF(S)$
		\State $m_x \leftarrow \min\{x_k: \anglebrac{x_k, c_k, v_k} \in V \mbox{ and } x_k \neq \bot\}$
		\State $M_x \leftarrow \max\{x_k: \anglebrac{x_k, c_k, v_k} \in V \mbox{ and } x_k \neq \bot\}$
		\State $r_{\max} \leftarrow f(M_x-n_x)$
		\State $c_i \leftarrow \max\{c_k: \anglebrac{x_k, c_k, v_k} \in V \mbox{ and } c_k \neq \bot\}$
		\State $m_v \leftarrow \min\{v_k: \anglebrac{x_k, c_k, v_k} \in V \mbox{ and } c_k = c_i\}$
		\State $M_v \leftarrow \max\{v_k: \anglebrac{x_k, c_k, v_k} \in V \mbox{ and } c_k = c_i\}$
		\State $v_i \leftarrow \frac{m_v + M_v}{2}$
		\State $c_i \leftarrow c_i + 1$
	\EndWhile
	\State output $v_i$
	\end{algorithmic}
\end{algorithm}

\paragraph{Proof of Correctness}
The termination condition holds by Lemma \ref{lem:terminationv2}. The validity condition holds by an argument similar to Lemma \ref{lem:validityv1} and the $\epsilon$-agreement condition holds by an argument similar to Lemma \ref{lem:agreementv1}.

\begin{lemma}
\label{lem:terminationv2}
For every process $p_i$, Algorithm \ref{pseudocode:approxagv2} terminates after $c$ iterations where 
\[c = \left\lceil\log\left(2\cdot\frac{\max\{|x_0|, ..., |x_{n-1}|\}}{\epsilon}\right)\right\rceil+1.\] 
\end{lemma}
\begin{proof}
Suppose that there exists a process $p_i$ whose while-loop iterates more than $c$ times. The exist condition is $c_i \geq r_{\max}$. Since $c_i$ is initially $1$ and increments by at least $1$ in every iteration, $r_{\max}$ must be greater than $c$. However $r_{\max} = f(M_x-n_x)$ where $M_x$ is the largest input that $p_i$ has seen and $m_x$ is the smallest. Observe that $M_x - m_x \leq 2\cdot\max\{x_0, ..., x_{n-1}\}$. Thus $f(M_x - m_x) \leq c$. This contradicts our assumption so the while-loop of every process terminates in at most $c$ steps. 
\end{proof}

% ===> END ASSIGNMENT
\end{document}