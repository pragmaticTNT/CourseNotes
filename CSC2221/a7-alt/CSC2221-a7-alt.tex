\documentclass[11pt]{article}

% ===> PACKAGES
\usepackage{amsmath,amsthm,amssymb}
\usepackage{algorithm}
\usepackage{algpseudocode}
\usepackage{color}
\usepackage{comment}

\usepackage{graphicx}
\usepackage{epstopdf}
	\epstopdfsetup{update}

\usepackage{fancyhdr}
\usepackage{mathtools}
\usepackage[margin=1in]{geometry} 
\usepackage{thmtools}

% ===> PARAMETERS
\pagestyle{fancy}

% ===> MACROS
\setlength{\parindent}{0em}
\setlength{\parskip}{0.5em}
\algdef{SE}[DOWHILE]{Do}{doWhile}{\algorithmicdo}[1]{\algorithmicwhile\ #1}%

%	\def\macrosName{Fill in the content of the macros and use \textit{\\macrosName} whenever necessary}
\newcommand\writeF{\textsc{write}}
\newcommand\updateF{\textsc{update}}
\newcommand\scanF{\textsc{scan}}
\newcommand\COLLECT{\textsc{COLLECT}}
\newcommand\READ{\textsc{READ}}
\newcommand\N{\mathbb{N}}
\newcommand\R{\mathbb{R}}

% Use these for theorems, lemmas, proofs, etc.
\newtheorem{theorem}{Theorem}
\newtheorem{lemma}[theorem]{Lemma}
\newtheorem{proposition}[theorem]{Proposition}
\newtheorem{claim}[theorem]{Claim}
\newtheorem{corollary}[theorem]{Corollary}
\newtheorem{definition}[theorem]{Definition}
\newtheorem{problem}{Problem}

\DeclarePairedDelimiter\ceil{\lceil}{\rceil}
\DeclarePairedDelimiter\floor{\lfloor}{\rfloor}
\DeclarePairedDelimiter\anglebrac{\langle}{\rangle}

\begin{document}

\lhead{CSC 2221}
\chead{Xinyuan Li: 1000858244}
\rhead{\today}

\section*{Assignment $7^*$}
% ===> START ASSIGNMENT
\paragraph{Problem Statement}
Give an algorithm \textcolor{red}{using only multi-writer registers} for solving $\epsilon$-approximate agreement with any $0 < \epsilon < 1$ for inputs in $[0,1]$ that has $O\left(\log_2\left(\frac{1}{\epsilon}\right)\right)$ step complexity. Prove your algorithm is correct. \textcolor{red}{This is another submission of assignment 7 in light of the above changes. If it is possible to use atomic snapshot objects please refer to the previous submission.}

\paragraph{Algorithm.}
Let the processes be $p_1, ..., p_n$ and let the input to process $p_i$ be $x_i \in [0,1]$. Our algorithm is based off of the one shown in class for the case where $\epsilon = \frac{1}{2}$. We make use of $r = \ceil{\log 1/\epsilon} + 1$ sets of multi-reader multi-writer registers $W^1, ..., W^r$ where each $W^k$ contains $2^k$ registers $W^k[0], ..., W^k[2^k-1]$. All registers are initialized to $0$.  

The high-level description of the algorithm is as follows: the processes will interact with the sets of registers in order. In round $k$ the unit interval is divided into $2^k$ sub-intervals $I_0 = [0,1/2^k]$, $I_1 = [1/2^k, 2/2^k]$, ..., $I_{2^k-1} = [(2^k-1)/2^k, 1]$ with each $W^k[j]$ associated with $I_j$. When we read or write from $I_j$ we are actually reading and writing to $W^k[j]$. 

If $x_i = \frac{j}{2^k}$ for $j = 0, ..., 2^k-1$, process $p_i$ sets the intervals to the left and right of $x_i$ to $1$. Then $p_i$ reads the interval two spaces to the left of $x_i$. If $p_i$ gets a $1$ then $p_i$ updates $x_i$ to $\frac{j-1}{2^{k}}$. Otherwise, $p_i$ reads the interval two spaces to the right of $x_i$. If this interval is $1$ then $p_i$ updates $x_i$ to $\frac{j+1}{2^{k}}$. If this value is $0$ then $x_i$ remains unchanged. If $x_i$ is only contained in one interval $I_j = [j/2^k, (j+1)/2^k]$ then $p_i$ will write $1$ to $I_j$. Next $p_i$ reads the value of the interval to the left of $I_c$. If this value is $1$ then $p_i$ updates $x_i$ to be $j/2^k$. Otherwise $p_i$ reads the value of the interval to the right of $I_c$. If its value is $1$ then $p_i$ updates $x_i$ to $(j+1)/2^k$. If not then $x_i$ remains unchanged. (\emph{In the above, we only read and write if the interval exists.}) 

\begin{claim}
The above algorithm solves $\epsilon$-agreement for any $\epsilon \in (0,1)$ and inputs in $[0,1]$. Further the step complexity for each process is $O(\log 1/\epsilon)$.
\end{claim}
\begin{proof}
It is easy to see that the algorithm satisfies the termination condition since each process iterates for $r$ rounds and each round consists of constantly many read and write operations. By the same reasoning, each process takes $O(\log 1/\epsilon)$ steps. 

We show that the algorithm satisfies the validity condition by induction on the number of rounds. Initially $x_i$ are the inputs so validity is satisfied. Consider process $p_i$ in round $k > 1$. By the induction hypothesis there exists $x$ and $x'$ such that $x \leq x_i \leq x'$ for $x_i$ in round $k-1$. If $x_i$ did not change in round $k$, then validity holds. Otherwise, w.l.o.g suppose $x_i$ changed to $\frac{c}{2^k}$ such that $x_i$ was smaller than $\frac{c}{2^k}$. Then there must be some $p_j$ which set $I_c$ to $1$. By the induction hypothesis there exists input $x''$ with $\frac{c}{2^k} \leq x''$. Then $x \leq x_i \leq x''$ in the $k^{th}$ round. 

We prove $\epsilon$-agreement by induction on the number of rounds. Namely, at the end of round $k$, all $x_i$ will be in an interval of length $1/2^k$. This is true initially since the inputs are in $[0,1]$. Consider $x_i$ and $x_j$ of process $p_i$ and $p_j$ in at the end of round $k$. By the induction hypothesis, at the beginning of round $k$, $x_i$ and $x_j$ are in some interval $I_{c} = [c/2^{k-1}, (c+1)/2^{k-1}]$ of length $1/2^{k-1}$. In round $k$, $I_c$ is divided into $I_{2c} = [2c/2^k, (2c+1)/2^k]$ and $I_{2c+1} = [(2c+1)/2^k, (2c+2)/2^k]$. If $x_i$ and $x_j$ are in the same interval then we are done. Thus  W.l.o.g suppose $x_i \in I_{2c}$ and $x_j \in I_{2c+1}$ and $p_j$ wrote $1$ to its intervals first. $p_j$ will its updated value of $x_j$ to some value in $I_{2c+1}$. $p_i$ will read the value written by $p_j$ and update $x_i$ to $\frac{2c+1}{2^k}$. Thus at the end of round $k$, all $x_i$ are in the same interval of length $\frac{1}{2^k}$. After $r$ rounds all outputs will be within $\epsilon$ of one another.
\end{proof}
% ===> END ASSIGNMENT
\end{document}