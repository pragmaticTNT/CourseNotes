%
% This is a borrowed LaTeX template file for lecture notes for CS267,
% Applications of Parallel Computing, UCBerkeley EECS Department.
\documentclass[twoside]{article}
\setlength{\oddsidemargin}{0.25 in}
\setlength{\evensidemargin}{-0.25 in}
\setlength{\topmargin}{-0.6 in}
\setlength{\textwidth}{6.5 in}
\setlength{\textheight}{8.5 in}
\setlength{\headsep}{0.75 in}
\setlength{\parindent}{0 in}
\setlength{\parskip}{0.1 in}

%
% ADD PACKAGES here:
%

\usepackage{amsmath,amsfonts,amssymb, graphicx}
\usepackage{algorithm}
\usepackage{algpseudocode}
\usepackage{color}
\usepackage{comment}
\usepackage{mathtools}

%
% The following commands set up the lecnum (lecture number)
% counter and make various numbering schemes work relative
% to the lecture number.
%
\newcounter{lecnum}
\renewcommand{\thepage}{\thelecnum-\arabic{page}}
\renewcommand{\thesection}{\thelecnum.\arabic{section}}
\renewcommand{\theequation}{\thelecnum.\arabic{equation}}
\renewcommand{\thefigure}{\thelecnum.\arabic{figure}}
\renewcommand{\thetable}{\thelecnum.\arabic{table}}

%
% The following macro is used to generate the header.
%
\newcommand{\lecture}[4]{
   \pagestyle{myheadings}
   \thispagestyle{plain}
   \newpage
   \setcounter{lecnum}{#1}
   \setcounter{page}{1}
   \noindent
   \begin{center}
   \framebox{
      \vbox{\vspace{2mm}
    \hbox to 6.28in { {\bf CSC2221: Introduction to Distributed Computing
	\hfill Fall 2017} }
       \vspace{4mm}
       \hbox to 6.28in { {\Large \hfill Lecture #1: #2  \hfill} }
       \vspace{2mm}
       \hbox to 6.28in { {\it Lecturer: #3 \hfill Scribe: #4} }
      \vspace{2mm}}
   }
   \end{center}
   \markboth{Lecture #1: #2}{Lecture #1: #2}
   \vspace*{4mm}
}
\renewcommand{\cite}[1]{[#1]}
\def\beginrefs{\begin{list}%
        {[\arabic{equation}]}{\usecounter{equation}
         \setlength{\leftmargin}{2.0truecm}\setlength{\labelsep}{0.4truecm}%
         \setlength{\labelwidth}{1.6truecm}}}
\def\endrefs{\end{list}}
\def\bibentry#1{\item[\hbox{[#1]}]}

\newcommand{\fig}[3]{
			\vspace{#2}
			\begin{center}
			Figure \thelecnum.#1:~#3
			\end{center}
	}
% Use these for theorems, lemmas, proofs, etc.
\newtheorem{theorem}{Theorem}[lecnum]
\newtheorem{lemma}[theorem]{Lemma}
\newtheorem{proposition}[theorem]{Proposition}
\newtheorem{claim}[theorem]{Claim}
\newtheorem{corollary}[theorem]{Corollary}
\newtheorem{definition}[theorem]{Definition}
\newtheorem{example}[theorem]{Example}
\newenvironment{proof}{{\bf Proof:}}{\hfill\rule{2mm}{2mm}}

% **** IF YOU WANT TO DEFINE ADDITIONAL MACROS FOR YOURSELF, PUT THEM HERE:

\newcommand\readF{\mathsf{read}}
\newcommand\writeF{\mathsf{write}}
\newcommand\timestamp{\mathsf{TimeStamp}}
\newcommand\GetTS{\mathsf{GetTS}}
\newcommand\N{\mathbb{N}}
\newcommand\updateF{\mathsf{update}}
\newcommand\scanF{\mathsf{scan}}

\DeclarePairedDelimiter\ceil{\lceil}{\rceil}
\DeclarePairedDelimiter\floor{\lfloor}{\rfloor}
\DeclarePairedDelimiter\anglebrac{\langle}{\rangle}

\begin{document}
\lecture{6}{Timestamps}{Faith Ellen}{Lily Li}

\section{Lock-Freedom, Obstruction-Freedom}
\begin{definition}
An implementation is \textbf{wait-free} if in every admissible execution, \emph{every} non-faulty process completes its operations on the implemented object within a finite number of its own steps.

A \textbf{non-blocking/ lock-free} implementation is one were every configuration has some process that finishes its operation within a finite number of its own steps. Consider the following example.

\begin{algorithm}
	\caption{Implementation of Fetch\&Increment using Weak \textproc{CAS}}
    \label{pseudocode:FNI}
    \begin{algorithmic}[1]
	\State Repeat
	\State $v \leftarrow \mathsf{read}(C)$
	\State until $\mathsf{CAS}(C, v, v+1) =$ true
    \State return $v$
    \end{algorithmic}
\end{algorithm}

\textbf{Obstruction-free/ solo-terminating}: from every reachable configuration in which some process has a pending operation the process can complete its operation if it is given sufficiently many consecutive steps.
\end{definition}

\section{Timestamps (AW 220-222)}

$\timestamp$ objects are used to record info about various operations or even occurring in relation to one another. Suppose $T$ is a $\timestamp$ object. The function $\mathsf{GetTS}(T)$ returns the values from a partially ordered set $U$ such that if $\GetTS(T)$ operation $g_2$ that returns $t_2$ is invoked after $\GetTS(T)$ operation $g_1$ that returns $t_1$. Then $t_1 < t_2$. If $g_1$ and $g_2$ are concurrent then any order of $t_1$ and $t_2$ is possible, even $t_1$ and $t_2$ are incomparable. Below is the implementation of a $\timestamp$ object: Algorithm \ref{pseudocode:timestamp}.
\begin{algorithm}
	\caption{Implementation of $\timestamp$ Object}
    \label{pseudocode:timestamp}
    \begin{algorithmic}[1]
	\State \# Using a Fetch and Increment
	\State $\GetTS(T)$
	\State returns $\mathsf{Fetch\&Increment}(T)$
	\State
	\State \# Using a Counter: $U = \N - \{0\}$
	\State $\GetTS(C)$
	\State Increment $C$
	\State return read $C$
	\State
	\State \# Using a Fetch and Increment again (for processor $i$ when there are a total of $n$ processors) 
	\State $\GetTS(T)$
	\State Increment $C$
	\State return $n \cdot \readF(C) + i$
	\State
	\State \# Use registers for process $p_i$
	\State $\GetTS(R)$
	\State $y = 0$
	\For{$j$ from $0$ to $n-1$}
		\State $y \leftarrow \max{y, \readF(x_j)}$
	\EndFor
	\State $y \leftarrow y + 1$
	\State $x_i \leftarrow y$
	\State return $y$
	\State
	\State \# Vector timestamps $U = \N^n$
	\State $\GetTS$	
	\For{$j$ from $0$ to $n-1$}
		\State $v_j \leftarrow \readF(x_j)$ \# this is called a \textbf{collect}
	\EndFor
	\State $v_i \leftarrow v_i + 1$
	\State $x_i \leftarrow v_i$
	\State return $v$
	\State \# Here we would use lexicographical or component-wise for $v$
	\State
	\State \# Bounded timestamps: process only has timestamp from its last $\GetTS$ operation
	\State $n = 2$
	\State $U = \{0, 1, 2\}$ and $0 < 1$, $1 < 2$, $2 < 0$
	\State $\GetTS$ by process $p_i$
	\State $x \leftarrow \readF(T_{1-i})$
	\State $T_i \leftarrow \writeF(x + 1 \mod 3)$
    \end{algorithmic}
\end{algorithm}

\section{Atomic Snapshots (AW 10.3)}
This object has $m$-components $S_0, ..., S_{m-1}$. It supports two operations $\updateF(S, i, w)$ which sets $S_i$ to have value $w$ and $\scanF(S)$ which returns $(S_0, ..., S_{m-1})$. Observe that when $m = 1$ then we simply have a register. Let us consider how to implement these two operations so that they are linearizable. 

\begin{algorithm}
	\caption{Implementation of $\timestamp$ Object}
    \label{pseudocode:timestamp}
    \begin{algorithmic}[1]
	\State Scan		
	\end{algorithmic}
\end{algorithm}

Notice that the sequence numbers are really problematic since they are unbounded. Here is what we can do to improve the situation: the scanner is going to have to write (though in a really limited way). The problem is call the ABA problem (the register changed from $A$ to $B$ then back to $A$ and we want a way to distinguish between the first and last $A$). What we need to do is implement a register that avoids the ABA problem without using sequence numbers and timestamps. For single writer/reader register $R$ with reader $p_r$ and writer $p_w$.  We make use of two additional single bit registers $H_w$ (single writer by $p_w$) and $H_r$ (single writer by $p_r$). The new write will be:

\begin{algorithm}
	\caption{Implementation of $\timestamp$ Object}
    \label{pseudocode:register}
    \begin{algorithmic}[1]
	\State $WRITE(R, v)$ by $p_w$
	\State $\writeF(R, v, process ID of writer)$
	\State $h \leftarrow \readF(H_r)$
	\State $\writeF(H_w, 1-h)$
	\State
	\State $READTWICE(R)$ by $p_r$
	\State $h \leftarrow READ(H_w)$
	\State $\writeF(H_r, h)$
	\State $r \leftarrow \readF(R)$
	\State $r' \leftarrow \readF(R)$
	\State $h' \leftarrow \readF(H_W)$
	\If{$r = r'$ and $h = h'$}
		\State return $True$
	\Else
		\State return $False$
	\EndIf		
	\end{algorithmic}
\end{algorithm}

\begin{claim}
If $READ-TWICE$ return $F$ then either the value of $R$ changed between the two reads of $R$ or the value of $H_w$ changed between the reads of $H_w$.
\end{claim}

\begin{claim}
If $READ-TWICE(R)$ returns $True$ then the value of $T$ did not change between the two reads of $R$.
\end{claim}
\begin{proof}
Suppose that $READ-TWICE(R)$ returns $True$ and $r = r'$ and $h = h'$. If you think about it a little bit you see that it is impossible to have some writes between the two reads. Every time a write occurs the writer will force $H_w$ to differ from $H_r$. Since only $w$ can write to $H_w$ there is no way that $H_w = H_r$ until $r$ successfully finished a write. Thus the reader can detect any writes that occurred.  
\end{proof}

The way to extend this construction to $n$ processes who behave as readers and writers is to have such a pair of bits for every pair of process (actually you want two pair for process pair since both might want to be readers and writers).

\subsection{Handshaking (AW 10.3.1)}
Actually we already saw an implementation of handshaking from the above involving registers. This idea can be further implemented to Atomic-Snapshot objects to remove the need of the sequence numbers.  

\end{document}