\documentclass[11pt]{article}

% ===> PACKAGES
\usepackage{amsmath,amsthm,amssymb}
\usepackage{algorithm}
\usepackage{algpseudocode}
\usepackage{color}
\usepackage{comment}
\usepackage{fancyhdr}
\usepackage{mathtools}
\usepackage[margin=1in]{geometry} 
\usepackage{thmtools}

% ===> PARAMETERS
\pagestyle{fancy}

% ===> MACROS
\setlength{\parindent}{0em}
\setlength{\parskip}{0.5em}
\algdef{SE}[DOWHILE]{Do}{doWhile}{\algorithmicdo}[1]{\algorithmicwhile\ #1}%

%	\def\macrosName{Fill in the content of the macros and use \textit{\\macrosName} whenever necessary}
\newcommand\writeF{\textsc{write}}
\newcommand\updateF{\textsc{update}}
\newcommand\scanF{\textsc{scan}}
\newcommand\COLLECT{\textsc{COLLECT}}
\newcommand\readF{\textsc{read}}

% Use these for theorems, lemmas, proofs, etc.
\newtheorem{theorem}{Theorem}
\newtheorem{lemma}[theorem]{Lemma}
\newtheorem{proposition}[theorem]{Proposition}
\newtheorem{claim}[theorem]{Claim}
\newtheorem{corollary}[theorem]{Corollary}
\newtheorem{definition}[theorem]{Definition}
\newtheorem{problem}{Problem}

\DeclarePairedDelimiter\ceil{\lceil}{\rceil}
\DeclarePairedDelimiter\floor{\lfloor}{\rfloor}
\DeclarePairedDelimiter\anglebrac{\langle}{\rangle}

\begin{document}

\lhead{CSC 2221}
\chead{Xinyuan Li: 1000858244}
\rhead{\today}

\section*{Assignment 6}
% ===> START ASSIGNMENT
\begin{claim}
Algorithm \ref{pseudocode:multiwritersnapshot} is a linearizable, obstruction-free implementation of a multi-writer snapshot object. Here $R$ is a multi-writer register and $S$ is an array of $m$ multi-writer registers. The multi-writer snapshot object described in Algorithm \ref{pseudocode:multiwritersnapshot} is not non-blocking.

\begin{algorithm}
	\caption{Operations for the multi-writer snapshot object.}
    \label{pseudocode:multiwritersnapshot}
    \begin{algorithmic}[1]
	\State $\updateF(j, v)$ by process $p_i$:
	\State $R \leftarrow \writeF(i)$
	\State $S[j] \leftarrow \writeF(v)$
	\State return
	\State
	\State $\scanF()$ by processor $p_i$:
	\Do 
		\State $R \leftarrow \writeF(i)$
		\State $c \leftarrow \COLLECT(S)$
	\doWhile{$\readF(R) \neq i$}
	\State return $c$
    \end{algorithmic}
\end{algorithm}
\end{claim}
\begin{proof}
Lemma \ref{lem:linearization} shows a linearization of the update and scan operations of Algorithm \ref{pseudocode:multiwritersnapshot}. Lemma \ref{lem:obsfree} shows that the multi-writer snapshot object is obstruction-free. Lemma \ref{lem:notnonblock} shows that the object is \emph{not} non-blocking by presenting a configuration in which no process can finish its operation in a finite number its own steps.
\end{proof}

\begin{lemma}
\label{lem:linearization}
The implementation of a multi-writer snapshot object in Algorithm \ref{pseudocode:multiwritersnapshot} is linearizable.
\end{lemma}
\begin{proof}
Let $s$ be the schedule of a set of operations on the multi-writer snapshot object. Since $R$ is a multi-writer register, the sequence of writes to $R$ is linearizable. Let $\pi_R$ be the linearization of the low-level write operations to $R$ in the execution of $s$. Our linearization $\pi$ of $s$ sets the linearization point of each update and scan operation at the point of their last write operation to $R$. Note that the linearization point of every operation occurs within its interval of invocation and response. 

We consider the following cases to show that $\pi$ preserves the order of $\updateF$ and $\scanF$ operations. Let $\sigma_1$ and $\sigma_2$ be two operations such that $\sigma_1$ ended before $\sigma_2$ began and let $\pi(\sigma_1)$ and $\pi(\sigma_2)$ be their respective linearization points under $\pi$.
\begin{enumerate}
\item Let $\sigma_1$ and $\sigma_2$ both be $\updateF$ operations. Suppose for a contradiction that $\pi(\sigma_1) > \pi(\sigma_2)$. The write-to-$R$ of $\sigma_1$ must have taken place after the write-to-$R$ of $\sigma_2$. This is a contradiction since $\sigma_1$ terminated before $\sigma_2$ (and by extension the write-to-$R$ of $\sigma_2$) began. 
\item Suppose $\sigma_1$ and $\sigma_2$ are both $\scanF$ operations. If $\pi(\sigma_1) > \pi(\sigma_2)$ then the last write of $\sigma_2$ occurs before the last write-to-$R$ of $\sigma_1$. This is impossible as $\sigma_1$ ended before the start of $\sigma_2$.
\item Suppose $\sigma_1$ is an $\updateF$ operation while $\sigma_2$ is a $\scanF$ operation. Suppose for a contradiction that $\pi(\sigma_1) > \pi(\sigma_2)$. Again, this order of the linearization points implies a non-empty intersection between the execution intervals of $\sigma_1$ and $\sigma_2$. This is a contradiction since $\sigma_1$ terminated before the start of $\sigma_2$. 
\item If $\sigma_1$ is as $\scanF$ operation while $\sigma_2$ is an $\updateF$ operation we still cannot have $\pi(\sigma_1) > \pi(\sigma_2)$ by the same reasoning as the above.  
\end{enumerate}
\end{proof}

\begin{lemma}
\label{lem:obsfree}
The implementation of a multi-writer snapshot object in Algorithm \ref{pseudocode:multiwritersnapshot} is obstruction-free.
\end{lemma}
\begin{proof}
It is easy to see that the $\updateF$ operation is obstruction-free since it only consists of two write operations to obstruction-free multi-writer registers. Next consider the solo-execution of some process $p_i$ during a $\scanF$ operation. $p_i$ writes $i$ to $R$ then performs a $\COLLECT$ operation on $S$ and stores the result in $c$. Then $p_i$ reads register $R$. Since no other processor has written to $R$, the value in $R$ is still $i$. Thus $p_i$ exists the while-loop and terminates after returning $c$. 
\end{proof}

\begin{lemma}
\label{lem:notnonblock}
The implementation of a multi-writer snapshot object in Algorithm \ref{pseudocode:multiwritersnapshot} is not non-blocking.
\end{lemma}
\begin{proof}
Let $p_1$ and $p_2$ be two processes sharing such a multi-writer snapshot object. The initial value in $R$ is $0$ and the initial value in $S$ is $\anglebrac{0,0}$. Suppose both $p_1$ and $p_2$ initiate a $\scanF$ operation and consider the following sequence of low-level $\readF$ and $\writeF$ operations where $op_i$ is an operation performed by process $p_i$ (ignore the $\COLLECT$ operations since they do not interact with $R$).
\[\writeF_1(1), \writeF_2(2), \readF_1().\] 
Since $p_2$ was the most recent process to write to $R$, $\readF_1()$ will return $2$ and $p_1$ will perform another iteration of the while-loop. Then, if we have
\[\writeF_1(1), \readF_2(),\]
$\readF_2()$ will return $1$ since $p_1$ was the most recent process to write to $R$ so $p_2$ will also perform another iteration of the while-loop. The sequence can be extended indefinitely by repeating
\[\writeF_2(2), \readF_1(), \writeF_1(1), \readF_2().\] 
Neither process can finish within a finite number of its own steps so the multi-writer snapshot object described in Algorithm \ref{pseudocode:multiwritersnapshot} is not non-blocking.
\end{proof}
% ===> END ASSIGNMENT
\end{document}