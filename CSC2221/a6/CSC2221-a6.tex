\documentclass[11pt]{article}

% ===> PACKAGES
\usepackage{amsmath,amsthm,amssymb}
\usepackage{algorithm}
\usepackage{algpseudocode}
\usepackage{color}
\usepackage{comment}
\usepackage{fancyhdr}
\usepackage{mathtools}
\usepackage[margin=1in]{geometry} 
\usepackage{thmtools}

% ===> PARAMETERS
\pagestyle{fancy}

% ===> MACROS
\setlength{\parindent}{0em}
\setlength{\parskip}{0.5em}
\algdef{SE}[DOWHILE]{Do}{doWhile}{\algorithmicdo}[1]{\algorithmicwhile\ #1}%

%	\def\macrosName{Fill in the content of the macros and use \textit{\\macrosName} whenever necessary}
\newcommand\writeF{\textsc{write}}
\newcommand\updateF{\textsc{update}}
\newcommand\scanF{\textsc{scan}}
\newcommand\COLLECT{\textsc{COLLECT}}
\newcommand\READ{\textsc{READ}}

% Use these for theorems, lemmas, proofs, etc.
\newtheorem{theorem}{Theorem}
\newtheorem{lemma}[theorem]{Lemma}
\newtheorem{proposition}[theorem]{Proposition}
\newtheorem{claim}[theorem]{Claim}
\newtheorem{corollary}[theorem]{Corollary}
\newtheorem{definition}[theorem]{Definition}
\newtheorem{problem}{Problem}

\DeclarePairedDelimiter\ceil{\lceil}{\rceil}
\DeclarePairedDelimiter\floor{\lfloor}{\rfloor}
\DeclarePairedDelimiter\anglebrac{\langle}{\rangle}

\begin{document}

\lhead{CSC 2221}
\chead{Xinyuan Li: 1000858244}
\rhead{\today}

\section*{Assignment 6}
% ===> START ASSIGNMENT
\begin{claim}
The following (Algorithm \ref{pseudocode:multiwritersnapshot}) is a linearizable, obstruction-free implementation of a multi-writer snapshot object. Here $R$ is a multi-writer register and $S$ is an array of $m$ multi-writer registers. Explain why the object is not non-blocking.

\begin{algorithm}
	\caption{Operations for the multi-writer snapshot object.}
    \label{pseudocode:multiwritersnapshot}
    \begin{algorithmic}[1]
	\State $\updateF(j, v)$ by process $p_i$:
	\State $R \leftarrow \writeF(i)$
	\State $S[j] \leftarrow \writeF(v)$
	\State return
	\State
	\State $\scanF$ by processor $p_i$:
	\Do 
		\State $R \leftarrow \writeF(i)$
		\State $c \leftarrow \COLLECT(S)$
	\doWhile{$\READ(R) = i$}
	\State return $c$
    \end{algorithmic}
\end{algorithm}
\end{claim}
\begin{proof}

\end{proof}
% ===> END ASSIGNMENT
\end{document}