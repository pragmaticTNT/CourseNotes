\documentclass[11pt]{article}

% ===> PACKAGES
\usepackage{amsmath,amsthm,amssymb}
\usepackage{algorithm}
\usepackage{algpseudocode}
\usepackage{color}
\usepackage{comment}
\usepackage{fancyhdr}
\usepackage{mathtools}
\usepackage[margin=1in]{geometry} 
\usepackage{thmtools}

% ===> PARAMETERS
\pagestyle{fancy}

% ===> MACROS
\setlength{\parindent}{0em}
\setlength{\parskip}{0.5em}

%	\def\macrosName{Fill in the content of the macros and use \textit{\\macrosName} whenever necessary}
\newcommand\A{\mathcal{A}}
\newcommand\comp{\mathsf{comp}}
\newcommand\crash{\mathsf{crash}}
\newcommand\inbuf{\mathsf{inbuf}}
\newcommand\testNsetF{\textsc{Test\&Set}}
\newcommand\twoconsensusF{\textsc{T\&S-Two-Concensus}}
\newcommand\writeF{\textsc{write}}
\newcommand\readF{\textsc{read}}
\newcommand\tns{\textproc{test\&set }}
\newcommand\resetF{\textproc{reset }}

% Use these for theorems, lemmas, proofs, etc.
\newtheorem{theorem}{Theorem}
\newtheorem{lemma}[theorem]{Lemma}
\newtheorem{proposition}[theorem]{Proposition}
\newtheorem{claim}[theorem]{Claim}
\newtheorem{corollary}[theorem]{Corollary}
\newtheorem{definition}[theorem]{Definition}
\newtheorem{problem}{Problem}

\DeclarePairedDelimiter\ceil{\lceil}{\rceil}
\DeclarePairedDelimiter\floor{\lfloor}{\rfloor}
\DeclarePairedDelimiter\anglebrac{\langle}{\rangle}

\begin{document}

\lhead{CSC 2221}
\chead{Xinyuan Li: 1000858244}
\rhead{\today}

\section*{Assignment 4}
% ===> START ASSIGNMENT
\begin{theorem}
It is impossible to solve two processor binary consensus using only one Test\&Set object.
\end{theorem}
\begin{proof}
We will prove the theorem using a valency argument. Let $p_0$ and $p_1$ be the two processors. As shown during the first lecture, there exists an initial multi-valent configuration. Suppose that $C$ is a critical configuration. Then, without loss of generality (WLOG), $p_0$ is the processor whose step $\alpha_0$ takes $C$ to a $0$-valent configuration, $C_0 = C\alpha_0$, and $p_1$ is the processor whose step $\alpha_1$ takes $C$ to a $1$-valent configuration, $C_1 = C\alpha_1$. If $\alpha_0$ and $\alpha_1$ are both \resetF then $C_0\alpha_1$ and $C_1\alpha_0$ are indistinguishable to both processors. Thus executions starting from $C_0\alpha_1$ and $C_1\alpha_0$ should have the same valency, but this is a contradiction.

Next suppose WLOG that $\alpha_0$ is \resetF and $\alpha_1$ is \textproc{test\&set}. Observe that $C_1\alpha_0 \stackrel{p_0}{\sim} C_0$ since $p_0$ is in the same state and the test\&set object has the same value for both configurations. Thus $p_0$ should output the same value starting in both configurations. This is again a contradiction.

Finally suppose both $\alpha_0$ and $\alpha_1$ are both \textproc{test\&set}. WLOG suppose $p_0$ received $0$ and $p_1$ received $1$ from the \tns call. Then observe that $C_0\alpha_1 \stackrel{p_1}{\sim} C_1$. In both configurations the Test\&Set object contains $1$ and $p_1$ received $1$ from the \tns call. As above, $p_1$ should output the same value starting in both configurations, but this is a contradiction.

Thus starting from the initial multi-valent configuration it is always possible to move to a bi-valent configuration. Any supposed algorithm which solves binary consensus using only one test\&set object will have an infinite execution and will contradict the termination condition. 
\end{proof}


\begin{theorem}
It is possible to solve two processor binary consensus using two Test$\&$Set objects.
\end{theorem}
\begin{proof}
We will present an algorithm to solve two processor binary consensus using two Test\&Set objects and prove its correctness. Let the processors be denoted $p_1$, $p_2$ and let the Test\&Set objects be denoted $t_1$, $t_2$. Further let the input to processor $p_i$ be $u_i$. 

The two Test\&Set registers will play different roles in the algorithm. $t_1$ will be the competition object while $t_2$ will hold the input value of the processors. First processors ``set" $t_2$ to its input value, then compete for $t_1$. A processor $p_i$ ``sets" $t_2$ by calling \textproc{Test\&Set} $(t_2)$ if and only if $u_i = 1$. Further, if $u_i = 1$, $p_i$ will continue to compete for $t_1$ if it was the first to set $t_2$. The winner of the competition for $t_1$ (received $0$ from $\testNsetF(t_1)$) will output its own value. The loser will either check the value stored in $t_2$ or output $0$ depending on whether its input was zero or one respectively.  See Algorithm \ref{pseudocode:BCTwoTestNSet} for the associated pseudo-code.

Next we prove the correctness of the algorithm. It is easy to see that the algorithm terminates for every processor $p_i$. To see that validity holds, consider inputs $(u_1, u_2) = (0,0)$ and $(u_1, u_2) = (1,1)$. If $u_1 = u_2 = 0$, then both processors will compete for $t_1$. WLOG suppose that $p_1$ received $0$ and $p_2$ received $1$. $p_1$ will output $0$. $p_2$ will $\testNsetF(t_2)$, obtain $0$ since $t_2$ was not set, and output $0$ as well. Both processors, independent of whether the other processor crashed or not, will output $0$. Similarly, suppose $u_1 = u_2 = 1$. Both processors will try to set $t_2$. WLOG suppose that $p_1$ received $0$ and $p_2$ received $1$. $p_2$ will output $1$ safe in the knowledge that the other processor must have input $1$ as well. $p_1$ will compete for $t_1$, receive a $0$ since it is the only processor competing, and output $1$. Again, both processors, independent of whether the other processor crashed or not, will output $1$. Thus the validity condition holds. 

We show agreement by consider the algorithm on input $(u_1, u_2) = (1,0)$ (the case where the two inputs are equal is considered above and the case where $(u_1, u_2) = (0,1)$ is identical by symmetry). $p_1$ sets $t_2$, receives $0$, then competes for $t_1$. $p_2$ competes for $t_1$ directly. Two cases are possible:
\begin{enumerate}
\item[$p_1$ gets $0$:] $p_1$ wins $t_1$ and outputs $0$. $p_2$ received $1$ from $t_1$ and outputs $\lnot 1 = 0$. In essence $p_2$ \emph{knows} that the other processor had input $0$ since it ``saw" a $0$ in $t_2$ prior to setting $t_2$. Both processors were able to output a value irregardless of whether the other processor crashed or not. 
\item[$p_1$ gets $1$:] $p_2$ wins $t_1$ and outputs $1$. $p_1$ received $1$ from $t_1$ and performs $\testNsetF(t_2)$. Since this case can only occur when $p_1$ successfully set $t_2$, $p_1$ will receive and output $1$. Both processors were able to output a value irregardless of whether the other processor crashed or not.   
\end{enumerate} 
By the above case analysis, we see that the algorithm satisfies the agreement condition.
\end{proof}

\begin{algorithm}
	\caption{Algorithm for two processor binary consensus using two Test$\&$Set objects: code for processor $p_i$.}
    \label{pseudocode:BCTwoTestNSet}
    \begin{algorithmic}[1]
    \If{$u_i = 0$}
		\State $c \leftarrow \testNsetF(t_1)$
		\If{$c = 0$}
			\State output $0$
		\Else
			\State $s \leftarrow \testNsetF(t_2)$
			\State output $s$
		\EndIf  
    \Else
    	\State $s \leftarrow \testNsetF(t_2)$
		\If{$s = 1$}
			\State output $1$
		\EndIf
		\State $c \leftarrow \testNsetF(t_1)$
		\State output $\lnot c$
    \EndIf
    \end{algorithmic}
\end{algorithm}

\begin{claim}
It is possible to achieve 1-resilient consensus for $3$-processes using Test$\&$Set objects and registers. 
\end{claim}
\begin{proof}
We will presents an algorithm which achieves $1$-resilient consensus for $3$-processors using only Test\&Set objects and registers and prove its correctness. 

Let the processors be $p_0, p_1, p_2$ and let $u_i$ be the input to processor $p_i$. In addition, processor $p_2$ has two registers $r_2$ and $c_2$ used to store the output and check values respectively. Since Test\&Set objects have consensus number $2$ (Attiya page 326), there exists an asynchronous wait-free $2$-consensus algorithm, \textproc{T\&S-Two-Concensus}, which only uses Test\&Set objects and read/write registers. We will use \textproc{T\&S-Two-Concensus} as a subroutine in our algorithm. The function \textproc{T\&S-Two-Concensus}$(p, u, p', u')$ takes two processors ($p$ and $p'$) along with their respective inputs ($u$ and $u'$) and outputs the consensus value $v \in \{u, u'\}$. The write function of the registers is \textproc{write}$(r, v)$ and it writes value $v$ to register $r$. The read function of the registers is \textproc{read}$(r)$ and it outputs the value stored in register $r$. 

The high-level overview of our algorithm is as follows: $p_0$ and $p_1$ use \textproc{T\&S-Two-Consensus} to agree on value $val$. Both processors will write $val$ to $r_2$ then write a $1$ to the check register $c_2$. Finally they output $val$ and terminate. See Algorithm \ref{pseudocode:TCTestNSetCase1} for the associated pseudo-code. $p_2$ will adopt the value from \textproc{T\&S-Two-Consensus} by continuously checking to see if $c_2 = 1$. If so, $p_2$ will read and output the value of $r_2$. See Algorithm \ref{pseudocode:TCTestNSetCase2} for the associated pseudo-code. 

Next we show the correctness of the algorithm. The termination condition clearly holds for $p_0$ and $p_1$. Since at most one processor can fail, one of $p_0$ and $p_1$ will successfully write $val$ to $r_2$ and set $c_2$. Thus $p_2$ will eventually see a $1$ in $c_2$ and will terminate as well. Since we assumed that \textproc{T\&S-Two-Concensus} is an asynchronous wait-free $2$-consensus algorithm, $p_0$ and $p_1$ must output the same value $val$ and $val \in \{u_0, u_1\}$. $val$ is then written to $r_2$ by $p_0$ or $p_1$ so $p_2$ must also output $val$. Thus the validity and agreement conditions hold as well. 

\end{proof}

\begin{algorithm}
	\caption{Algorithm for $1$-resilient consensus for $3$-processors using only Test$\&$Set objects and read/write registers: code for processors $p_0$ and $p_1$.}
    \label{pseudocode:TCTestNSetCase1}
    \begin{algorithmic}[1]
	\State $val \leftarrow \twoconsensusF(p_0, u_0, p_1, u_1)$
	\State $\writeF(r_2, val)$
	\State $\writeF(c_2, 1)$
	\State output $val$
    \end{algorithmic}
\end{algorithm}

\begin{algorithm}
	\caption{Algorithm for $1$-resilient consensus for $3$-processors using only Test$\&$Set objects and read/write registers: code for processor $p_2$.}
    \label{pseudocode:TCTestNSetCase2}
    \begin{algorithmic}[1]
	\State $c \leftarrow \readF(c_2)$
	\While{$c = 0$}
		\State $c \leftarrow \readF(c_2)$
	\EndWhile
	\State $val \leftarrow \readF(r_2)$
	\State output $val$
    \end{algorithmic}
\end{algorithm}

% ===> END ASSIGNMENT
\end{document}