\documentclass[11pt]{article}

% ===> PACKAGES
\usepackage{amsmath,amsthm,amssymb}
\usepackage{algorithm}
\usepackage{algpseudocode}
\usepackage{color}
\usepackage{comment}
\usepackage{fancyhdr}
\usepackage{mathtools}
\usepackage[margin=1in]{geometry} 
\usepackage{thmtools}

% ===> PARAMETERS
\pagestyle{fancy}

% ===> MACROS
\setlength{\parindent}{0em}
\setlength{\parskip}{0.5em}
\algdef{SE}[DOWHILE]{Do}{doWhile}{\algorithmicdo}[1]{\algorithmicwhile\ #1}%

%	\def\macrosName{Fill in the content of the macros and use \textit{\\macrosName} whenever necessary}
\newcommand\GetTS{\mathsf{GetTS}}
\newcommand\request{\mathsf{request}}
\newcommand\propose{\mathsf{propose}}
\newcommand\confirm{\mathsf{confirm}}

% Use these for theorems, lemmas, proofs, etc.
\newtheorem{theorem}{Theorem}
\newtheorem{lemma}[theorem]{Lemma}
\newtheorem{proposition}[theorem]{Proposition}
\newtheorem{claim}[theorem]{Claim}
\newtheorem{corollary}[theorem]{Corollary}
\newtheorem{definition}[theorem]{Definition}
\newtheorem{problem}{Problem}

\DeclarePairedDelimiter\ceil{\lceil}{\rceil}
\DeclarePairedDelimiter\floor{\lfloor}{\rfloor}
\DeclarePairedDelimiter\anglebrac{\langle}{\rangle}

\begin{document}

\lhead{CSC 2221}
\chead{Xinyuan Li: 1000858244}
\rhead{\today}

\section*{Assignment 8}
% ===> START ASSIGNMENT
\paragraph{Problem Statement}
Consider the problem of implementing a timestamp object in an asynchronous message passing system with at most one Byzantine process. Whenever a non-faulty process performs $\GetTS$, it must eventually get a timestamp which is larger than all the timestamps returned by $\GetTS$ operations performed by non-faulty processes that finished  before this $\GetTS$ began.

For what number of processes, $n$, is it possible to solve this problem? Justify your answer.

\begin{claim}
It is possible to implement a timestamp object when $n = 1$ or when $n > 3$.
\end{claim}
\begin{proof}
It is trivial to implement a timestamp object for one process. The process keeps a local counter and increments it for every call to $\GetTS$. Lemma \ref{lem:lowerbound} shows that it is impossible to implement the timestamp object when $n = 2$ and $n = 3$. 

In the following we will present an algorithm which implements a timestamp object in an asychronous message passing system with at most one Byzantine process when there are $\geq 4$ processes. Denote the processes by $p_1, ..., p_n$. The general idea of this algorithm will be similar to that of simulating shared registers using message passing systems (Attiya Section 10.4).

A high-level description of the algorithm is as follows: every process $p_i$ has a local variable $t_i$ which stores the largest timestamp $p_i$ has seen and a vector $S_i$ to store proposed timestamps. $t_i$ is initialized to zero and $S_i$ is empty at the start of a $\GetTS$ operation. When $p_i$ executes $\GetTS$, $p_i$ broadcasts message $\request$ to all other processes (including itself). When a process $p_j$ gets a $\request$ message from $p_i$, it sends $p_i$ message $\propose(t_j)$. Upon receiving $\propose(t)$, $p_i$ adds $t$ to $S_i$. When $|S_i| \geq n-1$ process $p_i$ sets $t_i \leftarrow \max \{t \in S_i\} + 1$ and broadcasts message $\confirm(t_i)$ to all other processes (again, including itself). Upon receiving message $\confirm(t_i)$ from $p_i$, $p_j$ updates $t_j \leftarrow \max(t_i,t_j)$ and sends back $\anglebrac{ACK}$. $p_i$ counts the number of $\anglebrac{ACK}$ messages that it has received. When this number hits $n-1$, $p_i$ empties $S_i$ and outputs $t_i$.  

It remains to show that a $\GetTS$ operation $g_i$ which finished before the start of another $\GetTS$ operation $g_j$ began will return a smaller timestamps than the one returned by $g_j$. Suppose $p_i$ performed $g_i$ and received $t_i$ and then $p_j$ performed $g_j$ and received $t_j$ (it is possible that $p_i = p_j$ but then the condition is satisfied trivially). Before $g_i$ ended, $p_i$ sent $t_i$ to all processes and received $n-1$ $\anglebrac{ACK}$ messages. Thus all processes $p_k$ which successfully returned $\anglebrac{ACK}$ must have $t_k \geq t_i$. If $p_j$ sent $\anglebrac{ACK}$ to $p_i$ in $g_i$ then $t_j \geq \max\{t\in S_j\} + 1 \geq t_i + 1 > t_i$. Suppose not and consider the contents of $S_j$ just before $p_j$ updates $t_j$ during $g_j$. $|S_j| \geq n-1$ so $p_j$ must have received $n-2$ messages from other processes. Since $n \geq 4$, $p_j$ must have received a message from a non-faulty process $p_k$. $p_k$ sent $\anglebrac{ACK}$ to $p_i$ during $g_i$ so $t_k >
\geq t_i$. Thus $t_j = \max\{t\in S_j\} + 1 \geq t_k + 1 > t_i$.   
\end{proof}


\begin{lemma}
\label{lem:lowerbound}
When there are $2$ or $3$ processes and at most one Byzantine process, it is impossible to implement a timestamp object in an asynchronous message passing system.
\end{lemma}
\begin{proof}
It is easy to see that there cannot be an implementation of a time stamp object when $n = 2$. The adversary simply delays all messages sent between the two processes $p_0$ and $p_1$. From the perspective of $p_i$, it is unclear if $p_{1-i}$ is byzantine or just slow. Thus the value $p_i$ outputs from a $\GetTS$ operation is independent of what $p_{i-1}$ outputs.

Next consider the case when $n = 3$. Suppose for a contradiction that there exists an implementation of a timestamp object in the stated model. Let the processes be denoted $p_0$, $p_1$, and $p_2$. Let the initial configuration be $C$. The adversary will delay all messages passed between $p_0$ and $p_1$. 

Suppose $p_0$ performs a $\GetTS$ from $C$. Since $p_0$ cannot determine if $p_1$ is Byzantine (and is pretending to crash) or just slow, $p_0$ must decide upon some timestamp $s_0$ at time $t_0$. Let $\alpha_0$ be the execution where $p_0$, interacting only with $p_2$, gets $s_0$ at $t_0$. Next suppose that $p_1$ performs a $\GetTS$ operation after $t_0$. Since messages passed between $p_0$ and $p_1$ are still delayed, $p_1$ also cannot determine if $p_0$ is Byzantine or just slow so $p_1$ must decide upon some $s_1$ at time $t_1 > t_0 + 1$. Let $\alpha_1$ be the execution where $p_1$, interacting only with $p_2$, gets $s_1$ at $t_1$. Since the $\GetTS$ operation by $p_0$ finished before the start of the $\GetTS$ operation by $p_1$, $s_0 < s_1$. 

Suppose instead that $p_1$ performs $\GetTS$ from $C$. $p_2$ will behave just as it did in $\alpha_1$. Since $(C\alpha_0)\alpha_1 \stackrel{p_1}{\sim} C\alpha_1$, $p_1$ must decide upon timestamp $s_1$ at time $t_1'$. Next $p_0$ performs $\GetTS$ at some time $t_0'$ after $t_1'$. $p_2$ will behave just as it did in $\alpha_0$. Again, since $C\alpha_0 \stackrel{p_0}{\sim} (C\alpha_1)\alpha_0$, $p_0$ must decide upon timestamp $s_0$. This is a contradiction since the $\GetTS$ operation by $p_1$ finished before the start of the $\GetTS$ operation by $p_0$.     
\end{proof}

% ===> END ASSIGNMENT
\end{document}