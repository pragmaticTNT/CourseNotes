%%%%%%%%%%%%%%%%%%%%%%%%%%%%%%%%%%%%%%%%%
% Article Notes
% LaTeX Template
% Version 1.0 (1/10/15)
%
% This template has been downloaded from:
% http://www.LaTeXTemplates.com
%
% Authors:
% Vel (vel@latextemplates.com)
% Christopher Eliot (christopher.eliot@hofstra.edu)
% Anthony Dardis (anthony.dardis@hofstra.edu)
%
% License:
% CC BY-NC-SA 3.0 (http://creativecommons.org/licenses/by-nc-sa/3.0/)
%
%%%%%%%%%%%%%%%%%%%%%%%%%%%%%%%%%%%%%%%%%

%----------------------------------------------------------------------------------------
%	PACKAGES AND OTHER DOCUMENT CONFIGURATIONS
%----------------------------------------------------------------------------------------

\documentclass[
10pt, % Default font size is 10pt, can alternatively be 11pt or 12pt
letterpaper, % Alternatively letterpaper for US letter
onecolumn, % Alternatively twocolumn
% Alternatively landscape
]{article}

\usepackage[english]{babel}
\usepackage{amsfonts, amsmath, amssymb, amsthm}
\usepackage{bbold}
\usepackage{color}
\usepackage[utf8]{inputenc}
\usepackage{geometry}
	\geometry{margin=1in}
\usepackage{graphicx}
\usepackage{epstopdf}
	\epstopdfsetup{update}
\usepackage{comment}
\usepackage{fullpage}
\usepackage{hyperref}
\usepackage{mathtools}
\usepackage{thmtools}
\usepackage{tikz}
\usetikzlibrary{arrows, automata, fit, positioning}

%% ===> Page Settings <===
\setlength{\parindent}{0em}
\setlength{\parskip}{0.5em}
%% ===> End Page Settings <===

%% ===> Theorem Type Definitions <===
\newtheorem{theorem}{Theorem}
\newtheorem{lemma}[theorem]{Lemma}
\newtheorem{claim}[theorem]{Claim}
\newtheorem{corollary}[theorem]{Corollary}
\newtheorem{definition}[theorem]{Definition}
\newtheorem{example}[theorem]{Example}
\newtheorem{conjecture}[theorem]{Conjecture}
%% ===> END Theorem Type Definitions <===

%% ===> Deliminaters <===
\DeclarePairedDelimiter\ceil{\lceil}{\rceil}
\DeclarePairedDelimiter\floor{\lfloor}{\rfloor}
\DeclarePairedDelimiter\anglebrac{\langle}{\rangle}
\DeclarePairedDelimiter\norm{\lVert}{\rVert}
%% ===> END Deliminaters <===

%% ===> Math Operators <===
%% ---> General <---
\DeclareMathOperator*{\round}{rd}
\DeclareMathOperator*{\yes}{\mathsf{YES}}
\DeclareMathOperator*{\no}{\mathsf{NO}}
\newcommand{\ind}[1]{\mathbb{#1}}
%% ---> Logic <---
\DeclareMathOperator{\xor}{\oplus}
\DeclareMathOperator*{\false}{\mathsf{FALSE}}
\DeclareMathOperator*{\true}{\mathsf{TRUE}}
%% ---> Complexity <---
\newcommand\class[1]{\mathsf{#1}}
\DeclareMathOperator*{\poly}{poly}	% As in running-time NOT complexity class
\DeclareMathOperator*{\circuitsize}{\mathcal{C}}
\DeclareMathOperator*{\leafsize}{\mathcal{L}}
\DeclareMathOperator*{\depth}{depth}
%% ---> Probability <---
\DeclareMathOperator*{\expected}{\mathbb{E}}
\DeclareMathOperator*{\var}{Var}
\newcommand\distrib[1]{\mathcal{#1}}
\newcommand\mm[1]{\mathbf{#1}}	% Bolded capitals are MATRICES
\newcommand\vv[1]{\mathbf{#1}}	% Bolded lowercases are VECTORS
%% ===> End Math Operators <===

%% ===> Macros <===
\newcommand\NN{\mathbb{N}}
\newcommand\ZZ{\mathbb{Z}}
\newcommand\QQ{\mathbb{Q}}
\newcommand\RR{\mathbb{R}}
\newcommand\CC{\mathbb{C}}
%% ===> END Macros <===

% ===> Command Definitions <===
% ---> usage: \fig{NAME}{SCALE}{CAPTION}
\newcommand\fig[3]{
	\begin{figure}[ht]
		\centering
		\includegraphics[scale=#2]{figures/#1}
		\caption{#3}
		\label{fig:#1}
	\end{figure}
}	
% ---> usage: \makeheader{LECTURE-DATE}{LECTURE-NUM}{LECTURE-TITLE}{SCRIBE-NAME}
\newcommand\makeheader[4]{
	\fbox{\parbox{\textwidth}{CSC2429 / MAT1304: Circuit Complexity 
			\hfill #1%
			\begin{center}\LARGE 
				Lecture #2: #3
			\end{center}
			{\em Instructor: Benjamin Rossman
				\hfill
				Scribe: #4}
		}}	
}
%% ===> END Command Definitions <=== % Input the file specifying the document layout and structure

%----------------------------------------------------------------------------------------
%	ARTICLE INFORMATION
%----------------------------------------------------------------------------------------

\articletitle{How to Elect a Leader Faster than a Tournament} % The title of the article
\articlecitation{\cite{alistarh2015elect}} % The BibTeX citation key from your bibliography

\datenotesstarted{November 21, 2017} % The date when these notes were first made
\docdate{\datenotesstarted; rev. \today} % The date when the notes were lasted updated (automatically the current date)

\docauthor{Lily Li} % Your name

%----------------------------------------------------------------------------------------

\begin{document}

\pagestyle{myheadings} % Use custom headers
\markright{\doctitle} % Place the article information into the header

%----------------------------------------------------------------------------------------
%	PRINT ARTICLE INFORMATION
%----------------------------------------------------------------------------------------

\thispagestyle{plain} % Plain formatting on the first page

\printtitle % Print the title

%----------------------------------------------------------------------------------------
%	ARTICLE NOTES
%----------------------------------------------------------------------------------------

\section{Initial Board Configuration}
\paragraph{Problem Definition.} An algorithm which solves leader election among $n$ processes must satisfy: (1) termination - every non-faulty process $p_i$ must eventually output $win$ or $lose$ and (2) unique winner - only one process may output $win$. Further, no process may lose before the eventual winner starts its execution. 

\paragraph{Model.} A complete asynchronous message passing system with at most $f < \ceil{n/2}$ faulty processes. The randomized algorithm we will presents works against a strong(adaptive) adversary. 

%------------------------------------------------

\section{Introduction}
Today I will present ``How to Elect a Leader Faster than a Tournament'' by Alistarh et al. (\emph{Gelashvili, and Vladu}) which appeared in the proceeding of PODC 2015. 

\textcolor{blue}{\emph{Move to the left board with the prepared material.}}

The problem is define as follows:
\textcolor{blue}{\emph{Read material on ``Problem Definition''.}}

To ensure that operations are linearizable, the first operation is $win$ and all subsequent operations are $lose$, and the order of the non-overlapping operations is respected. Observe that this problem is equivalent to implementing a \emph{Test-and-Set} object. The \emph{winner} is analogous to the process which receives $0$ from the $Test\&Set$ call and the \emph{losers} are analogous to the processes which receives $1$ afterwards. Hopefully everyone remembers our discussion of $Test\&Set$ objects earlier in the semester.    

Our model is:
\textcolor{blue}{\emph{Read material on ``Model''.}}

To appreciate the results obtained in this paper, we first survey related work. \textcolor{red}{\emph{Change Slide.}} The best known previous algorithm, by Afek et al. in '92, makes use of a \emph{tournament tree}. In-class, we have a seen \emph{tournament trees} in the context of bounded mutual exclusion. The $n$ processes are the leaves of the tree. Processes compete at each internal nodes and the survivor proceeds up the tree. The \emph{winner} (resp. process which gets to enter the critical section) is the survivor at the root node. This algorithm has $O(\log n)$ step complexity and uses $O(n\log n)$ registers. In the context of a message passing system, we can use the ABD algorithm to simulate registers with $O(n)$ overhead, so the tournament tree has $O(n^2\log n)$ message complexity. 

\textcolor{red}{\emph{Change Slide.}} The randomized algorithm presented in this paper makes significant improvements to the expected step and message complexities. Each process is expected to take $O(\log^*n)$ step and expected to send $O(n^2)$ point-to-point messages. \textcolor{red}{\emph{Change Slide.}} The authors also proved a lower bound of $\Omega(n^2)$ required messages for this problem, so the algorithm also has optimal message complexity.

%------------------------------------------------

\section{Leader Election Algorithm}

\subsection{Intuition}
Before we look at the code for the algorithm, lets try to get a idea of what it wants to do at a high-level. \textcolor{red}{\emph{Change Slide.}} The algorithm works in phases. At each phase we must ensure that \emph{there is at least one survivor}. During a phase, a process $p_i$ flips a biased bit and decides whether to $survive$ (continue to the next phase) or $lose$ (drop out of contention). If $p_i$ gets $1$, then it survives. If $p_i$ gets $0$ then it must make sure that some other process will survive before $losing$. Thus if $p_i$ observes that some $p_j$ flipped a $1$  then it is safe in the knowledge that someone will $survive$ to the next phase and can $lose$. Otherwise, if $p_i$ only observes processes which flipped $0$, it is not sure if they will choose to $lose$ or $survive$. To ensure that some process survives to the next phase, $p_i$ must survive even though it flipped $0$.

Ideally we want many processes to drop out of contention at every phase, but a strong adversary can tamper with a naive implementation of the above idea. In particular, the adversary watch all the bit flips and schedule those processes which flipped $0$ first. Then all the processes must survive (those which flipped $0$ will only see other processes which flipped $0$ and will $survive$ and those which flipped $1$ will always survive). One of the main ideas in this algorithm, called the Poison Pill Technique, will ensure that this does not occur.

\subsection{Local Variables and Communication Primitives}
\textcolor{red}{\emph{Change Slide.}} \textcolor{blue}{\emph{Move to board bottom left.}} A process $p_i$ has two local variables: a vector $S_i$ of length $n$ which records the states of all observed processes and an $n\times n$ matrix $V_i$ storing the view as seen by other processes. States can be any of $w$ (waiting), $c$ (committed), $0$ (lose), and $1$ (survive). All entries of $S_i$ are initialized to $w$ and all entries of $V_i$ are empty. A process $p_j$ is \textbf{active} from the perspective of $p_i$ if column $j$ in $V_i$ contains at least one entry of status $1$ or $c$ and no entries of status $0$.  

\textcolor{red}{\emph{Change Slide.}} The following are the communication primitives we will use in the algorithm. $\communicate$ is a high-level protocol which takes a procedure $p$ which broadcasts a message and waits for a response. $\communicate\anglebrac{p}$ will execute $p$ and wait for at least $\ceil{n/2}+1$ responses before proceeding. \emph{This is reminiscent to what we did when we simulated single-reader single-writer registers in a message passing system}. Possible procedures $p$ include: \textcolor{blue}{\emph{read off $\broadcast(v)$ and $\collect()$ from the slide.}} \textcolor{red}{\emph{Change Slide.}} Further we define the method $\random$ as \textcolor{blue}{\emph{read off $\random(r)$ from the slide.}}
\begin{comment}
\begin{enumerate}
\item[$\broadcast(v)$:] process $p_i$ broadcasts value $v$ to all other processes (including itself). If process $p_j$ receives $\broadcast{v}$ from $p_i$, it updates $S_j[i] \leftarrow v$.
\item[$\collect()$:] process $p_i$ sends a collect message to all other processes (including itself). If process $p_j$ receives $\collect()$ from $p_i$, it sends message $\anglebrac{S_j}$ to $p_i$. 
\item[$\random(p)$:] (local operation) process $p_i$ receives $1$ with probability $p$ and $0$ otherwise. 
\end{enumerate}

What we want is an algorithm which iteratively reduces the number of participating processes. At each round processes flip a random coin and output $lose$ or $survive$ depending on the result. Surviving processes will try again in the next round. A na\"{i}ve implementation would fail since the adversary can schedule the ``losing'' processes first.
\end{comment}

We call a set of $\geq \ceil{n/2}+1$ processes a \textbf{quorum}. \textcolor{blue}{\emph{Add the definition of quorum to the board.}} They will arise frequently in our analysis.  

\subsection{Poison Pill (Simple)}
\textcolor{red}{\emph{Change Slide.}} Here is the algorithm for the poison pill algorithm (\emph{see Algorithm \ref{pseudocode:homPP}}). This code is executed by each process at every phase to determine whether it should $survive$ or $lose$ the current phase. \textcolor{blue}{\emph{Go through the code line by line.}} To analyze the step complexity of the algorithm we need to prove two claims. \textcolor{red}{\emph{Change Slide.}}

\begin{algorithm}[ht]
	\caption{Poison Pill: code for process $p_i$ during one phase.}
    \label{pseudocode:homPP}
    \begin{algorithmic}[1]
    \State Initialize $S_i[1...n] = [w, ... ,w]$ and $V_i[1...n][1...n] = \emptyset$
	\State $S_i[i] \leftarrow c$
	\State $\communicate\anglebrac{\broadcast(S_i[i])}$
	\State $S_i[i] \leftarrow \random\left(\frac{1}{\sqrt{n}}\right)$
	\State $\communicate\anglebrac{\broadcast(S_i[i])}$
	\State $V_i \leftarrow \communicate\anglebrac{\collect()}$
	\If{$S_i[i] = 0$ and $\exists k:$ $p_k$ is active}
		\State return $0$
	\EndIf
	\State return $1$ 
	\end{algorithmic}
\end{algorithm}

\begin{claim}
If all processes return then some process outputs $1$.
\end{claim}
\begin{proof}
\textcolor{red}{\emph{(c)}} Suppose for a contradiction that all processes output $0$. If a process receives $1$ from $\random\left(\frac{1}{\sqrt{n}}\right)$, then it must output $1$ so all processes received $0$. Let process $p_i$ be the last to complete the $\broadcast$ operation on Line 5. When $p_i$'s broadcast operation finishes, the $0$ of every process gets stored by a quorum. Consider $V_i$ at the completion of Line 6. Every column must have one $0$-entry since the quorum that stored $0$ must overlap with the quorum which successfully sent their status vector to $p_i$. Thus $p_i$ must output $1$. 
\end{proof}

\begin{lemma}
\label{lem:firstone}
If a process received $1$ from its execution of $\random$ at $t$. Then all process which received $0$ from $\random$ at any time $\geq t$ must output $0$. 
\end{lemma}
\begin{proof}
Suppose $p_i$ received $1$ from its execution of $\random$ at $t$ and $p_j$ received $0$ at any time $\geq t$. Observe the $\communicate$ of $S_i[i] = c$ on Line 3 has finished by $t$. Thus the quorum which received $S_i[i]$ and the quorum which sent their status to $p_j$ upon its execution of $\collect()$ on Line 6 overlaps. $p_j$ will see that $p_i$ is active and will return $0$. 
\end{proof}

\begin{claim}
\label{claim:expected}
The expected number of processes that return $1$ is $O(\sqrt{n})$.
\end{claim}
\begin{proof}
The expected number of $1$s is $\sqrt{n}$. All these processes will survive. The expected number of trails before the first $1$ is flipped is also $\sqrt{n}$. By Lemma \ref{lem:firstone}, at most $\sqrt{n}$ processes which flipped $0$ will survive.  
\end{proof}

By Claim \ref{claim:expected}, $O(\sqrt{n})$ processes are expected to survive each phase. Thus by a careful argument (which we can go through if there is time) there are $O(\log \log n)$ expected phases. 

%------------------------------------------------

\section{Improvements}
Observe that this is not the $O(\log^* n)$ step complexity that we claimed for the algorithm at the beginning. The issue is with our choice of $r$ (probability of flipping 1). In this simple version of the algorithm that we have presented, this value is fixed. It can be shown that if we must fix $r$ then $O(\sqrt{n})$ is the best expected number of survivors we can achieve. Instead what we need is the probability to change dynamically throughout a phase. Intuitively, we want the $r$ to be high at the beginning. When a process gets a 1 and broadcasts its value, all subsequent processes which flip a 0 can $lose$. So as more and more processes become active we want to decrease $p$ so more processes receive 0.

By implementing this idea in the complete form of Poison Pill, the expected number of processes to survive at each phase drops to $O(\log n)$.   

%------------------------------------------------

\section{Conclusion and Recap}
\textcolor{red}{\emph{(c)}} To recap what we have discussed:  \textcolor{blue}{\emph{Talk about the main ideas of the Poison Pill Technique.}} \textcolor{red}{\emph{(c)}} Using these ideas, the authors developed a randomized algorithm for the Leader Election problem (or alternatively, $Test\&Set$ in a message passing model) with expected $O(\log^* n)$ step complexity and $O(n^2)$ message complexity. \textcolor{red}{\emph{(c)}} We did not get an opportunity to talk about this today, but they also applied their algorithm to the renaming problem \textcolor{blue}{\emph{read results obtained for the renaming problem.}}  

%------------------------------------------------

%\section{Message Complexity Lower Bound?}


\begin{comment}
\subsection{Poison Pill (Heterogeneous)}
It can be shown that there will be $\Omega(\sqrt{n})$ survivors at every round of the homogeneous algorithm --- the probability of getting $1$ is fixed. To improve this bound we need to change this probability. 
\begin{algorithm}
	\caption{Heterogeneous Poison Pill: code for process $p_i$.}
    \label{pseudocode:hetPP}
    \begin{algorithmic}[1]
	\State $S_i[i] \leftarrow \{c, \{\}\}$
	\State $\communicate\anglebrac{broadcast(S_i[i])}$
	\State $V_i \leftarrow \communicate\anglebrac{\collect()}$
	\State $t \leftarrow \{j: \exists k: V_i[k][j] \neq \bot\}$
	\State $p \leftarrow \frac{\log |t|}{|t|}$ \# $p \leftarrow 1$ if $|t| = 1$
	\State $S_i[i] \leftarrow \left\{\random(p), t\right\}$
	\State $\communicate\anglebrac{\broadcast(S_i[i])}$
	\State $V_i \leftarrow \communicate\anglebrac{\collect()}$
	\If{$S_i[i][0] = 1$ and $\exists k:$ $p_k$ is active}
		\State return $0$
	\Else
		\State return $1$
	\EndIf
	\State return $1$
	\end{algorithmic}
\end{algorithm}

\subsection{Analysis}
Now we just need to show that the expected number of processes which survive using the heterogeneous poison pill algorithm is $\Theta(\log n)$. 

\begin{claim}
\label{claim:closure} Let $S$ be the set of processes which flipped $0$ and survived and let $U$ be the union of all the elements that they found not waiting. Then for $i \in U$, every $j \in t_i$ is in $U$.   
\end{claim}
\begin{proof}
Intuitively this should make sense as $U$ is pretty comprehensive (we are taking the union of a bunch of stuff). Since a process $p_i$ which flipped zero can only survive if is sees only other processes which lost, every $p \in U$ obtained $0$. There is some entry of the view $V_i$ corresponding to $p$ which has $0$. Further it also has the list associated with $p$. This list is taken into account during the union thus $p_i$ see all the elements in the list $t$ of $p$.
\end{proof}

\begin{claim}
\label{claim:reproc}
If processor $q$ completed the first broadcast call no later than $p$ completed its first propagate call, then $q$ will be included in the $l$ list of $p$.
\end{claim}

\begin{claim}
\label{claim:mainflip0}
The maximum expected number of processors that flip $0$ and survive is $O(\log n) + O(1)$.
\end{claim}
\begin{proof}
Let $S$ be the set of $z$ processes. And $U$ be defined as above. By Claim \ref{claim:reproc} and Claim \ref{claim:closure}, it must be the case that if $p \in U$ and $q$ finished the first broadcast no later than $p$, then $q \in U$. Notice that if we order the processes by the time they finish the first broadcast, all processes in $U$ must come before the processes not in $U$. 

Since $U$ forms a closed set and all processes in $U$ flipped $0$, the probability that each process flipped $0$ is at most $\left(1 - \frac{\log |U|}{|U|}\right)$. The probability for all processes in $U$ to flip $0$ is $\left(1 - \frac{\log |U|}{|U|}\right)^{|U|} = O\left(\frac{1}{|U|}\right)$. Since $z \in U$, This is $O\left(\frac{1}{z}\right)$.  
\end{proof}

\begin{claim}
\label{claim:mainflip1}
The maximum expected number of processors that flip $0$ and survive is $O(\log^2 n) + O(1)$.
\end{claim}
\begin{proof}
Order the processes according to the time they finish their first broadcast. By property \ref{claim:reproc}, the process ordered first has $|t| \geq 1$, the process ordered second has $|t| \geq 2$ and so on. Since the probability of flipping $1$ decreases as $|t|$ increases, the best expectation achievable is
\[1 + \sum_{l = 2}^{n} \frac{log|t|}{|t|} \in O(\log^2 n).\]
\end{proof}
\end{comment}
%----------------------------------------------------------------------------------------
%	BIBLIOGRAPHY
%----------------------------------------------------------------------------------------

\renewcommand{\refname}{Reference} % Change the default bibliography title

\bibliography{bibliography} % Input your bibliography file

%----------------------------------------------------------------------------------------

\end{document}