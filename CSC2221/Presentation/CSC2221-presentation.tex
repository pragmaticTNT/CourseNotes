%%%%%%%%%%%%%%%%%%%%%%%%%%%%%%%%%%%%%%%%%
% Article Notes
% LaTeX Template
% Version 1.0 (1/10/15)
%
% This template has been downloaded from:
% http://www.LaTeXTemplates.com
%
% Authors:
% Vel (vel@latextemplates.com)
% Christopher Eliot (christopher.eliot@hofstra.edu)
% Anthony Dardis (anthony.dardis@hofstra.edu)
%
% License:
% CC BY-NC-SA 3.0 (http://creativecommons.org/licenses/by-nc-sa/3.0/)
%
%%%%%%%%%%%%%%%%%%%%%%%%%%%%%%%%%%%%%%%%%

%----------------------------------------------------------------------------------------
%	PACKAGES AND OTHER DOCUMENT CONFIGURATIONS
%----------------------------------------------------------------------------------------

\documentclass[
10pt, % Default font size is 10pt, can alternatively be 11pt or 12pt
a4paper, % Alternatively letterpaper for US letter
onecolumn, % Alternatively twocolumn
% Alternatively landscape
]{article}

%
% This is a borrowed LaTeX template file for lecture notes for CS267,
% Applications of Parallel Computing, UCBerkeley EECS Department.
\documentclass[twoside]{article}
\setlength{\oddsidemargin}{0.25 in}
\setlength{\evensidemargin}{-0.25 in}
\setlength{\topmargin}{-0.6 in}
\setlength{\textwidth}{6.5 in}
\setlength{\textheight}{8.5 in}
\setlength{\headsep}{0.75 in}
\setlength{\parindent}{0 in}
\setlength{\parskip}{0.1 in}

%
% ADD PACKAGES here:
%

\usepackage{amsmath,amsfonts,amssymb, graphicx}
\usepackage{algorithm}
\usepackage{algpseudocode}
\usepackage{color}
\usepackage{comment}
\usepackage{mathtools}

%
% The following commands set up the lecnum (lecture number)
% counter and make various numbering schemes work relative
% to the lecture number.
%
\newcounter{lecnum}
\renewcommand{\thepage}{\thelecnum-\arabic{page}}
\renewcommand{\thesection}{\thelecnum.\arabic{section}}
\renewcommand{\theequation}{\thelecnum.\arabic{equation}}
\renewcommand{\thefigure}{\thelecnum.\arabic{figure}}
\renewcommand{\thetable}{\thelecnum.\arabic{table}}

%
% The following macro is used to generate the header.
%
\newcommand{\lecture}[4]{
   \pagestyle{myheadings}
   \thispagestyle{plain}
   \newpage
   \setcounter{lecnum}{#1}
   \setcounter{page}{1}
   \noindent
   \begin{center}
   \framebox{
      \vbox{\vspace{2mm}
    \hbox to 6.28in { {\bf CSC2429: Proof Complexity
	\hfill Winter 2018} }
       \vspace{4mm}
       \hbox to 6.28in { {\Large \hfill Lecture #1: #2  \hfill} }
       \vspace{2mm}
       \hbox to 6.28in { {\it Lecturer: #3 \hfill Scribe: #4} }
      \vspace{2mm}}
   }
   \end{center}
   \markboth{Lecture #1: #2}{Lecture #1: #2}
   \vspace*{4mm}
}
\renewcommand{\cite}[1]{[#1]}
\def\beginrefs{\begin{list}%
        {[\arabic{equation}]}{\usecounter{equation}
         \setlength{\leftmargin}{2.0truecm}\setlength{\labelsep}{0.4truecm}%
         \setlength{\labelwidth}{1.6truecm}}}
\def\endrefs{\end{list}}
\def\bibentry#1{\item[\hbox{[#1]}]}

\newcommand{\fig}[3]{
			\vspace{#2}
			\begin{center}
			Figure \thelecnum.#1:~#3
			\end{center}
	}
% Use these for theorems, lemmas, proofs, etc.
\newtheorem{theorem}{Theorem}[lecnum]
\newtheorem{lemma}[theorem]{Lemma}
\newtheorem{proposition}[theorem]{Proposition}
\newtheorem{claim}[theorem]{Claim}
\newtheorem{corollary}[theorem]{Corollary}
\newtheorem{definition}[theorem]{Definition}
\newtheorem{example}[theorem]{Example}
\newenvironment{proof}{{\bf Proof:}}{\hfill\rule{2mm}{2mm}}

% **** IF YOU WANT TO DEFINE ADDITIONAL MACROS FOR YOURSELF, PUT THEM HERE:

\newcommand\degree{\mbox{deg}}
\newcommand\rd{\mbox{rd}}
\newcommand\opt{\mathsf{opt}}
\newcommand\trace{\mathsf{trace}}
\newcommand\deter{\mathsf{det}}
\newcommand\gaussian{\mathcal{N}}
\newcommand\Pf{\mathcal{P}}
\newcommand\Ecal{\mathcal{E}}
\newcommand\Fcal{\mathcal{F}}
\newcommand\Hcal{\mathcal{H}}
\newcommand\E{\mathbb{E}}
\newcommand\F{\mathbb{F}}
\newcommand\R{\mathbb{R}}
\newcommand\N{\mathbb{N}}

\newcommand\ClassP{\mathsf{P}}
\newcommand\ClassNP{\mathsf{NP}}
\newcommand\ClasscoNP{\mathsf{coNP}}
\newcommand\SAT{\mathsf{SAT}}
\newcommand\MaxSAT{\mathsf{MaxSAT}}
\newcommand\MaxCUT{\mathsf{MaxCUT}}
\newcommand\SDP{\mathsf{SDP}}

\DeclarePairedDelimiter\ceil{\lceil}{\rceil}
\DeclarePairedDelimiter\floor{\lfloor}{\rfloor}
\DeclarePairedDelimiter\anglebrac{\langle}{\rangle}
\DeclarePairedDelimiter\norm{\parallel}{\parallel} % Input the file specifying the document layout and structure

%----------------------------------------------------------------------------------------
%	ARTICLE INFORMATION
%----------------------------------------------------------------------------------------

\articletitle{How to Elect a Leader Faster than a Tournament} % The title of the article
\articlecitation{\cite{alistarh2015elect}} % The BibTeX citation key from your bibliography

\datenotesstarted{November 21, 2017} % The date when these notes were first made
\docdate{\datenotesstarted; rev. \today} % The date when the notes were lasted updated (automatically the current date)

\docauthor{Lily Li} % Your name

%----------------------------------------------------------------------------------------

\begin{document}

\pagestyle{myheadings} % Use custom headers
\markright{\doctitle} % Place the article information into the header

%----------------------------------------------------------------------------------------
%	PRINT ARTICLE INFORMATION
%----------------------------------------------------------------------------------------

\thispagestyle{plain} % Plain formatting on the first page

\printtitle % Print the title

%----------------------------------------------------------------------------------------
%	ARTICLE NOTES
%----------------------------------------------------------------------------------------

\section{Introduction}
\paragraph{Problem Definition.} An algorithm which solves leader election among $n$ processes must satisfy: (1) termination - every process $p_i$ must eventually output $win$ or $lose$ and (2) unique winner - there is exactly one process which outputs $win$. Further, no process may lose before the eventual winner starts its execution and the algorithm must be linearizable; the first operation is $win$ and all subsequent operations is $lose$. In-class, we have a seen an leader election algorithm which builds a \emph{tournament tree} with $n$ leaves. The step and message complexities of this decades old algorithm are $\Theta(\log n)$ and $\Theta(n^2\log n)$ respectively. 

\paragraph{Model.} We have a complete asynchronous message passing system with at most $f < \ceil{n/2}$ faulty processes. Further we assume a strong adversary which can  examine the systems state and the outcome of random coin flips adjusting the schedules of steps, message delivery events, and failures.  

\paragraph{Results.} In this work by Alistarh et al. presented at PODC 2015, the authors present a leader election algorithm which requires $O(\log^*n)$ steps by each process and $O(n^2)$ point-to-point messages. They pair this with a lower bound of $\Omega(n^2)$ message, proving optimal message complexity.

%------------------------------------------------

\section{Leader Election Algorithm}
In the code we will make use of the high-level protocol $\communicate$. Let $p$ be a procedure which consists of broadcasting a message and waiting for a response. $\communicate\anglebrac{p}$ will wait for at least $\ceil{n/2}+1$ responses before proceeding (this is reminiscent to what we did when we tried to simulate single-reader single-writer registers in a message passing system). We call a set of $\ceil{n/2}+1$ processes a \textbf{quorum} since these arise often in our analysis. A process can execute the following operations:
\begin{enumerate}
\item[$\broadcast(v)$:] process $p_i$ broadcasts value $v$ to all other processes (including itself). If process $p_j$ receives $\broadcast{v}$ from $p_i$, it updates $S_j[i] \leftarrow v$.
\item[$\collect()$:] process $p_i$ sends a collect message to all other processes (including itself). If process $p_j$ receives $\collect()$ from $p_i$, it sends message $\anglebrac{S_j}$ to $p_i$. 
\item[$\random(p)$:] (local operation) process $p_i$ receives $1$ with probability $p$ and $0$ otherwise. 
\end{enumerate}

What we want is an algorithm which iteratively reduces the number of participating processes. At each round processes flip a random coin and output $lose$ or $survive$ depending on the result. Surviving processes will try again in the next round. A na\"{i}ve implementation would fail since the adversary can schedule the ``losing'' processes first.

\subsection{Poison Pill (Homogeneous)}
See Algorithm \ref{pseudocode:homPP}. Process $p_i$ has two local variables: a vector $S_i$ of length $n$ which record the states of processes (with all entries initially set to $w$) and an $n\times n$ matrix $V_i$ storing the view as seen by other processes (initially empty). States can be any of $w$ (waiting), $c$ (commited), $0$ (lose), and $1$ (survive). A process $p_j$ is \textbf{active} from the perspective of $p_i$ if column $j$ in $V_i$ contains at least one entry of status $1$ or $c$ and no entries of status $0$.  
\begin{algorithm}[ht]
	\caption{Homogeneous Poison Pill: code for process $p_i$.}
    \label{pseudocode:homPP}
    \begin{algorithmic}[1]
    \State Initialize $S_i[1...n] = [w, ... ,w]$ and $V_i[1...n][1...n] = \emptyset$
	\State $S_i[i] \leftarrow c$
	\State $\communicate\anglebrac{\broadcast(S[i])}$
	\State $S_i[i] \leftarrow \random\left(\frac{1}{\sqrt{n}}\right)$
	\State $\communicate\anglebrac{\broadcast(S_i[i])}$
	\State $V_i \leftarrow \communicate\anglebrac{\collect()}$
	\If{$S_i[i] = 0$ and $\exists k:$ $p_k$ is active}
		\State return $0$
	\EndIf
	\State return $1$ 
	\end{algorithmic}
\end{algorithm}

\begin{claim}
If all processes return then some process outputs $1$.
\end{claim}
\begin{proof}
Suppose for a contradiction that all processes output $0$. If a process receives $1$ upon executing $\random\left(\frac{1}{\sqrt{n}}\right)$, then it must output $1$ so all processes received $0$. Let process $p_i$ be the last to complete the $\broadcast$ operation on Line 5. When $p_i$'s broadcast operation finishes, the $0$ of every process gets stored by a quorum. Consider $V_i$ at the completion of Line 6. Every column must have one $0$-entry since the set of processes that stored $0$ must overlap with the set of processes which successfully sent their status vector to $p_i$. Thus $p_i$ must output $1$. 
\end{proof}

\begin{claim}
The expected number of processes that return $1$ is $O(\sqrt{n})$.
\end{claim}
\begin{proof}
Suppose $p_i$ received $1$ from its execution of $\random$ at $t_i$. Then all process which received $0$ at any time $\geq t_i$ must output $0$. Consider such a process $p_j$. Observe that at $t_i$, $S_i[i] = \top$ has been broadcast to a quorum. Thus the set of processes which received $S_i[i]$ and the set of processes which sends their status to $p_j$ upon its execution of $\collect()$ overlaps. $p_j$ will see that $p_i$ has $\top$ or $1$ and will return $0$. Now simply observe that the expected number of $1$s as well as the expected index for the first $1$ are both $\sqrt{n}$.  
\end{proof}

\subsection{Poison Pill (Heterogeneous)}
It can be shown that there will be $\Omega(\sqrt{n})$ survivors at every round of the homogeneous algorithm --- the probability of getting $1$ is fixed. To improve this bound we need to change this probability. 
\begin{algorithm}
	\caption{Heterogeneous Poison Pill: code for process $p_i$.}
    \label{pseudocode:hetPP}
    \begin{algorithmic}[1]
	\State $S_i[i] \leftarrow \{c, \{\}\}$
	\State $\communicate\anglebrac{broadcast(S_i[i])}$
	\State $V_i \leftarrow \communicate\anglebrac{\collect()}$
	\State $t \leftarrow \{j: \exists k: V_i[k][j] \neq \bot\}$
	\State $p \leftarrow \frac{\log |t|}{|t|}$ \# $p \leftarrow 1$ if $|t| = 1$
	\State $S_i[i] \leftarrow \left\{\random(p), t\right\}$
	\State $\communicate\anglebrac{\broadcast(S_i[i])}$
	\State $V_i \leftarrow \communicate\anglebrac{\collect()}$
	\If{$S_i[i][0] = 1$ and $\exists k:$ $p_k$ is active}
		\State return $0$
	\Else
		\State return $1$
	\EndIf
	\State return $1$
	\end{algorithmic}
\end{algorithm}

\subsection{Analysis}
Now we just need to show that the expected number of processes which survive using the heterogeneous poison pill algorithm is $\Theta(\log n)$. 

\begin{claim}
\label{claim:closure} Let $S$ be the set of processes which flipped $0$ and survived and let $U$ be the union of all the elements that they found not waiting. Then for $i \in U$, every $j \in t_i$ is in $U$.   
\end{claim}
\begin{proof}
Intuitively this should make sense as $U$ is pretty comprehensive (we are taking the union of a bunch of stuff). Since a process $p_i$ which flipped zero can only survive if is sees only other processes which lost, every $p \in U$ obtained $0$. There is some entry of the view $V_i$ corresponding to $p$ which has $0$. Further it also has the list associated with $p$. This list is taken into account during the union thus $p_i$ see all the elements in the list $t$ of $p$.
\end{proof}

\begin{claim}
\label{claim:reproc}
If processor $q$ completed the first broadcast call no later than $p$ completed its first propagate call, then $q$ will be included in the $l$ list of $p$.
\end{claim}

\begin{claim}
\label{claim:mainflip0}
The maximum expected number of processors that flip $0$ and survive is $O(\log n) + O(1)$.
\end{claim}
\begin{proof}
Let $S$ be the set of $z$ processes. And $U$ be defined as above. By Claim \ref{claim:reproc} and Claim \ref{claim:closure}, it must be the case that if $p \in U$ and $q$ finished the first broadcast no later than $p$, then $q \in U$. Notice that if we order the processes by the time they finish the first broadcast, all processes in $U$ must come before the processes not in $U$. 

Since $U$ forms a closed set and all processes in $U$ flipped $0$, the probability that each process flipped $0$ is at most $\left(1 - \frac{\log |U|}{|U|}\right)$. The probability for all processes in $U$ to flip $0$ is $\left(1 - \frac{\log |U|}{|U|}\right)^{|U|} = O\left(\frac{1}{|U|}\right)$. Since $z \in U$, This is $O\left(\frac{1}{z}\right)$.  
\end{proof}

\begin{claim}
\label{claim:mainflip1}
The maximum expected number of processors that flip $0$ and survive is $O(\log^2 n) + O(1)$.
\end{claim}
\begin{proof}
Order the processes according to the time they finish their first broadcast. By property \ref{claim:reproc}, the process ordered first has $|t| \geq 1$, the process ordered second has $|t| \geq 2$ and so on. Since the probability of flipping $1$ decreases as $|t|$ increases, the best expectation achievable is
\[1 + \sum_{l = 2}^{n} \frac{log|t|}{|t|} \in O(\log^2 n).\]
\end{proof}
%------------------------------------------------

%\section{Message Complexity Lower Bound}

%----------------------------------------------------------------------------------------
%	BIBLIOGRAPHY
%----------------------------------------------------------------------------------------

\renewcommand{\refname}{Reference} % Change the default bibliography title

\bibliography{bibliography} % Input your bibliography file

%----------------------------------------------------------------------------------------

\end{document}