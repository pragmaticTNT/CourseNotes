% This is a borrowed LaTeX template file for lecture notes for CS267,
% Applications of Parallel Computing, UCBerkeley EECS Department.
% Now being used for my class notes.

\documentclass[twoside]{article}
\setlength{\oddsidemargin}{0.25 in}
\setlength{\evensidemargin}{-0.25 in}
\setlength{\topmargin}{-0.6 in}
\setlength{\textwidth}{6.5 in}
\setlength{\textheight}{8.5 in}
\setlength{\headsep}{0.75 in}
\setlength{\parindent}{0 in}
\setlength{\parskip}{0.1 in}

%
% ADD PACKAGES here:
%

\usepackage{amsmath,amsfonts,amssymb, graphicx}
\usepackage{color}
\usepackage{comment}
\usepackage{mathtools}

%
% The following commands set up the lecnum (lecture number)
% counter and make various numbering schemes work relative
% to the lecture number.
%
\newcounter{lecnum}
\renewcommand{\thepage}{\thelecnum-\arabic{page}}
\renewcommand{\thesection}{\thelecnum.\arabic{section}}
\renewcommand{\theequation}{\thelecnum.\arabic{equation}}
\renewcommand{\thefigure}{\thelecnum.\arabic{figure}}
\renewcommand{\thetable}{\thelecnum.\arabic{table}}

%
% The following macro is used to generate the header.
%
\newcommand{\lecture}[4]{
   \pagestyle{myheadings}
   \thispagestyle{plain}
   \newpage
   \setcounter{lecnum}{#1}
   \setcounter{page}{1}
   \noindent
   \begin{center}
   \framebox{
      \vbox{\vspace{2mm}
    \hbox to 6.28in { {\bf CMPT 409: Theoretical Computer Science
	\hfill Summer 2017} }
       \vspace{4mm}
       \hbox to 6.28in { {\Large \hfill Lecture #1: #2  \hfill} }
       \vspace{2mm}
       \hbox to 6.28in { {\it Lecturer: #3 \hfill Scribe: #4} }
      \vspace{2mm}}
   }
   \end{center}
   \markboth{Lecture #1: #2}{Lecture #1: #2}

   %{\bf Note}: {\it LaTeX template courtesy of UC Berkeley EECS dept.}

   %{\bf Disclaimer}: {\it These notes have not been subjected to the
   %usual scrutiny reserved for formal publications.  They may be distributed
   %outside this class only with the permission of the Instructor.}
   \vspace*{4mm}
}

%Use this command for a figure; it puts a figure in wherever you want it.
%usage: \fig{NUMBER}{SPACE-IN-INCHES}{CAPTION}
\newcommand{\fig}[3]{
			\vspace{#2}
			\begin{center}
			Figure \thelecnum.#1:~#3
			\end{center}
	}
% Use these for theorems, lemmas, proofs, etc.
\newtheorem{theorem}{Theorem}[lecnum]
\newtheorem{lemma}[theorem]{Lemma}
\newtheorem{proposition}[theorem]{Proposition}
\newtheorem{claim}[theorem]{Claim}
\newtheorem{corollary}[theorem]{Corollary}
\newtheorem{definition}[theorem]{Definition}
\newtheorem{example}[theorem]{Example}
\newenvironment{proof}{{\bf Proof:}}{\hfill\rule{2mm}{2mm}}

% **** IF YOU WANT TO DEFINE ADDITIONAL MACROS FOR YOURSELF, PUT THEM HERE:

\newcommand\E{\mathbb{E}}
\def\N{\mathbb{N}}
\def\Z{\mathbb{Z}}
\def\Q{\mathbb{Q}}
\def\R{\mathbb{R}}
\def\C{\mathbb{C}}
\def\F{\mathbb{F}}
\def\L{\mathcal{L}}
\def\M{\mathcal{M}}
\def\Next{\mathsf{Next}}
\def\Dom{\mbox{Dom}}
\def\Nex{\mathsf{Nex}}
\def\Succ{\mathsf{Succ}}
\def\Zero{\mathsf{Zero}}
\def\Jump{\mathsf{Jump}}
\def\Command{\mathsf{Command}}
\def\Prog{\mathsf{Prog}}
\def\halt{\mathsf{halt}}

\DeclarePairedDelimiter\ceil{\lceil}{\rceil}
\DeclarePairedDelimiter\floor{\lfloor}{\rfloor}
\DeclarePairedDelimiter\anglebrac{\langle}{\rangle}

\begin{document}
%\lecture{**LECTURE-NUMBER**}{**DATE**}{**LECTURER**}{**SCRIBE**}
\lecture{6}{Incompleteness (13 July - 4 Aug)}{Ternovska, Eugenia}{Lily Li}
%\footnotetext{These notes are partially based on those of Nigel Mansell.}

\section{G\"{o}del Incompleteness Theorem}

\begin{definition}
\textbf{True Arithmetic} (TA) is the set of all sentence in 
\[\delta_A = [0, s, + , \cdot, =] \]
that are true in the standard model $\underline{\N}$.

Notation: $s_0 = 0$ and $s_{k+1} = ss_{k}$ for all $k = 0, 1...$ so $s_3 = 3$. $s_k$ is a syntactic object which represent semantic object (the number $k$). Further $A(s_{\vec{a}})$ means $A(s_{a_1}, ..., s_{a_n})$. 
\end{definition}

\begin{definition}
If $R$ is an $n$-ary relation $A(\vec{x})$ is a formula such that all free variables among $x_1, ..., x_n$. Then $A(\vec{x})$ \textbf{represents} $R$ if and only if $\forall \vec{a} \in \N^{n}$
\[R(\vec{a}) \iff \underline{\N} \vdash A(s_{\vec{a}})\]
\end{definition}

\begin{definition}
A relation $R$ is \textbf{arithmetical} if and only if $R$ is representable by some formula (in the vocabulary of $\delta_A$).
\end{definition}

Consider some arithmetical relations
\begin{enumerate}
\item
\end{enumerate}

\begin{definition}

Let $\Delta_0$ be the set of all bounded formulas.

$R(\vec{x})$ is a \textbf{$\Delta_0$ relation} if and only if some $\Delta_0$ formula $A$ represents $R$. This implies what about $\Delta_0$ relations and arithmetical?
\end{definition}

\begin{lemma}
The $\Delta_0$ relations are closed under $\land, \lor, \lnot$ and bounded $\forall \leq$, $\exists \leq$. 
\end{lemma}
\begin{proof}

\end{proof}

\begin{lemma}
Every $\Delta_0$ relation is PR (though the converse is not true).
\end{lemma}
\begin{proof}
In the
\end{proof}

%%===> LATER
\begin{theorem}
TA is not a recursive set (not even RE) and does not have a recursive set of axioms.
\end{theorem}
\begin{proof}

\end{proof}

Since TA is so unwieldy, we will consider instead the subset of Peano Arithmetics (PA). 

\begin{theorem}
(G\"{o}del's Second Incompleteness Theorem) the consistency of PA cannot be proved in PA. 
\end{theorem}
\begin{proof}

\end{proof}

Even though Peano arithmetics is incomplete, a simpler arithmetics (Presburger Arithmetics, containing only the $+$ operation) is decidable. 
\end{document}