%
% This is a borrowed LaTeX template file for lecture notes for CS267,
% Applications of Parallel Computing, UCBerkeley EECS Department.
% Now being used for CMU's 10725 Fall 2012 Optimization course
% taught by Geoff Gordon and Ryan Tibshirani.  When preparing 
% LaTeX notes for this class, please use this template.
%
% To familiarize yourself with this template, the body contains
% some examples of its use.  Look them over.  Then you can
% run LaTeX on this file.  After you have LaTeXed this file then
% you can look over the result either by printing it out with
% dvips or using xdvi. "pdflatex template.tex" should also work.
%

\documentclass[twoside]{article}
\setlength{\oddsidemargin}{0.25 in}
\setlength{\evensidemargin}{-0.25 in}
\setlength{\topmargin}{-0.6 in}
\setlength{\textwidth}{6.5 in}
\setlength{\textheight}{8.5 in}
\setlength{\headsep}{0.75 in}
\setlength{\parindent}{0 in}
\setlength{\parskip}{0.1 in}

%
% ADD PACKAGES here:
%

\usepackage{amsmath,amsfonts,amssymb, graphicx}
\usepackage{color}
\usepackage{comment}
\usepackage{mathtools}

%
% The following commands set up the lecnum (lecture number)
% counter and make various numbering schemes work relative
% to the lecture number.
%
\newcounter{lecnum}
\renewcommand{\thepage}{\thelecnum-\arabic{page}}
\renewcommand{\thesection}{\thelecnum.\arabic{section}}
\renewcommand{\theequation}{\thelecnum.\arabic{equation}}
\renewcommand{\thefigure}{\thelecnum.\arabic{figure}}
\renewcommand{\thetable}{\thelecnum.\arabic{table}}

%
% The following macro is used to generate the header.
%
\newcommand{\lecture}[4]{
   \pagestyle{myheadings}
   \thispagestyle{plain}
   \newpage
   \setcounter{lecnum}{#1}
   \setcounter{page}{1}
   \noindent
   \begin{center}
   \framebox{
      \vbox{\vspace{2mm}
    \hbox to 6.28in { {\bf CMPT 409: Theoretical Computer Science
	\hfill Summer 2017} }
       \vspace{4mm}
       \hbox to 6.28in { {\Large \hfill Lecture #1: #2  \hfill} }
       \vspace{2mm}
       \hbox to 6.28in { {\it Lecturer: #3 \hfill Scribe: #4} }
      \vspace{2mm}}
   }
   \end{center}
   \markboth{Lecture #1: #2}{Lecture #1: #2}

   %{\bf Note}: {\it LaTeX template courtesy of UC Berkeley EECS dept.}

   %{\bf Disclaimer}: {\it These notes have not been subjected to the
   %usual scrutiny reserved for formal publications.  They may be distributed
   %outside this class only with the permission of the Instructor.}
   \vspace*{4mm}
}
%
% Convention for citations is authors' initials followed by the year.
% For example, to cite a paper by Leighton and Maggs you would type
% \cite{LM89}, and to cite a paper by Strassen you would type \cite{S69}.
% (To avoid bibliography problems, for now we redefine the \cite command.)
% Also commands that create a suitable format for the reference list.
\renewcommand{\cite}[1]{[#1]}
\def\beginrefs{\begin{list}%
        {[\arabic{equation}]}{\usecounter{equation}
         \setlength{\leftmargin}{2.0truecm}\setlength{\labelsep}{0.4truecm}%
         \setlength{\labelwidth}{1.6truecm}}}
\def\endrefs{\end{list}}
\def\bibentry#1{\item[\hbox{[#1]}]}

%Use this command for a figure; it puts a figure in wherever you want it.
%usage: \fig{NUMBER}{SPACE-IN-INCHES}{CAPTION}
\newcommand{\fig}[3]{
			\vspace{#2}
			\begin{center}
			Figure \thelecnum.#1:~#3
			\end{center}
	}
% Use these for theorems, lemmas, proofs, etc.
\newtheorem{theorem}{Theorem}[lecnum]
\newtheorem{lemma}[theorem]{Lemma}
\newtheorem{proposition}[theorem]{Proposition}
\newtheorem{claim}[theorem]{Claim}
\newtheorem{corollary}[theorem]{Corollary}
\newtheorem{definition}[theorem]{Definition}
\newtheorem{example}[theorem]{Example}
\newenvironment{proof}{{\bf Proof:}}{\hfill\rule{2mm}{2mm}}

% **** IF YOU WANT TO DEFINE ADDITIONAL MACROS FOR YOURSELF, PUT THEM HERE:

\newcommand\E{\mathbb{E}}
\def\N{\mathbb{N}}
\def\Z{\mathbb{Z}}
\def\Q{\mathbb{Q}}
\def\R{\mathbb{R}}
\def\C{\mathbb{C}}
\def\F{\mathbb{F}}
\def\L{\mathcal{L}}
\def\M{\mathcal{M}}
\def\Next{\mathsf{Next}}
\def\Dom{\mbox{Dom}}
\def\Nex{\mathsf{Nex}}
\def\Succ{\mathsf{Succ}}
\def\Zero{\mathsf{Zero}}
\def\Jump{\mathsf{Jump}}
\def\Command{\mathsf{Command}}
\def\Prog{\mathsf{Prog}}

\DeclarePairedDelimiter\ceil{\lceil}{\rceil}
\DeclarePairedDelimiter\floor{\lfloor}{\rfloor}
\DeclarePairedDelimiter\anglebrac{\langle}{\rangle}

\begin{document}
%\lecture{**LECTURE-NUMBER**}{**DATE**}{**LECTURER**}{**SCRIBE**}
\lecture{5}{Computability (26 - 30 June)}{Ternovska, Eugenia}{Lily Li}
%\footnotetext{These notes are partially based on those of Nigel Mansell.}

\section{Recursive Functions}
\begin{definition}
Minimization is another operations we need for recursion (not primitive recursion).
\[f(\bar{x}) = \mu y[g(\bar{x}, y) = 0] \iff \mbox{$f(\bar{x})$ is the least $b$ s.t. $g(\bar{x}, b) = 0$ and $g(\bar{x}, i) \neq \infty$ for $i < b$} \]
$f(\bar{x}) = \infty$ if no such $b$ exists.

\emph{Note:} here $0$ represents true, while $1$ represents false. 

If $g$ is computable then so is $f$ since we can calculate $g(\bar{x}, i)$ for increasingly large $i$ starting from $0$. 
\end{definition}

\begin{definition}
$f$ is recursive if and only $f$ can be obtained from the initial functions from finitely many applications of primitive recursion, composition, and minimization.
\end{definition}
Observe that P.R. function is recursive, and all recursive function are computable (since we can apply induction on the number of primitive recursion, composition and minimizations used).

We want to show the converse (below), but first we need some background.
\begin{theorem}
All computable functions (those computable by a RM) are recursive.
\end{theorem}

In fact, any function computable in exponential time is P.R. More generally if $f$ can be computed in time $T(x) = O(A_m(x))$ for some fixed $m$, where $A_m$ is the Ackermann's function, then $f$ is P.R.

Let us see some examples of how we define function using our basic tools of primitive recursion, complementation, and minimization (mostly just p.r.).
\begin{example}
First we define the predecessor function $\rho$, taking care that the range is within $\N$. 
\[\rho(x) = \begin{cases}
x - 1 &\qquad \mbox{ if } x > 0 \\
0 &\qquad \mbox{ if } x = 0
\end{cases}\]
We can define $\rho$ by P.R. as $\rho (0) = 0 = g$ and $\rho(x+1) = x = h(x, \rho(x))$ where $g = Z$ and $h = I_{2,1}$. 

Next we define the limited subtraction function $\Delta$:
\[f_{\Delta}(x,y) = x \Delta y = \begin{cases}
x - y &\qquad \mbox{ if } y \leq x \\
0 &\qquad \mbox{ otherwise } 
\end{cases}\]
Define $f_{\Delta}(x, 0) = x = g(x)$ where $g(x) = I_{1,1}(x)$. Further define $f_{\Delta}(x, y+1) = h(x,y,x \Delta y) = \rho(x \Delta y)$ where $\rho$ is the predecessor function from before and $h = \rho(I_{3,3})$. 

Finally we define the maximum function using our limited subtraction in $\N \times \N \rightarrow \N$. Define
\[\max (x,y) = (x \Delta y) + y\]
Observe that if $x > y$ then $\max (x,y) = x$ and if $x \leq y$ then $\max (x,y) = y$ as one would expect. Since both the subtraction and addition functions are P.R., their composition is P.R. as well.
\end{example}

\section{Relations and 0-1 Functions}
For $R \subseteq \N^n$, $n$-ary relation.
Define the characteristic function of $R$ as:
\[R(\bar{x}) = \begin{cases}
0 &\qquad \mbox{ if } \bar{x} \in R \\
1 &\qquad \mbox{ if } \bar{x} \notin R
\end{cases} \]
Again note $0$ is true and $1$ is false for this part of the course. We consider $R$ computable (resp. P.R.) if and only if the characteristic function is computable (resp. P.R.).

Next we will try to obtain logic using primitive recursive functions. But before we do that lets define a useful P.R. function sign, $\bar{sg} = 1\Delta x$ i.e.
\[\bar{sg}(x) = \begin{cases}
0 &\mbox{ if } x > 0 \\
1 &\mbox{ if } x = 0
\end{cases}\]

\begin{proposition}
If $R$ and $S$ are P.R. (resp. computable) then so are
\begin{enumerate}
\item $\lnot R$: $(\lnot R)(\bar{x}) = \bar{sg}(R(x))$.
\item $R \lor S$: $(R \lor S)(\bar{x}) = R(\bar{x}) \cdot S(\bar{x})$.
\item $R \land S$: $(R \land S)(\bar{x}) = \lnot(\lnot R \lor \lnot S)(\bar{x})$.
\end{enumerate}
\end{proposition}

MOAR operations which preserves recursive functions and relations (as well as preserving computable relations and computable functions).

\begin{example}
\textbf{Relations:}
\begin{align*}
(x < y) &= \bar{sg}\left(y \Delta x\right) = \begin{cases} 
			0 &\mbox{ if } x < y \\
			1 &\mbox{ if } x \geq y
		\end{cases} \\
(x = y) &= \lnot (x < y) \land \lnot (y < x) \\
(x \leq y) &= (x < y) \lor (x = y) 
\end{align*}
\end{example}

\begin{example}
\textbf{Bounded Sum}
\[g(\bar{x}, y) = \sum_{z < y}f(\bar{x}, z) = f(\bar{x}, 0) + \cdots + f(\bar{x}, y-1)\]
$g$ can be defined from $f$ by P.R. as: $g(\bar{x}, 0) = 0$ and $g(\bar{x}, y+1) = g(\bar{x}, y) + f(\bar{x},y)$.
\end{example}

\begin{example}
\textbf{Bounded Product}
\[h(\bar{x},y) = \prod_{z<y} f(\bar{x},z) = f(\bar{x},0) \cdots f(\bar{x},y-1).\]
Similarly, $h$ can be defined from $f$ by P.R. as $h(\bar{x}, 0) = 1 = g$ and $h(\bar{x}, y+1) = h(\bar{x},y) \cdot f(\bar{x}, y)$.
\end{example}

\begin{example}
\textbf{Bounded Quantification}
\begin{align*}
S(\bar{x},y) &= \exists z< y R(\bar{x}, z) \\
&= \prod_{z < y} R(\bar{x}, z) = R(\bar{x}, 0) \cdots R(\bar{x}, y-1) \\
&= h(\bar{x}, y)
\end{align*}
Similarly, the universal quantifier is P.R. since
\begin{align*}
T(\bar{x}, y) &= \forall z < y R(\bar{x},z) \\
&= \lnot \exists z < y \lnot R(\bar{x},z)
\end{align*}
Thus the computability of $R$ implies the computability of $S$ and $T$.
\end{example}

\begin{example}
\textbf{Number Theoretical Functions}
\begin{align*}
\mathsf{Prime}(x) &= 	\begin{cases}
							0 \mbox{ if $x$ is prime }
							1 \mbox{ otherwise }						
						\end{cases}\\
&= (1 < x) \land (\forall z < x: \lnot (z | x) \lor z = 1)
\end{align*}
where 
\[x | y = \exists z \leq y: x \cdot z = y \]
\end{example}

\begin{definition}
We will define a conditional by cases:
\begin{align*}
f(\bar{x}) &= 	\begin{cases}
					g(\bar{x}) \mbox{ if } R(\bar{x}) \\
					h(\bar{x}) \mbox{ otherwise }
				\end{cases}\\
&= \mbox{Cond}(R(\bar{x}), g(\bar{x}), h(\bar{x}))\\
&= \bar{sg}(x)\cdot y + \bar{sg}(\bar{sg}(x)) \cdot z
\end{align*}
Where 
\[\mbox{Cond}(x,y,z) = \begin{cases}
y &\mbox{ if } x = 0 \\
z &\mbox{ if } x > 0
\end{cases}\]
As before, if $g, h, R$ are P.R. (resp. computable) then so is $f$.			
\end{definition}

\begin{example}
\textbf{Bounded Minimization}
\[f(\bar{x}, y) = \min z < y R(\bar{x}, z) = \begin{cases}
\mbox{ least $z$ s.t. $z < y$ and $f$} &\mbox{ if such $z$ exists }\\
y &\mbox{ otherwise }
\end{cases}\]
Notice that (1) $f$ is always total since $R$ is a total $0-1$ valued function, and (2) $f$ is P.R. (resp. computable) whenever $R$ is P.R. (resp. computable) since 
\[f(\bar{x}, y) = \sum_{z<y} \left(\exists v \leq z: R(\bar{x}, v) \right)\]

The idea is to use bounded summation as bounded minimization.
\end{example}

\section{Simulating RM with Recursive Function}
We want to consider all computable function (those computable by RM) and show that these can be expressed recursively. Consider 
\[T([P], \bar{x}, y) \iff y \mbox{ is a halting computation of program $P$ on input $\bar{x}$ }\]
where $[P]$ is the encoding of program $P$. If $T$ is P.R. then the function $\min_y: T([P],\bar{x},y)$ is recursive and so the function associated with $P$ is recursive. In the following we construct $T$ and show that it is P.R.

\section{G\"{o}del Numbering}
We wish to encode the follow: commands of RMs, programs (sequences of commands), states, computations (sequences of states) as natural numbers. To do this, we must first define the $i^{\mbox{th}}$ prime function $f(i) = p_i$ and note that $f(i)$ is P.R. Let
\begin{align*}
p_0 &= f(0) &= 2 \\
p_{x+1} &= f(x+1) &= \min y < p_x^{p_x}
\end{align*}
This means that $p_x < y$ and $\mathsf{Prime}(y)$ so $f(x+1)$ is the first prime after $p_x$.

Next we need to observe that prime decomposition is also P.R. Notation: $z_x$ is the exponent of $p_x$ in the prime decomposition of $z$ where
\[z = p_0^{z_0} \cdot p_1^{z_1} \cdots p_m^{z_m}\]
Thus we can write $z_x$ as
\[\min y < z: \lnot (p_x^{y+1}| z)\]

Finally we need length $l(z)$ to be P.R. Length is defined as follows:
\[l(z) = m + 1 \qquad \mbox{ where } z = p_0^{z_0} \cdots p_m^{z_m}.\]
Alternatively, we can encode $l(z)$ as
\begin{align*}
l(z) &= \min y < z : \prod_{x < y} P_x^{z_x} = z\\
&= 1 + \mbox{ subscript of largets prime which divides $z$}\\
&= \max y < z: (p_y | z) + 1
\end{align*}

\subsection{Numbering Programs}
Using the three P.R functions above, we can number programs as follows. If $P = \anglebrac{c_0, ..., c_{n-1}}$ then $\#(P) = p_0^{\#(c_0)} \cdots p_m^{\#(c_{n-1})}$ where 
\begin{align*}
\#(z_i) &= 2^i \\
\#(s_i) &= 3^i \\
\#(J_{i,j,k}) &= 5^i 7^j 11^k
\end{align*}
Thus distinct programs get distinct numbers and the encoding $\# P$ of $P$ is P.R.

\subsection{Encoding Relations}
\begin{align*}
\Succ(x,i) &\iff x = \#(s_i)\\
\Zero(x,i) &\iff x = \#(z_i)\\
\Jump(x,i,j,k) &\iff x = \#(J_{i,j,k})\\
\Command(x) &\iff \exists i < x: (\Succ(x, i) \lor \Zero(x,i)) \lor (\exists i < x, \exists j < x, \exists k < x: \Jump(x,i,j,k))
\end{align*}

Let $\Prog (z) \iff z = \# (P)$ for some program $P$. $\Prog (z)$ is P.R since 
\begin{align*}
\Prog (z) &\iff z \mbox{ encodes a sequence of commands }\\
&\iff \forall j < l(z): \Command (z_j)
\end{align*}

\subsection{States}
Consider RM program $P$, which uses registers $R_1, ..., R_m$. A state for $P$ is $s = \anglebrac{k, R_1, ..., R_m}$ where $k$ is the index of the next command to execute and $R_1, ..., R_m$ is the value in each register.

Next we consider computations. Define
\[\hat{z} = \begin{cases}
z \mbox{ if $Prog(z)$ encodes a program } \\
1 \mbox{ otherwise }
\end{cases}\]
And for our use define
\[\{z\} = \begin{cases}

\end{cases}\]

$\Nex(u,z) = $ the code of the state $u'$ which results when program $\{z\}$ executes one step from the state encoded by $u$. If $u$ is a halting state $u' = u$. We need to show that $\Nex$ is P.R.

\section{Kleene $T$ Predicate}
Recall $T(P,\bar{x},y)$ from above. Using G\"{o}del encoding we can rewrite $T$ as $T_n(z, x_1, ..., x_n, y) = y$ encodes the computation of $\{z\}$ on input $\bar{x} = (x_1, ..., x_n)$ where $T_{n}$ is a $n+2$-ary relation. 
\end{document}